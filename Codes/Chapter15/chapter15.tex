%fix cor comment:  In terms of referencing: personally I disagree with referring to Gradstein et al., 2004 for the time scale of a period. For example, I refer to Ogg et al., 2004 (in Gradstein et al., 2004 in the actual reference) for the Cretaceous time scale. And to Lourens et al. for the Neogene time scale. I find that more appropriate, since otherwise Grastein gets all the citations, while the original authors who did all the work to make a particular time scale get few or none. 



\chapter{The GPTS and magnetostratigraphy}


The geological time scale is a list of ordered events placed in a temporal/spatial context.  Time  is broken into Eons (e.g., Phanerozoic, Proterozoic), Eras (e.g., Mesozoic, Cenozoic),  Periods (e.g., Cretaceus, Paleogene), Series (e.g, Oligocene, Miocene) and Stages (e.g, Messinian, Zanclean).   The fundamental unit, the stage is ideally defined by its base at a particular place and many such  
\index{global standard section and points}
{\it Global Standard Section and Points}, or GSSPs have been identified.  Numerical ages are attached to these stage boundaries by a variety of methods.    Some methods have explicit numerical age control (e.g., from the predictable decay of radioactive isotopes or  variations in climate caused by the relationship of the Earth and the sun), while others have only relative age  information (e.g., the progressive change  of fossil assemblages, or the identification of contemporaneous events in the geological record).   Almost always numerical ages are estimated by correlation, interpolation,  and/or extrapolation.  As such, the geological time scale is a work in constant revision.  The website of the International Stratigraphic Commission at stratigraphy.org has a wealth of information about ages, stages, GSSPs, etc.

One of the important tools in assembling the geological time scale is the  
\index{geomagnetic!polarity time scale}
geomagnetic polarity time scale (GPTS).  
Identification of a particular polarity reversal allows direct correlation of isochronous events between continental and marine sequences, between northern and southern hemispheres and between the Pacific and Atlantic realms.  Apart from the identification of unique ash layers or the very rare geochemical tracers like an iridium spike, there is no better way to tie together the stratigraphic record.  In fact, it is increasingly the case that stages are defined by certain polarity intervals, as opposed to biostratigraphic information (see for example, the definition of the Messinian Stage on the stratigraphy.org web site.)   In this chapter we will review how the modern GPTS was constructed and will briefly consider some applications of the GPTS to geological problems.  



\section{Early efforts in defining the GPTS}
\label{sect:gpts}

Scientists discovered reversely magnetized rocks in the early 20$^{th}$ century (see Chapter 14), and some suspected that there was a globally synchronous pattern of polarity reversals (e.g., 
\index{Matuyama, M.}
Matuyama, 1929). \nocite{matuyama29}  However, it was not until combined  studies of both age (using the newly developed age dating technique using the decay of radioactive potassium to argon) and polarity (from globally distributed lava flows) that the first Geomagnetic Polarity Time Scales (GPTS) began to take shape (Figure~\ref{fig:cox64}; see e.g., 
\index{Cox, A.}
Cox et al., 1963, 1964). \nocite{cox63}   \nocite{cox64}   

\begin{figure}[htb]
\epsfxsize 14cm
\centering \epsffile{EPSfiles/cox64.eps}
\caption{ Magnetic polarities from volcanic units plotted against age as determined by the potassium-argon method.   The first three  long intervals were named after famous geomagnetists.   [Figure redrawn from Cox et al., 1964].}
\label{fig:cox64}
\end{figure} \nocite{cox64}


\index{Cox, A.}
Cox et al. (1964) broke the  polarity sequence   into times of
dominantly normal polarity (i.e., field vector more or less parallel to today's field) and times of dominantly reverse polarity (i.e., field vector more or less antipodal to today's field).
They  called these  time units ``Epochs''  (note that  they are now known as 
\index{chron}
{\it Chrons}).  
The first three epochs were named after some major players in geomagnetism:  B. Brunhes (who first discovered reversely magnetized rocks), 
\index{Matuyama, M.}
M. Matuyama (who first demonstrated that  the reversely magnetized rocks were older than the normal ones),  and F. Gauss (who worked out the first geomagnetic field model).  A fourth was later named after  W. Gilbert (who first realized that the Earth itself was a magnet).   
 \index{Cox, A.}
 Cox et al. (1964) also recognized the existence of shorter intervals which they called ``Events'' (e.g., the Olduvai and Mammoth events in Figure~\ref{fig:cox64}; note that events are now known as
\index{sub-chrons}
 {\it sub-chrons}).   These shorter intervals are traditionally named after the place where they were first documented.  

Time scales constructed in the manner of 
\index{Cox, A.}
Cox and colleagues that pair dates with particular polarity boundaries are necessarily limited by the uncertainty in the dating of young basalts.    In the early 60's this uncertainty exceeded the average duration of 
polarity intervals for times prior to about five million years (except for the very long intervals of a single polarity like the Permian 
\index{Kiaman}
``Kiaman'' interval which lasted over 50 million years).    


\begin{figure}[htb]
\epsfxsize 9cm
\centering \epsffile{EPSfiles/mason61.eps}
\caption{Map of the pattern of magnetic anomalies off northwestern North America.  [Modified from Mason and Raff, 1961.]}
\label{fig:mason61}
\end{figure}



The publication of 
\index{Cox, A.}
Cox et al. (1963) (see also 
\index{McDougall, I.}
\index{Tarling, D.H.}
McDougall and Tarling, 1963), \nocite{mcdougall63}  essentially laid to rest doubts about the validity of geomagnetic reversals and sketched the rudiments of the first GPTS.  Shortly thereafter,  
\index{Vine, F.}
\index{Matthews, D.H.}
Vine and Matthews (1963) \nocite{vine63} put ideas about polarity reversals and the bizarre ``magnetic stripes'' in marine magnetic anomaly data (Figure~\ref{fig:mason61}; e.g., 
\index{Mason, R.G.}
\index{Raff, A.D.}
Mason and Raff, 1961) together \nocite{mason61}  as strong support for the notion of sea floor spreading.   The realization that the marine magnetic anomalies were a record of polarity history meant that  the template for the pattern of reversals could be extended far beyond the resolution of the K-Ar method.     It was not long before such a template for paleomagnetic reversals based on magnetic anomalies (numbering  1 to 31) was proposed (see Figure~\ref{fig:pitman66}, e.g., Pitman and Heirtzler 1966).  By assigning an age of 0 Ma to the ridge crest, an age of 3.35 Ma to the of the Gauss/Gilbert boundary (stars in  Figure~\ref{fig:pitman66}) and assuming constant spreading for the South Atlantic anomaly sequence, 
\index{Heirtzler, J.R.}
Heirtzler et al. (1968) \nocite{heirtzler68} produced a GPTS that extended to about 80 Ma. The age of anomaly 31 was estimated to be about 71.5 Ma.  The truly astounding thing is that the currently accepted age for anomaly 31 is about 68 Ma (e.g.,
\index{Cande, S.C.}
\index{Kent, D.V.}
 Cande and Kent, 1995) \nocite{cande95} a difference of only a few percent!      


\begin{figure}[h!tb]
\epsfxsize 12cm
\centering \epsffile{EPSfiles/pitman66.eps}
\caption{A profile of bathymetry (bottom panel) and magnetic anomalies (labelled ``profile'') obtained from the East Pacific Rise (Eltanin 19 profile, also known as ``the magic profile''.)  The magnetic anomaly profile,  flipped east-to-west is replotted above (labelled ``profile backwards'').  Assuming a magnetization of a 500 m thick section of oceanic crust (black and white pattern above),  a model for the predicted anomalies could be generated (labelled ``model'').   Above is the inferred time scale.  The position of the Gauss/Gilbert boundary is marked by stars.   [Adapted from  Pitman and Heirtzler, 1966.]}
\label{fig:pitman66}
\end{figure}
\eject


In a parallel effort to the marine magnetic anomaly work, several groups were investigating the magnetic stratigraphy of deep sea sediment cores (e.g.,
\index{Harrison, C.G.A.}
 Harrison, 1966 and 
 \index{Opdyke, N.D.}
 Opdyke et al. 1966).   In Figure~\ref{fig:opdyke66} we show the record of inclination versus depth of Opdyke et al., (1966) obtained from  a core taken off the coast of Antarctica.    Upwardly pointing (negative)  inclinations are  normal and positive inclinations are reversely magnetized. This polarity pattern was correlated  to the currently available time scale which included the new ``event'' known as the Jaramillo 
\index{Doell, R.R.}
\index{Dalrymple, G.B.}
(Doell  and Dalrymple, 1966) \nocite{doell66} and revised age estimates for the ``epoch'' boundaries.   \nocite{harrison66} \nocite{opdyke66}  

\begin{figure}[htb]
\epsfxsize 10cm
\centering \epsffile{EPSfiles/opdyke66.eps}
\caption{left: Inclinations from core V16-134 plotted against depth.  middle: The GPTS as it was known in 1966.  Faunal zones of the southern ocean identified within the core.    [Adapted from  Opdyke et al., 1966.]}
\label{fig:opdyke66}
\end{figure}

The polarity sequence from magnetostratigraphic records was extended back into the Miocene by Opdyke et al.  (1974, see Figure~\ref{fig:opdyke74}).  The epochs, defined by the magnetostratigraphy could not easily be correlated to the anomaly data shown in Figure~\ref{fig:pitman66} and the two numbering schemes (anomaly numbers and epoch numbers) remained separate until the correlation between the two was deemed sufficiently robust.  



By the early 70s  the large scale structure of the marine magnetic anomalies had been sketched out.  There was a young set numbered 1-34 which terminated in a vast expanse of oceanic crust with no correlatable anomalies known as the 
\index{Cretaceous quiet zone}
{\it Cretaceous Quiet Zone} or CQZ.   The Cretaceous Quiet Zone is well established as being a period of time with very few (or no!) reversals (see 
\index{Helsley, C.E.}
\index{Steiner, M.B.}
Helsley and Steiner, 1969). \nocite{helsley69}  The CQZ is synonymous with the Cretaceous Normal Superchron, or CNS and extends from the middle of the middle of the Santonian ($\sim$ 84 Ma) to the middle of the Aptian stage ($\sim$ 125 Ma).  
On the old end of the CQZ was another set of anomalies,  known as the 
\index{M-sequence}
{\it M-sequence}   (e.g., 
\index{Larson, R.L.}
\index{Heirtzler, J.R.}
Larson and Heirtzler, 1972).  \nocite{larson72}    These extend from M0 (which bounds the old end of the  CQZ)  to M25 based on marine magnetic anomalies.      Unlike the younger set of anomalies, the M-sequence anomalies are associated with reverse polarity intervals and now stretch from M0 to at least M29.  

Because the oldest sea floor is about 180 Ma and the oldest marine magnetic anomaly sequence is very poorly expressed (it is known as the  
\index{Jurassic quiet zone}
{\it Jurassic Quiet Zone}), polarity intervals older than  about M29 are defined 
 using various magnetostratigraphic  sections obtained from land exposures.   
The M-squence of polarity intervals was extended to about M39 using sections from  Spain and Poland.  The M-sequence has now been fairly firmly tied to geological stages  and thereby calibrated in terms of numerical ages (see e.g., 
\index{Channell, J.E.T.}
Channell et al. 1995).  \nocite{channell95}   Recently, the age estimated by Channell et al. (1995) of about 121 Ma for M0 was firmly established by 
\index{He, H.}
He et al. (2008).  \nocite{he08}    

As we go back farther in time, the GPTS  necessarily becomes  more sketchy.  Long sequences of stratigraphic sections are required with few gaps and reasonably constant sediment accumulation rates.   Such sequences are difficult to identify and piece together so the GPTS will only slowly be completed.   

 One very long part of the GPTS in the middle to late Triassic is, however, quite well known.     Using a series of drill cores with overlapping sections, 
 \index{Kent, D.V.}
 Kent et al. (1995) defined a set of polarity intervals labelled E1 to E23 (see Figure~\ref{fig:newark}).  Kent and 
 \index{Olsen, P.E.}
 Olsen (1999) \nocite{kent95} \nocite{kent99b} interpreted  lithologic cycles within sections as 400 kyr climatic cycles and calibrated their composite depth scale to time.    Their resulting time scale is shown to the right in Figure~\ref{fig:newark}.    
 
Painstakingly acquired overlapping stratigraphic sections will be the basis for future extensions  of the GPTS.  Stay tuned -- this is very much a work in progress and is advancing rapidly.    




\subsection{ A note about terminology}

\index{epoch}
\index{event}
The Epoch/Event terminology was changed to Chron/sub-chron in 1979 by international agreement (Anonymous, 1979).  \nocite{anonymous79}  Along with chrons and sub-chrons, the international subcommission defined ``superchrons''.  
\index{Cande, S.C.}
\index{Kent, D.V.}
Cande and Kent (1992) later defined  \nocite{cande92} 
\index{cryptochron}
{\it cryptochrons}.  Superchrons are extremely long polarity intervals, such as the Kiaman (also known as the  
\index{Permo-Carbaniferous Reverse Superchron}
Permo-Carbaniferous Reverse Superchron or PCRS) which lasted from 298 to 265 Ma \index{Gradstein, F.}
(Gradstein et al., 2004) and the \nocite{gradstein04} Cretaceous Normal Superchron (CNS: 83-121 Ma in 
\index{Gee, J.S.}
\index{Kent, D.V.}
Gee and Kent, 2007). \nocite{gee07}   
Cryptochrons are 
\index{tiny wiggles}
{\it tiny wiggles} in the marine magnetic anomaly record that are too short to be unequivocally interpreted as full reversals (i.e., shorter than about 30 kyr).  Some of these may be related to geomagnetic excursions, microchrons  or just periods of low geomagnetic field strength (see Chapter 14.)     

In an attempt to ``rationalize'' the Neogene chron (event) terminology (which numbered chrons from 5-22) and the anomaly terminology (running from 1 to about 6C),  \index{Cande, S.C.}
\index{Kent, D.V.}
Cande and Kent (1992) \nocite{cande92}  broke the time scale into chrons and sub-chrons based on the anomaly numbering scheme distinguishing chrons from anomalies with the letter ``C''.     The ``C'' stands for ``Chron'' and is meant to distinguish the time unit from the anomaly.   Because the anomaly numbering system only had 34 anomalies from the end of the CNS to the present, many more subdivisions were required, particularly in the very ``busy'' interval between Anomalies 5 and 6.  These anomalies were denoted 5' 5A, 5AA, 5AB and the like.   
For a complete listing of the current GPTS, please refer to Table~\ref{tab:gpts}.     This is the time scale of
\index{Gee, J.S.}
\index{Kent, D.V.}
 Gee and Kent (2007) which is a hybrid of the Cande and Kent (1995) and the 
 \index{Channell, J.E.T.}
 Channell et al. (1995) time scales with the addition of sub-chrons  recognized by 
 \index{Lowrie, W.}
 \index{Kent, D.V.}
 \nocite{lowrie04}
 Lowrie and Kent (2004). 

In the interval from 83 to the present, the anomalies are associated with normal polarity, so the chrons are designated C1n, C1r, C2n, C2r for the normal associated with Anomaly 1 (the Brunhes), the dominantly reverse interval between Anomaly 1 and Anomaly 2 (the Olduvai) and the dominantly reverse interval between Anomalies 2 and 2A (the Gauss).    There are many more subchrons than this.  For example, the Jaramillo and  a little subchron known as the Cobb Mountain are within subchron C1r.  And it gets worse.   Subchrons can be further subdivided and the Jaramillo is now known as C1r.1n.     Cryptochrons are designated with a "-" after the subchron within which it lies.  One such, the  Cobb Mountain  excursion(1.201-1.211 Ma) is labelled C1r.2r-1n in the most recent time scale.       Because of the complexity of the Anomaly naming scheme, and the continous discovery of new features,  the  GPTS has become  a nightmare of chron and sub-chron names like C5r.2r-2n or C5ADr where the ``n''s and ``r''s refer to polarity and the .1s, .2s and so on refer to the sub-chrons within chrons (e.g, C4n). The C5r.2r-2n is, for example,  the second cryptochron within the second reverse polarity interval of the chron.       \nocite{gee07,channell95,cande95,lowrie04}

The M-sequence names   follow on from the younger chrons.  The youngest of these is therefore CM0r (see Table~\ref{tab:gpts}).  

\begin{figure}[htb]
\epsfxsize 15cm
\centering \epsffile{EPSfiles/opdyke74.eps}
\caption{Declinations from deep-sea piston core RC12-65 from the equatorial Pacific Ocean (using an arbitrary zero line because the cores were not oriented).  The epoch system of magnetostratigrahic nomenclature was extended back to Epoch 11 in this core and to Epoch 19 in companion cores.   [Figure from Opdyke et al., 1974].}
\label{fig:opdyke74}
\end{figure}

\subsection{The addition of biostratigraphy}
  
 An interesting aspect to the magnetostratigraphic work typified by Opdyke et al. (1966) was the identification of biostratigraphic zones ($\Omega$ to $\phi$ in Figure~\ref{fig:opdyke66})  based on faunal assemblages in the core. These zones are therefore tied directly to the magnetostratigraphic record.     The addition of biostratigraphy to the GPTS problem brought new possibilities for the calibration of the time scale in that certain boundaries could  be dated by radioisotopic means using datable layers (e.g., ash beds) within stratigraphic sections.  If a particular well dated biostratigraphic horizon could be tied to the magnetostratigraphic record, then the associated  numerical ages could be attached to the GPTS.     Exploiting this possibility, 
 \index{LaBrecque, J.L.}
 LaBrecque et al. (1977) \nocite{labrecque77} used the magnetostratigraphic record  in Italian carbonates  (e.g., 
 \index{Alvarez, W.}
 Alvarez et al., 1977) \nocite{alvarez77}  which tied the Cretaceous/Tertiary (K/T) boundary to a reverse polarity zone between two normal polarity intervals correlated with marine magnetic anomalies 29 and 30.  The accepted age for the K/T boundary at the time was 65 Ma 
 \index{van Hinte, J.E.}
 (van Hinte, 1976) \nocite{vanhinte76} which is virtually identical to the currently accepted age of 65.5$\pm$ 0.3 Ma 
 \index{Gradstein, F.}
 (Gradstein et al., 2004), so ages for the anomalies numbered 1-34 could be estimated by interpolation and extrapolation.  Note that anomaly 14 is now thought to be a cryptochron (S. Cande, pers. comm.) and has not been included as a numbered anomlay in timescales since 
 \index{LaBrecque, J.L.}
 LaBrecque et al. (1977).   






\nocite{shackleton90} \nocite{hilgen91} 

\begin{figure}[h!tb]
\epsfxsize 10cm
 \centering \epsffile{EPSfiles/hilgen91.eps}
 \caption{Illustration of the  ``astrochronology'' dating method.  The sequence of polarity intervals  and climatically induced sapropel layers is correlated to the GPTS (left) and the orbital cycles (right).  The numerical ages from the orbital cycles  can then be tranferred to the GPTS.  [Adapted from Hilgen et al., 1991.]}
 \label{fig:hilgen91}
 \end{figure} 
\clearpage
\begin{figure}[h!tb]
\epsfxsize 13cm
\centering \epsffile{EPSfiles/newark.eps}
\caption{ Left: Lithostratigraphic and  magnetostratigraphic pattern derived from overlapping drill cores into the Newark Basin.  Right:  Interpretation for the GPTS based on astrochronology and correlation to the Geological Time Scale.  [Adapted from Kent et al., 1995 and Kent and Olson, 1999.]}
\label{fig:newark}
\end{figure}

\clearpage

\subsection{Astrochronology}

Until 1990, the GPTS was dated using numerical ages based on the decay of radioactive elements (largely the K/Ar method).  An alternative approach to dating of stratigraphic sequences long in  use  is based on the climatically induced changes in lithology or stable isotopic records in sediments that are caused by variations in the Earth's orbit around the sun.    The relationship of the Earth's orbit to the sun results in changes in the amount and latitudinal distribution of solar radiation ( ``insolation'') reaching the Earth as a function of time.  According to the
\index{Milankovitch hypothesis}
 {\it Milankovitch hypothesis} (e.g., 
 \index{Hays, J.D.}
 Hays et al. 1976), \nocite{hays76} changes in insolation at high northern latitudes vary with periodicities reflecting precession (with a beat of around 21 kyr), obliquity ($\sim$ 40 kyr) and eccentricity ($\sim$ 100 kyr).  These changes in insolation  resulted in measurable changes in the chemistry of the oceans and atmospheres and left an indelible mark on the lithostratigraphy (e.g., variations in carbonate) and the isotopic ratios of oxygen (the light isotope $^{16}$O gets preferentially incorporated into glacial ice at high latitudes, leaving the oceans richer in $^{18}$O.)  Because the precession, obliquity and eccentricity of Earth's orbit can be robustly predicted as a function of age at least for several million years (and perhaps even 10s of millions of years), identification of these patterns in the stratigraphic record allow numerical ages to be attached to the sedimentary sequence.  This is  a method known as 
 \index{astrochronology}
 {\it astrochronology}.  
Starting with Shackleton (1990) and Hilgen et al. (1991), astrochronology has been applied to the GPTS (see e.g., Figure~\ref{fig:hilgen91}).    



\begin{figure}[h!tb]
\epsfxsize 13cm
\centering \epsffile{EPSfiles/neogene.eps}
\caption{The Neogene of the Geological Time Scale. [Figure from Gradstein et al., 2004.]}
\label{fig:neogene}
\end{figure}



\section {Current status of the geological time scale}

For reference, we include the dates of the most recent GPTS in Table~\ref{tab:gpts}.  As an example of the detailed correlations between the polarity time scale and various biological time scales, we show the Neogene portion from Gradstein et al. (2004) in Figure~\ref{fig:neogene}.    For details, the reader is referred to the original reference.  Please note that the time scale is a consensus document that balances a tremendous amount of information from a variety of sources.  As such, it is subject to change, although  change should not be frequent or drastic.   


\begin{figure}[htb]
\epsfxsize 10cm
\centering \epsffile{EPSfiles/spreadingrate.eps}
\caption{Plot of distance from the ridge crest in the South Atlantic versus age using the GPTS of Gradstein et al., (2004).  The differential of this curve gives the inferred instantaneous spreading rate.}
\label{fig:spreadingrate}
\end{figure}

\section{Applications}

\subsection{Dating geological sequences}

An important application of having a time scale of geomagnetic polarity reversals is  as a dating tool for
stratigraphic sequences. The pattern of polarity zones is determined by
measuring
the magnetization of samples taken from the stratigraphic section.  If the polarity zones in the
so-called 
\index{magnetostratigraphy}
{\it magnetostratigraphy} can be
unambiguously correlated to the GPTS, 
they constitute a precise temporal framework
for sedimentary or volcanic sequences. Such records have proved invaluable
for correlating stratigraphic information on a global basis and are the
primary means for calibrating the Cenozoic fossil 
record with respect to time.   Furthermore, knowing the ages of polarity reversals allows the calculation of rates of processes such as sea-floor spreading, sediment accumulation,  extinctions and speciation and provide independent verification of orbital calculations.  




Sedimentation is not always a continuous process in many environments
and a stratigraphic section may  have gaps of significant duration.
Also, the magnetic recording process of the rock may be unreliable
over all or part of the section.  Furthermore, incomplete sampling may give a
polarity log that is undersampled.  For these reasons, there must be
ways of establishing the reliability of a given polarity sequence and
the robustness of a given correlation. For a more complete discussion of 
the subject of magnetostratigraphy, the reader is referred to the comprehensive
book by 
\index{Opdyke, N.D.}%
\index{Channell, J.E.T.}%
Opdyke and Channell (1996)  entitled {\it Magnetic Stratigraphy}. Briefly,
the elements of a good magnetostratigraphic study include the
following points.
\nocite{opdyke96}

\begin{itemize}
\item It must be established that a single component of magnetization can
be (and has been) isolated by stepwise demagetization.  To demonstrate this,
 examples of demagnetization
data should be shown (see Chapter 9).  There must also be
  a clear discussion of how directions were determined for each sample.

\item Geological materials are not always perfect recorders of the geomagnetic field.
It often happens that a given stratigraphic horizon has no consistent magnetization.
Multiple samples per horizon  (say three to five separately oriented samples)
with coherent directions (i.e., non-random by tests such as those discussed in 
Chapter 11)
indicate that the magnetization at a given level is reproducible. While it is not
always possible to take multiple samples (for example from limited drill core material),
it is alway desirable and certainly should be done whenever possible.

\item  The directional data must fall into two clearly separated groups
that are identifiable as either normal or 
reverse polarity.  If fully oriented samples have been
taken, the data can be plotted on an equal area projection
(see Appendix~\ref{app:eqarea}) and/or subjected to the
reversals test (Chapters 11 and 12).  Often drill cores are not azimuthally oriented,
and the paleomagnetic inclination is the only indicator of polarity.
 In this case, one can plot histograms of the inclination and establish that the two polarities (positive and negative) have discrete
``humps'' at the values expected for the site (paleo)latitude.

\item The average direction should be compared with the reference field
(the GAD field; see Chapter 2), and the expected direction based on the age 
of the formation for the
sampling location.  This can be done on an equal area projection, or in cartesian 
coordinates (e.g., using  Fisher statistics or a bootstrap as described in Chapters 11 and 12).   

\item  Field tests (such as the fold test or conglomerate test as
described in Chapter 9) that compare the age of magnetization relative to the
rock formation are desirable.

\item An independent estimate of the approximate age of the sequence is
necessary.  The better the age constraints, 
the more confident we can be in a given interpretation.

\item The magnetostratigraphic pattern should match the polarity time scale.
 Few polarity zones should be ignored either in the
section or in the time scale.  Ideally, each polarity zone should be
based on multiple sites in the section. 
\end{itemize}


\begin{figure}[htb]
\epsfxsize 12cm
\centering \epsffile{EPSfiles/isochron.eps}
\caption{Application of magnetostratigraphic techniques for delineating
isochronous horizons in a series of stratigraphic sections.  The
polarities of sampling sites are shown by open (reverse) and solid
(normal) symbols.  The light shading indicates   silts, while the
darker shaded units (labelled A-C)  represent sand bodies, 
which were not suitable for paleomagnetic analysis in this example. 
The inferred isochrons (horizons that separate 
polarity zones) are shown as heavy dashed lines. [Figure modified from Behrensmeyer and Tauxe, 1982.]}
\label{fig:isochron}
\end{figure}


\subsection{Measuring rates}

One very useful application of the GPTS is to infer rates of for example spreading, sediment accumulation, etc.   We illustrate this approach in Figure~\ref{fig:spreadingrate}.  Distance from the ridge crest of each identified anomaly is plotted against age.  The previous standard GPTS based on the work of  Cande and Kent (1992) \nocite{cande92} built smooth changes in spreading rate into the GPTS itself.   Gradstein et al. (2004)  included a version of the geomagnetic polarity time scale which  did not have this constraint for the Neogene, because much of it was calibrated using astrochronological methods.  As a result there are sharp changes in spreading rate implied, which could well be artifacts of the method of calibration.   Because of this behavior, we have reverted to the time scale of Cande and Kent (1995; see Table~\ref{tab:gpts}).    Future time scales will likely be calibrated using some balance between astrochronology, smooth variations in  spreading  rate and radioisotopic methods.       

\subsection{Tracing of magnetic isochrons}

Most magnetostratigraphic applications involve determination of a
magnetostratigraphy through a stratigraphic sequence of
sediments.  Because polarity transitions occur relatively rapidly, the horizon bounding two polarity
zones may represent an almost isochronous level.   
It is therefore possible to use magnetostratigraphy in a lateral
sense, in order to delineate isochronous horizons within a given package
of sediments 
\index{Behrensmeyer, A.K.}
(Behrensmeyer and 
\index{Tauxe, L.}
Tauxe. 1982).  \nocite{behrensmeyer82}
In Figure~\ref{fig:isochron}, we show the application of
magnetostratigraphy for tracing isochrons in a series of stratigraphic 
sections.   The small sand body (darker gray) labeled ``A'' appears
 to have removed  the normal polarity zone seen in sequences on the right of the
figure either by erosion or because of unsuitable paleomagnetic properties 
of sand.  
Sand bodies B and C appear to represent quasi-isochronous horizons. 

\vskip .5 in\noindent{\bf 
Suggested Supplemental Reading}
{\obeylines
\parskip 0pt
\hskip 1em Opdyke and Channell (1996)  \nocite{opdyke96}
\hskip 1em Oreskes (2001) \nocite{oreskes01}  
 \hskip 1em Glen (1982) \nocite{glen02}
 \hskip 1em http://www.stratigraphy.org/ 
\hskip 1em Gradstein et al., 2004 \nocite{gradstein04}
\hskip 1em Gee and Kent (2007) \nocite{gee07}
}

\nocite{gee07,cande95,channell95}
\clearpage
 \begin{center}
\begin{longtable}{ll|ll}
\caption[Geomagnetic polarity time scale]{Geomagnetic polarity time scale of Gee and Kent (2007). Period from present to top of C34n is after Cande and Kent (1995) and from M0 to the end is after Channell et al. (1995).   }
\label{tab:gpts}\\

\hline
 Age range (Ma) &  Normal polarity  & Age range (Ma)&Reverse polarity\\
 &subchrons&&subchrons\\
\hline
\endfirsthead

\multicolumn{4}{l}%
{\tablename\ \thetable{}\it  -- continued from previous page} \\
\hline
 Age range (Ma) &  Normal polarity  & Age range (Ma)&Reverse polarity\\
 &subchrons&&subchrons\\
\hline
\endhead

\hline \multicolumn{4}{r}{{\it Continued on next page --}} \\
\endfoot
\hline
\endlastfoot

0.000-0.780 &  C1n & 0.780-0.990 &  C1r.1r\\
0.990-1.070 & C1r.1n & 1.070-1.201 & C1r.2r\\
1.201-1.211 & C1r.2r-1 n & 1.211-1.770 & C1r.2r\\
1.770-1.950 & C2n & 1.950-2.140 & C2r.1r\\
2.140-2.150 & C2r.1n & 2.150-2.600 & C2r.2r\\
2.581-3.040 & C2An.1 n & 3.040-3.110 & C2An.1 \\
3.110-3.220 & C2An.2n & 3.220-3.330 & C2An.2r\\
3.330-3.580 & C2An.3n & 3.580-4.180 & C2Ar\\
4.180-4.290 & C3n. in & 4.290-4.480 & C3n. 1\\
4.480-4.620 & C3n.2n & 4.620-4.800 & C3n.2r\\
4.800-4.890 & C3n.3n & 4.890-4.980 & C3n.3r\\
4.980-5.230 & C3n.4n & 5.230-5.894 & C3r\\
5.894-6.137 & C3An.1 n & 6.137-6.269 & C3An.1 \\
6.269-6.567 & C3An.2n & 6.567-6.935 & C3Ar\\
6.935-7.091 & C3Bn & 7.091-7.135 & C3Br.1r\\
7.135-7.170 & C3Br.1n & 7.170-7.341 & C3Br.2r\\
7.341-7.375 & C3Br.2n & 7.375-7.432 & C3Br.3r\\
7.432-7.562 & C4n.i n & 7.562-7.650 & C4n.1\\
7.650-8.072 & C4n.2n & 8.072-8.225 & C4r. 1\\
8.225-8.257 & C4r.1n & 8.257-8.606 & C4r.2r*\\
8.606-8.664 & C4r.2r-1 n & 8.664-8.699 & C4r.2r*\\
8.699-9.025 & C4An & 9.025-9.097 & C4Ar.1 r*\\
9.097-9.117 & C4Ar.1r-1 n & 9.117-9.230 & C4Ar.1 r*\\
9.230-9.308 & C4Ar.1n & 9.308-9.580 & C4Ar.2r\\
9.580-9.642 & C4Ar.2n & 9.642-9.740 & C4Ar.3r\\
9.740-9.880 & C5n.1n & 9.880-9.920 & C5n.1\\
9.920-10.949 & C5n.2n & 10.949-11.052 & C5r.1\\
11.052-11.099 & C5r.1n &  11.099-11.167 & C5r.2r*\\
11.167-11.193 & C5r.2r-1n & 11.193-11.352 & C5r.2r*\\
11.352-11 .363 & C5r.2r-2n & 11.363-11.476 & C5r.2r*\\
11.476-11.531 & C5r.2n & 11.531-11.555 & C5r.3r*\\
11.555-11.584 & C5r.3r-1n & 11.584-11.935 & C5r.3r*\\
11.935-12.078 & C5An.1n & 12.078-12.184 & C5An.1 r\\
12.184-12.401 & C5An.2n & 12.401-12.678 & C5Ar.1 r\\
12.678-12.708 & C5Ar.1n & 12.708-12.775 & C5Ar.2r\\
12.775-12.819 & C5Ar.2n & 12.819-12.991 & C5Ar.3r\\
12.991-13.139 & C5AAn & 13.139-13.302 & C5AAr\\
13.302-13.510 & C5ABn & 13.510-13.703 & C5ABr\\
13.703-14.076 & C5ACn & 14.076-14.178 & C5ACr\\
14.178-14.612 & C5ADn & 14.612-14.800 & C5ADr\\
14.800-14.888 & C5Bn.1n & 14.888-15.034 & C5Bn.1r\\
15.034-15.155 & C5Bn.2n & 15.155-16.014 & C5Br\\
16.014-16.293 & C5Cn.1n & 16.293-16.327 & C5Cn.1\\
16.327-16.488 & C5Cn.2n & 16.488-16.556 & C5Cn.2r\\
16.556-16.726 & C5Cn.3n & 16.726-17.277 & C5Cr\\
17.277-17.615 & C5Dn & 17.615-17.793 & C5Dr*\\
17.793-17.854 & C5Dr-1 n & 17.854-18.281 & C5Dr*\\
18.281-18.781 & C5En & 18.781-19.048 & C5Er\\
19.048-20.131 & C6n & 20.131-20.518 & C6r\\
20.518-20.725 & C6An.1n & 20.725-20.996 & C6An.1\\
20.996-21.320 & C6An.2n &  21.320-21.768 & C6Ar\\
21.768-21.859 & C6AAn & 21.859-22.151 & C6AAr.1 r\\
22.151-22.248 & C6AAr.1 n & 22.248-22.459 & C6AAr.2r\\
22.459-22.493 & C6AAr.2n & 22.493-22.588 & C6AAr.3r\\
22.588-22.750 & C6Bn.1n & 22.750-22.804 & C6Bn.1r\\
22.804-23.069 & C6Bn.2n & 23.069-23.353 & C6Br\\
23.353-23.535 & C6Cn.1 n & 23.535-23.677 & C6Cn.1\\
23.677-23.800 & C6Cn.2n & 23.800-23.999 & C6Cn.2r\\
23.999-24.118 & C6Cn.3n & 24.118-24.730 & C6Cr\\
24.730-24.781 & C7n.1 n & 24.781-24.835 & C7n.1\\
24.835-25.183 & C7n.2n & 25.183-25.496 & C7r\\
25.496-25.648 & C7An & 25.648-25.678 & C7Ar*\\
25.678-25.705 & C7Ar-1 n & 25.705-25.823 & C7Ar*\\
25.823-25.951 & C8n.1 n & 25.951-25.992 & C8n.1\\
25.992-26.554 & C8n.2n & 26.554-27.027 & C8r\\
27.027-27.972 & C9n & 27.972-28.283 & C9r\\
28.283-28.512 & C10n.1n & 28.512-28.578 & 0n.1r\\
28.578-28.745 &  C10n.2n & 28.745-29.401 &  0r\\
29.401-29.662 &  C11n.1n & 29.662-29.765 & i n.1\\
29.765-30.098 & C11n.2n & 30.098-30.479 & C11r\\
30.479-30.939 & C12n & 30.939-33.058 & C1 2r\\
33.058-33.545 & C13n & 33.545-34.655 & C1 3r\\
34.655-34.940 & C15n & 34.940-35.343 & C1 5r\\
35.343-35.526 & C16n.1 n & 35.526-35.685 & C1 6n.1\\
35.685-36.341 & C16n.2n & 36.341-36.618 & C1 6r\\
36.618-37.473 & C17n.1n & 37.473-37.604 & C1 7n.1\\
37.604-37.848 & C17n.2n & 37.848-37.920 & C1 7n.2r\\
37.920-38.113 & C17n.3n & 38.113-38.426 & C1 7r\\
38.426-39.552 & C18n.1 n & 39.552-39.631 & C18n.1r\\
39.631-40.130 & C18n.2n & 40.130-41.257 & C1 8r\\
41.257-41.521 & C19n & 41.521-42.536 & C1 9r\\
42.536-43.789 & C20n & 43.789-46.264 & C20r\\
46.264-47.906 & C21 n & 47.906-49.037 & C21\\
49.037-49.714 & C22n & 49.714-50.778 & C22r\\
50.778-50.946 & C23n.1 n & 50.946-51.047 & C23n.1r\\
51.047-51.743 & C23n.2n & 51.743-52.364 & C23r\\
52.364-52.663 & C24n.1 n & 52.663-52.757 & C24n.1r\\
52.757-52.801 & C24n.2n & 52.801-52.903 & C24n.2r\\
52.903-53.347 & C24n.3n & 53.347-55.904 & C24r\\
55.904-56.391 & C25n & 56.391-57.554 & C25r\\
57.554-57.911 & C26n & 57.911-60.920 & C26r\\
60.920-61.276 & C27n & 61.276-62.499 & C27r\\
62.499-63.634 & C28n & 63.634-63.976 & C28r\\
63.976-64.745 & C29n & 64.745-65.578 & C29r\\
65.578-67.610 & C30n & 67.610-67.735 & C30r\\
67.735-68.737 & C31n & 68.737-71.071 & C31 r\\
71.071-71.338 & C32n.1n & 71.338-71.587 & C32n.1r\\
71.587-73.004 & C32n.2n & 73.004-73.291 & C32r.1\\
73.291-73.374 & C32r.1 n & 73.374-73.619 & C32r.2r\\
73.619-79.075 & C33n & 79.075-83.000 & C33r\\
83.00-120.60 & C34n (CNPS) & 120.60-121.00 & CM0r\\
121.00-123.19 & CM1n & 123.19-123.55 & CM1r\\
123.55-124.05 & CM2n & 124.05-125.67 & CM3r\\
125.67-126.57 & CM4n & 126.57-126.91 & CM5r\\
126.91-127.11 & CM6n & 127.11-127.23 & CM6r\\
127.23-127.49 & CM7n & 127.49-127.79 & CM7r\\
127.79-128.07 & CM8n & 128.07-128.34 & CM8r\\
128.34-128.62 & CM9n & 128.62-128.93 & CM9r\\
128.93-129.25 & CM10n & 129.25-129.63 & CM10r\\
129.63-129.91 & CM10Nn.1n & 129.91-129.95 & CM10Nn.1r\\
129.95-130.22 & CM10Nn.2n & 130.22-130.24 & CM10Nn.2r\\
130.24-130.49 & CM10Nn.3n & 130.49-130.84 & CM10Nr\\
130.84-131.50 & CM11n & 131.50-131.71 & CM11r.1r\\
131.71-131.73 & CM11r.1n & 131.73-131.91 & CM 11r.2r\\
131.91-132.35 & CM11An.1n & 132.35-132.40 & CM11An.1r\\
132.40-132.47 & CM1 1An.2n & 132.47-132.55 & CM1 1Ar\\
132.55-132.76 & CM12n & 132.76-133.51 & CM12r.1r\\
133.51-133.58 & CM12r.1n & 133.58-133.73 & CM1 2r.2r\\
133.73-133.99 & CM12An & 133.99-134.08 & CM12Ar\\
134.08-134.27 & CM13n & 134.27-134.53 & CM13r\\
134.53-134.81 & CM14n & 134.81-135.57 & CM14r\\
135.57-135.96 & CM15n & 135.96-136.49 & CM15r\\
136.49-137.85 & CM16n & 137.85-138.50 & CM16r\\
138.50-138.89 & CM17n & 138.89-140.51 & CM17r\\
140.51-141.22 & CM18n & 141.22-141.63 & CM18r\\
141.63-141.78 & CM19n.1n & 141.78-141.88 & CM19n.1r\\
141.88-143.07 & CM19n & 143.07-143.36 & CM19r\\
143.36-143.77 & CM20n.1n & 143.77-143.84 & CM20n.1r\\
143.84-144.70 & CM20n.2n & 144.70-145.52 & CM20r\\
145.52-146.56 & CM21n & 146.56-147.06 & CM21r\\
147.06-148.57 & CM22n.1n & 148.57-148.62 & CM22n.1r\\
148.62-148.67 & CM22n.2n & 148.67-148.72 & CM22n.2r\\
148.72-148.79 & CM22n.3n & 148.79-149.49 & CM22r\\
149.49-149.72 & CM22An & 149.72-150.04 & CM22Ar\\
150.04-150.69 & CM23n.1n & 150.69-150.91 & CM23n.1r\\
150.91-150.93 & CM23n.2n & 150.93-151.40 & CM23r\\
151.40-151.72 & CM24n.1n & 151.72-151.98 & CM24n.1\\
151.98-152.00 & CM24n.2n & 152.00-152.15 & CM24r\\
152.15-152.24 & CM24An & 152.24-152.43 & CM24Ar\\
152.43-153.13 & CM24Bn & 153.13-153.43 & CM24Br\\
153.43-154.00 & CM25n & 154.00-154.31 & CM25r\\
154.31-155.32 & CM26n & 155.32-155.55 & CM26r\\
155.55-155.80 & CM27n & 155.80-156.05 & CM27r\\
156.05-156.19 & CM28n & 156.19-156.51 & CM28r\\
156.51-157.27 & CM29n & 157.27-157.53 & CM29r\\
\end{longtable}
\end{center}



\clearpage

\section{Problems}

{\parindent 0pt  \parskip 12pt

{\bf Problem 1: }

a) Go to the MagIC web site at http;//earthref.org/MAGIC.  Click on the PMAG PORTAL
link and search the reference database for the data for Tauxe and Hartl (1997) \nocite{tauxe97} who published paleomagnetic data from a deep sea sediment core.   The core was taken at DSDP Site 522 located at  26$^{\circ}$S/5$^{\circ}$W.  The nannofossils in the core suggest an Oligocene age.   Download the SmartBook and unzip the downloaded archive.  Change directory into the tauxe-and-hartl1997 folder.   Use the program {\bf download\_magic.py} to unpack the file with the .txt extension.  

b) Use the program {\bf strip\_magic.py} to plot the data.  Run it once to see what is available for plotting.  Then plot inclination verus stratigraphic position.  
 What is the GAD inclination at the site?  (You can plot this with the -Pexp option in {\bf timeseries\_magic.py}!     Are the data consistent with that expected value?   What might be cause any discrepancies you observe?    

c) Plot VGP latitude versus age.  
 Find the age range for the Oligocene from the website: stratigraphy.org.  Run the program {\bf strip\_magic.py} again, this time using the -ts switch to select a timescale and an age range for the Oligocene. [Remember to use the -h option to find the correct syntax.]     Identify the Chron boundaries in the magnetostratigraphic section.   

d) Plot the data again, this time using the -x age option to plot the data against the age assigned by the original authors.   They used one of the two time scales (ck95 or gts04).  Which did they use?  

{\bf Problem 2:}

a) Write a program to plot Age versus Depth for the data in the {\it pmag\_results.txt} file used in Problem 1.  

b) What is the average sedimentation rate?  How does it change? 

c) The drill site (DSDP 522) was at the ridge crest  at the bottom of the data set and moved away as the data get younger.  How would you explain the changing sedimentation rates through this data set? 


}

