\chapter{ Getting a paleomagnetic direction}





Rocks become magnetized in a variety
of ways (see Chapter 7).  Both igneous and sedimentary rocks can be affected by chemical
change, thereby acquiring a secondary magnetization.  Many magnetic materials are
affected by viscous remanent magnetization.  The various
components of magnetization sum together to constitute the 
NRM which is the ``raw'' remanence of the sample after extraction. 
The goal of paleomagnetic laboratory work is to isolate the various components
of remanence and to ascribe origin, age and reliability to these
components. But before the
laboratory work can begin, samples must be collected. Sampling
strategy is crucial to a successful study.  We will briefly describe techniques
for sampling, methods of orientation and
overall philosophy.  We will then turn to an overview of some of the more
 useful field and laboratory techniques that wind up with an estimate of a paleomagnetic direction.


\section {Paleomagnetic sampling}

\index{paleomagnetic!sampling}
There are several goals in paleomagnetic samplin:  to average out
the errors involved in the sampling process itself, and  to
assess the reliability of the recording medium.  In addition, we
often wish  to  sample the range of  secular variation of the geomagnetic
 field in order to average it out or characterize its statistical properties
The objectives of averaging recording and sampling
``noise'' are achieved by taking a number $N$ of individually oriented 
\index{paleomagnetic!sample}
{\it samples} from a
single unit 
\index{paleomagnetic!site}
(called a {\it site}).   Samples should be taken such that they represent a single time horizon, that is, they are from a single cooling unit or   the same sedimentary horizon. 
The most careful sample orientation procedure has an uncertainty of several degrees.
Precision is gained proportional to $\sqrt{N}$, so to improve the precision, multiple individually oriented samples are required.  The number of samples taken should be tailored to the
particular project at hand.  If one wishes to know polarity, perhaps three samples would be sufficient
(these would be taken primarily to assess ``recording noise''). If, on
the other hand, one wished to make inferences about secular variation of
the geomagnetic field, more samples would be necessary to suppress sampling
noise.

\eject
\begin{figure}[h!tb]
\epsfxsize 11cm
 \centering \epsffile {EPSfiles/drill.eps}
\caption {Sampling technique with a water-cooled drill: a) drill the
sample, b) insert a non-magnetic slotted tube with an adjustable
 platform around the sample.  Rotate the slot to the top of the sample
and note the azimuth and plunge of the drill direction (into the outcrop)
 with a sun and/or  magnetic compass 
and inclinometer.
 Mark the sample through the slot with a brass or copper wire.
c) Extract the sample.  d) Make a permanent arrow on the side of the
sample in the direction of
drill and label the sample with the sample name. Make a note of the name
and orientation of the arrow in a field notebook. 
}
\label{fig:drill} 
\end{figure}

Some applications in paleomagnetism require that  
the secular variation of the geomagnetic field (the paleomagnetic ``noise'')
be averaged in order to determine the
 time-averaged field direction.  The geomagnetic field varies with
time constants ranging from milliseconds to millions of years.  It is a
reasonable first order approximation to assume that, when averaged 
over, say,
100,000 years, the geomagnetic field is similar to that of  a geocentric axial dipole 
(equivalent to the field that would be produced by 
a bar magnet at the center of the Earth, aligned with the spin axis;
see Chapter 2).
Thus, when a time-averaged field direction 
is required, enough sites can be sampled
to span sufficient time to achieve this goal.  A good rule of thumb is about ten
sites (each with nine to ten samples), spanning 100,000 years.   If the distribution of  geomagnetic field vectors is desired, then more like 100 sites are necessary.  


\subsection{Types of samples}

 Samples can be taken  using a gasoline or electric
powered drill,  as  ``hand samples'' (also known as ``block samples'' or as ``sub-samples''  from a piston core.  

\begin{enumerate}  
\item {\it Samples cored with portable drill. }  The most common type of paleomagnetic sample is collected by
using a gasoline-powered portable drilling apparatus with a water-cooled diamond bit Figure~\ref{fig:drill}a.
The diameter of cores is usually $\sim$2.5 cm. After drilling into the outcrop to a depth of 6 to 12 cm, 
an orientation device is slipped over the sample while it is still attached to the outcrop at its
base  (Figure~\ref{fig:drill}b and Figure~\ref{fig:orientation}).   Orientation devices have an inclinometer for determining dip (angle from the horizontal down or up) or the hade (angle from the vertical down direction)  of the
core axis.  They also have  a magnetic and/or sun compass  for determining azimuth of core axis. The accuracy
of orientation by such methods is about $\pm 3^{\circ}$. A fiducial mark is scratched on to the core with a brass wire (Figure~\ref{fig:drill}b), or if the core has broken free, a mark is made on the outcrop and transferred to the core (with a degrading of accuracy in orientation). 
After orientation, the core is broken from the
outcrop (Figure~\ref{fig:drill}c), marked for orientation and identification (Figure~\ref{fig:drill}d), and returned to the laboratory. Advantages
of the coring technique are the ability to obtain samples from a wide variety of natural or
artificial exposures and accurate orientation. Disadvantages include the necessity of transporting
heavy fluids (water and gasoline) to the sampling site, dependence on performance of the drilling
apparatus (often in remote locations), and herniated disks,  damaged shoulders  and hearing loss suffered by inveterate drillers.


\begin{figure}[h!tb]
\epsfxsize 11cm
 \centering \epsffile {EPSfiles/hand.eps }
\caption {Hand sampling technique for soft sediment:  a) dig down to fresh material, b)
rasp off a flat surface, c) mark the strike and dip on the sample, d)
extract the sample and label it.}
\label{fig:hand}
\end{figure}


\item {\it Block samples.}   In some locations or with particular lithologies that are not easily drilled, logistics (or
laws) might demand collection of oriented block samples. Some samples can be shaved with a hand rasp to create a flat surface which can be oriented (e.g., Figure~\ref{fig:hand}).  Joint blocks are often oriented (generally
by determining the strike and dip of a surface) and then removed from the outcrop.  For unlithified
sediments, samples may be carved from the outcrop (see also 
\index{Schnepp, E.}
Schnepp et al., 2008). \nocite{schnepp08}   Advantages of block sampling are freedom
from reliance on coring apparatus and the ability to collect lithologies that are unsuitable for coring.
There are, however, conspicuous disadvantages: limited accuracy of orientation, the need to collect
joint blocks (likely more weathered than massive portions of outcrops), and the need to transport
large numbers of cumbersome block samples out of the field and later subsample or sand these to
obtain specimens.


\begin{figure}[h!tb]
\epsfxsize 11cm
\centering \epsffile{EPSfiles/core.eps}
\caption{Sampling of a sediment core.  A plastic cube with a hole in it to let the air escape is pressed into the split surface of  a core.  The orientation arrow points ``up core''.  After extraction, a label with the sample name is put on. Figure from Kurt Schwehr's web site.}
\label{fig:core}
\end{figure}



\begin{figure}
\epsfxsize 12cm
\centering \epsffile{EPSfiles/orientation.eps}
\caption{Orientation system for sample collected by portable core drill. a)  Schematic
representation of core sample in situ. The $Z$ axis points into outcrop; the $X $axis is in the vertical
plane; the $Y$ axis is horizontal.  b) Orientation angles for core samples. The
angles measured are the hade of the $Z$ axis (angle of $Z$ from vertical) and geographic azimuth of the
horizontal projection of the +$X$ axis measured clockwise from geographic north.   }
\label{fig:orientation}
\end{figure}

\item {Lake-bottom or sea-bottom core samples.} Numerous devices have been developed to obtain columns
of sediment from lake or sea bottom. Diameters of these coring devices are typically $\sim$10 cm
and may be of circular or square cross section. Most such cores are azimuthally unoriented and are
assumed to penetrate the sediment vertically. Depth of penetration for ordinary piston cores is usually $<$ 20 m. 
Deep sea coring by the Deep Sea Drilling Program and its successors, the Ocean Drilling Program and the Integrated Ocean Drilling Program allow collection of 100s of meters of overlapping cores with virtually 100\% recovery. Samples for laboratory measurement are subsampled from
the sediment core (Figure~\ref{fig:core}.)

\end{enumerate} 



The diversity of paleomagnetic investigations and applications makes it hard to generalize about sample
collection, but there are some time-honored recommendations. One obvious recommendation is to collect
fresh, unweathered samples. Surface weathering oxidizes magnetite to hematite or iron-oxyhydroxides,
with attendant deterioration of NRM carried by magnetite and possible formation of modern CRM. Artificial
outcrops (such as road cuts) thus are preferred locations, and rapidly incising gorges provide the best
natural exposures.

Lightning strikes can produce significant secondary IRM, which can mask primary NRM. Although partial
demagnetization in the laboratory can often erase lightning-induced IRM, the best policy is to avoid lightning
prone areas. When possible, topographic highs should be avoided, especially in tropical regions. If samples
must be collected in lightning-prone areas, effects of lightning can be minimized by two procedures.

 Outcrops of strongly magnetic rocks such as basalts can be surveyed prior to sample collection to
find areas that have probably been struck by lightning. This is done by ``mapping'' the areas where
significant ($>5^{\circ}$) deflections of the magnetic compass occur. If a magnetic compass is passed over
an outcrop at a distance of $\sim$15 cm from the rock face while the compass is held in fixed azimuth, the
strong and inhomogeneous IRM produced by a lightning strike will cause detectable deflections of
the compass. These regions then can be avoided during sample collection.
 
 \subsection{Orientation in the field}
 
 In general, some direction (drill direction, strike and dip, direction of a horizontal line or even just the ``up'' direction is measured on the sample.  This direction is here called the {\it field arrow}.  When samples are prepared in to specimens for measurement, the field arrow is often replaced by a {\it lab arrow} which is frequently in some other direction.     Procedures for orientating the field arrow are varied, and no standard convention exists. However, all orientation
schemes are designed to provide an unambiguous {\it in situ} geographic orientation of each sample.    A variety of tools are used including orientation devices with magnetic and sun compasses, levels for measuring angles from the horizontal and even differential GPS devices for establishing the azimuth of a local baseline without the need for magnetic or sun compasses.  
 
If a magnetic compass is used to orient samples in the field.  The preferred practice is to set the compass declination to zero.  Then, in post-processing,  the
measured azimuth must be adjusted by the local magnetic declination,
which can be calculated from the known reference 
field (IGRF or DGRF; see Chapter 2). The hade (angle from vertical down) or plunge (angle down [positive] or up [negative] from horizontal) of the sample can also be gotten using an inclinometer (either with a Pomeroy orientation device as shown in Figure~\ref{fig:drill} or with some other inclinometer, such as that on a Brunton Compass.)


Sometimes large local magnetic anomalies, for example from a strongly magnetized rock unit,  can lead to a bias in the magnetic direction that is not compensated for by the IGRF magnetic declination.  In such cases, some other means of sample orientation is required.  One relatively straightforward way is to use a 
\index{sun compass}
{\it sun compass}.  Calculation of a direction using a 
sun compass is 
more involved than for magnetic compass, however.  A dial with a vertical needle (a {\it gnomon}) is placed on the
horizontal platform shown in Figure~\ref{fig:suncomp}.  The angle ($\alpha$) that the
sun's shadow makes with the drilling direction is noted as well as the
exact time of sampling and the location of the sampling site. 
 With this information and
the aid of the Astronomical Almanac or a simple algorithm (see Appendix), it is
possible to calculate the desired direction to reasonable
accuracy (the biggest cause of uncertainty is actually reading the
shadow angle!).  

\begin{figure}[htb]
\epsfxsize 12cm
\centering \epsffile{EPSfiles/orient.eps}
\centering \epsffile{EPSfiles/suncomp.eps}
\caption{a) Pomeroy orientation device in use as a sun compass.  b) Schematic of the principles of sun compass orientation.}
\label{fig:suncomp}
\end{figure}

Another way to avoid the deflection of compass needle by strong local magnetic anomalies is to check the direction by sighting to known landmarks or by moving  a second magnetic  compass well away from the outcrop and  
\index{back-sighting}
{\it back-sighting} along the drill direction.  This is easiest by   using the sun-compass gnomon and sighting tip of the original compass as guides (see Figure~\ref{fig:backbite}).  The original magnetic compass direction (near the outcrop) can be  compared to the backsighted direction in order to detect and remove any deflection.  Of course the compass reading made with the orientation device (near outcrop) is more precise ($\sim 3^{\circ}$), but backsighting can be done with a precision of $\sim 5^{\circ}$ with care.    

\begin{figure}[htb]
\epsfxsize 13cm
\centering \epsffile{EPSfiles/backbite.eps}
\caption{Back-sighting technique using a Pomeroy orientation device and two Brunton Compasses.  One is used with the Pomeroy to measure the direction of drill and the other is used to check for deflection caused by local magnetic anomalies.}
\label{fig:backbite}
\end{figure}
\begin{figure}[h!tb]
\epsfxsize 11cm
\centering \epsffile{EPSfiles/gps.eps}
\caption{ Differential GPS system for orienting paleomagnetic samples in polar regions.  Photo taken during sampling trip to the foothills of the Royal Society Ranges in Antarctica, Jan. 2004.  
}
\label{fig:gps}
\end{figure}


A new technique, developed by 
\index{Constable, C.G.}
\index{Vernon, F.L.}
Cathy Constable and Frank Vernon at Scripps Institution of Oceanography  and described by 
\index{Lawrence, K.L.}
Lawrence et al. (2008; see Figure~\ref{fig:gps}) \nocite{lawrence08}  uses 
\index{differential GPS}
differential Global Positioning System (GPS) technology to determine azimuth of a baseline.  Two GPS receivers are attached to either end of a  one meter non-magnetic rigid base.  The location and azimuth of the baseline can be computed from the signals detected by the two receivers.  The orientation of the baseline is transferred to the paleomagnetic samples using a laser mounted on the base which is focused on a prism attached to the orientation device used to orient the paleomagnetic samples.   The orientations derived by the differential GPS are nearly identical to those obtained by a sun compass, although it takes at least an additional half hour  and is rather awkward to transport.  Nonetheless, achieving sun-compass accuracy in orientations when the sun is unlikely to be readily available is a major break through for high latitude paleomagnetic field procedures.  



\begin{figure}[htb]
\epsfxsize 10cm
\centering \epsffile{EPSfiles/samples.eps}
\caption{Various types of possible specimen shapes and orientation conventions.  a) A one inch slice from a drilled core.    b) A cubic specimen of sediment sanded from a hand sample. c) A specimen (also sample) from a piston core.}
\label{fig:samples}
\end{figure}

\subsection {A note on terminology}

Samples are brought to the laboratory
 and trimmed into  standard sizes and shapes (see Figure~\ref{fig:samples}).  These sub-samples are called
 \index{paleomagnetic!specimen}
{\it paleomagnetic specimens}.  A rule of thumb about terminology is that a sample is something you take and a specimen is something you measure.   The two may be the same object, or there may be multiple specimens per sample.    A site is a single horizon or instant in time and may comprise multiple samples or may be only a single sample, depending on the application.  Multiple specimens from a single site are expected to record the same geomagnetic field.    




\section{Measurement of magnetic remanence}
\label{sect:meas}

     
We measure the magnetic remanence of paleomagnetic samples in a 
\index{magnetometer!rock}
{\it rock magnetometer}, of which there are various types.  The cheapest  are
\index{magnetometer!spinner}
 {\it spinner magnetometers} so named because
they spin the sample to create a fluctuating electromotive force 
(emf).  The emf is
proportional to the magnetization and can be determined relative to the three
axes defined by the sample coordinate system.   The magnetization along a
given axis is measured by detecting the voltages induced by the spinning
magnetic moment within a set of pick-up coils.




Another popular way to measure the magnetization of a sample is to use a
\index{magnetometer!cryogenic}
{\it cryogenic magnetometer} (see Figure~\ref{fig:magnetometer}).
 These magnetometers operate 
using so-called
\index{superconducting quantum interference device}
 {\it superconducting quantum interference
devices}  (SQUIDs). In a SQUID, 
the  flux of an inserted sample is opposed by a  
current in a loop of superconducting wire.  
The superconducting loop is constructed with a {\it weak link}
which stops superconducting at some very low current density, corresponding to some very small
quantum of flux.  Thus the flux within the loop can
change by discrete quanta.  Each incremental change is counted and the total
flux is proportional to  the magnetization along the axis of the SQUID.  
Cryogenic magnetometers are much faster and more sensitive than
spinner magnetometers, but they cost much more to buy and to operate.   

Magnetometers are used to
 measure the three components of the magnetization necessary to
define a vector (e.g., $x_1,x_2,x_3$ or equivalently $x,y,z$).  These data 
can be converted to the more common form of
$D$, $I$ and $M$  by methods described in Chapter 2. 


\begin{figure}[h!tb]
\epsfxsize 14cm
\centering \epsffile{EPSfiles/magnetometers.eps}
\caption{a) Cryogenic magnetometer.  The sample is inserted into the opening of the shields.   There are three SQUIDS that detec the magnetic moment which is read off the three electronic boxes to the left. b) Spinner magnetometer.  The sample is inserted into the opening in a cup.  It spins around, generating an electromagnetic force which is detected with a circular fluxgate.  Two components are measured at a time. }
\label{fig:magnetometer}
\end{figure}



\section {Changing coordinate systems}


Data often must be transformed from the specimen coordinate system into, 
\index{coordinate systems!geographic}%
for example, geographic coordinates.  This can be done graphically
with a stereonet or by means of matrix manipulation.  We outline the general case for transformation of coordinates in the Appendix.  Here we examine the specific cases of the transformation from specimen coordinates to geographic coordinates and the transformation of geographic coordinates to tilt corrected coordinates, the two most commonly used rotations in paleomagnetism.  

No matter how the sample was taken,  data in the laboratory are measured with respect to the specimen coordinate system, so all the field arrows, no matter how obtained, must be converted into the direction of the lab arrow ($x $; see example in  Figure~\ref{fig:orientation} and Figure~\ref{fig:samples}a for field drilled samples.)      Suppose we measured a magnetic moment $m$ (Figure~\ref{fig:digeo}a).  The components of $m$ in specimen coordinates are $x, y, z$ or equivalently, $x_1, x_2, x_3$.   Ordinarily, this coordinate system is at some arbitrary angle to the geographic coordinate system, but we know the Azimuth and Plunge ($Az, Pl$) of the lab arrow with respect to the geographic coordinate system (Figure~\ref{fig:digeo}b).   By substituting $Az$ and $Pl$ for $\phi$ and $\lambda$ into Equation~\ref{eq:aij}, the components of the direction of $m$ in geographic coordinates can be calculated.  These then can be converted back into $D, I$ and $m$ using the equations given in Chapter 2.  Note that $m$ stays the same during the transformation of coordinates.  

To correct for tilt, it is simplest to understand if this is performed as three rotations.  This is how it is done graphically with a stereonet and it is possible to do it the same way with a computer.  [It can also be done as a single rotation, which would be computationally faster, but much harder to visualize.]    First, rotate the direction of magnetic moment in specimen coordinates about a vertical axis by subtracting the dip direction from the declination of the measurement.   Then substitute $\phi=0$ and $\lambda=$ = - dip into Equation~\ref{eq:aij} to bring the dip back up to horizontal.  Finally, rotate the direction  back around the vertical axis by adding the dip direction back  on to the resulting rotated declination.   

\begin{figure}[h!tb]
\centering \epsffile{EPSfiles/digeo.eps}
\caption{a) Specimen coordinates with $X_1$ being along the ``lab arrow''.  A magnetic moment $m$ was measured relative to the specimen coordinate system with components $x_1, x_2, x_3$.  The orientation of the lab arrow with respect to geographic coordinates ($X'_1 = N$) is specified by the azimuth and plunge ($Az, Pl$) of the lab arrow. }
\label{fig:digeo}
\end{figure}


\section {Demagnetization techniques}
\label{sect:demag}


Anyone who has dealt with magnets (including magnetic tape, credit
cards, and  magnets) knows that they are delicate and  likely to 
demagnetize or change their magnetic properties if abused by heat or stress. 
Cassette tapes left on the dashboard of the car in the hot sun never sound the
same.  Credit cards that have been through the dryer may lead to acute
embarrassment at the check-out counter.
 Magnets that have been dropped, do not work as well
afterwards.  It is not difficult to imagine  that rocks that have been
left in the hot sun or buried deep in the crust (not to mention 
altered by diagenesis  or 
 bashed with hammers, drills, pick axes, etc.), may not have their
original magnetic vectors completely intact.  Because rocks often
contain
millions of tiny magnets, it is possible that some (or all) of these have
become realigned, or that they 
grew since the rock formed.  In many cases, there are
still grains that carry the original remanent vector, but there are
often populations of grains that have acquired  new components of
magnetization.

Through geologic time, certain grains may acquire sufficient energy to
overcome the magnetic anisotropy energy  and 
change their direction of magnetization (Chapter 7). In
this way, rocks can acquire a viscous magnetization in the direction of the ambient field. Because the grains
that carry the viscous magnetization 
necessarily have lower magnetic anisotropy energies (they are 
``softer'', magnetically speaking), we  expect 
their contribution to be more easily randomized than
the more stable (``harder'') grains carrying the ancient remanent
magnetization. 

There are several laboratory techniques that are available for
separating various components of magnetization. 
Paleomagnetists rely on the relationship of relaxation time, coercivity, and
temperature in order to remove ({\it demagnetize}) low stability remanence
components.  The
fundamental principle that underlies demagnetization techniques is that the lower
the relaxation time $\tau$, the more likely the grain will carry a secondary
magnetization.  The basis for 
\index{demagnetization!alternating field}
{\it alternating field} (AF) demagnetization 
 is that components with short relaxation times also have low coercivities.
The basis  for
\index{demagnetization!thermal}
 {\it thermal} demagnetization is that these grains also
have low blocking temperatures.

\begin{figure}[htb]
\epsfxsize 14cm
\centering \epsffile{EPSfiles/demagnetizers.eps}
\caption{a) Alternating field demagnetizer.  The sample is placed within the coil inside the tubular shield.  An alternating field is generated with a specified peak intensity.  This decays away, randomizing all magnetic moments that are softer than the peak field that have a component parallel to the applied field direction.  The procedure is repeated along all three axis. A small  DC field can  applied along the direction of the coils to produce an ARM.  b) Thermal demagnetizer.   Samples are placed in boats inside a non-inductively wound oven that is inside the tubular shields.  The ovens are heated to a specified temperature and allowed to cool either in zero field or in a laboratory controlled DC field produced by a coil inside the shield.   This either demagnetizes or remagnetizes all grains with blocking temperatures lower than the specified temperature.}
\label{fig:demagnetizers}
\end{figure}

In AF demagnetization (see Figure~\ref{fig:demagnetizers}a), an oscillating field is applied to a
paleomagnetic sample in a null magnetic
field environment (Figure 7.23 in Chapter 7).  All the grain moments with
coercivities below the peak AF will track the field.  These entrained moments
will become stuck as the peak field gradually decays below the
coercivities of individual grains.   Assuming that there is a range of
coercivities in the sample, the low stability grains will be stuck half along
one direction of the AF and half along the other direction; the net
contribution to the remanence will be zero. In practice, we
demagnetize samples sequentially along three orthogonal axes, or
while ``tumbling'' the sample around three axes during
demagnetization.


Thermal demagnetization (see Figure~\ref{fig:demagnetizers}b) exploits the relationship of relaxation time and temperature.  There will be a temperature below the Curie temperature at which the relaxation time is a few hundred seconds.  When heated to this temperature, grains with relaxation times this short will be in equilibrium with the field.  This is the {\it unblocking temperature}.  If the external field is zero, then there will be no net magnetization.  Lowering the temperature back to room temperature will result in the relaxation times growing exponentially until these moments are once again fixed.  In this way, the contribution of lower stability grains to the NRM can be randomized.  Alternatively, if there is a DC field applied during cooling, the grains whose unblocking temperatures has been exceeded will be realigned in the new field direction;  they will have acquired a partial thermal remanent magnetization (pTRM).  

\begin{figure}[p]
\epsfxsize  11cm
 \centering \epsffile {EPSfiles/comps.eps}
\caption {Principle of progressive demagnetization.
Specimens with two components of magnetization (shown by heavy arrows on
the right hand side), 
with discrete coercivities (plotted as histograms to the left).
The original ``NRM'' is the sum of the two magnetic
 components and is shown as the 
+ in the diagrams to the right.
Successive demagnetization steps (numbered)  remove the component with coercivities
lower than the peak field, 
and the NRM vector changes as a result. a) The two
distributions of coercivity are completely separate.
b) The two distributions partially overlap resulting in simultaneous removal of
both components.  c) The two distributions 
completely overlap.  d) One
distribution envelopes the other. 
}
\label{fig:comps}
\end{figure}

We sketch the principles of progressive 
\index{demagnetization!step-wise}
(step-wise) demagnetization in
Figure~\ref{fig:comps}.  Initially, the NRM is the sum 
of two components carried by populations with different coercivities.
The distributions of coercivities are shown in the histograms to the
left in Figure~\ref{fig:comps}.  Two components of magnetization
 are shown as heavy lines
in the plots to the right.  In these examples, the two components are
orthogonal.  The sum of the two components at the start (the NRM or demagnetization step `0') is shown as a
+ on the vector plots to the right.  
After the first AF demagnetization step, the contribution
of the lowest coercivity grains has been erased and the remanence
vector moves to the position of the first dot away from the +.
  Increasing the AF in successive treatment steps (some are numbered in the diagram)
gradually eats away at  the remanence vectors (shown as dashed
arrows and dots in the plots to the right) which eventually
approach
the origin.  

There are four different sets of coercivity spectra shown in Figure~\ref{fig:comps},
each with a distinctive behavior during demagnetization.
If the two coercivity fractions are completely distinct, the
two components are clearly defined (Figure~\ref{fig:comps}a) by the
progressive demagnetization. 
If there is some overlap in the coercivity distribution of the
components the resulting demagnetization 
diagram is curved (Figure ~\ref{fig:comps}b).
If the two components completely overlap, both components are removed
simultaneously and an apparently single component demagnetization diagram 
may result (Figure~\ref{fig:comps}c). 
It is also possible for one 
coercivity spectrum to include another as shown in Figure~\ref{fig:comps}d.  Such  cases result in ``S'' shaped demagnetization curves.
 Because complete overlap actually happens in ``real''
rocks, it is desirable to perform both AF and thermal demagnetization. If the
two components overlap completely in coercivity, they might not
have overlapping  blocking temperature distributions and vice versa.   
It is unlikely that samples from the same lithology will all have identical 
overlapping distributions, so multiple samples can 
provide clues to the possibility of  completely 
overlapped directions in a given
sample.


\begin{figure}[htb]
\epsfxsize 6cm
\centering \epsffile{EPSfiles/demag.eps}
\caption{Projection of NRM vector into three components (North, East and Down).  The dashed lines are South, West and Up.  The color of the dot is the RGB value assigning North to be red, East to be green and Down to be blue.   The NRM vector is gradually erased through progressive demagnetization to zero revealing several components.    }
\label{fig:demag}
\end{figure}

\begin{figure}[h!p]
\epsfxsize 10cm
\centering \epsffile {EPSfiles/zijd.eps}
\caption{a) Solid  (open) symbols are horizontal (vertical) projections respectively.  
 Peak alternating fields for each demagnetizing step (in mT) are indicated.  
  Inset is equal area plot of the same data.   Solid (open) symbols are projections onto the lower (upper)
hemisphere.  b) Intensity  as a function of demagnetization step.  Data from a).    The median destructive field (mdf of Chapter 8) also shown.  c)  Specimen with  two components with overlapping stabilities.  Inset  as in a).   Best fit great circle is shown as the curve through the data (dashed portion is upper hemisphere projection).  d) Data from c), plotted as  in b).  [Data from Tauxe et al. 2003.]  e) Data from Tauxe et al., 2004b in specimen coordinates.  During  demagnetization, the vector grows toward the last demagnetization direction (-Y).  Deviation ANGle, DANG also shown.  f) Data from e) as in b) Note increase in intensity at high demagnetizating fields.  }
\label{fig:zijd}
\end{figure}
\nocite{tauxe03b} \nocite{tauxe04b} 

\section {Estimating a direction from demagnetization data}

 Now we will consider briefly the issue of what to do with the demagnetization data in terms of display and estimating a best-fit direction for various components.

The standard practice in demagnetization is to measure the NRM and then
to subject the sample
 to a series of demagnetization steps  of increasing
severity using the equipment described earlier in the chapter.
The magnetization of the sample is measured after each step.
During demagnetization,
the remanent magnetization vector will change (see Figure~\ref{fig:demag}) until 
the most stable component has been isolated, 
at which point the
vector decays in  a straight line
 to the origin.  This final component is called the 
{\it characteristic remanent magnetization} or ChRM.  

Visualizing demagnetization data is a
three-dimensional problem and therefore
difficult to plot on paper. Paleomagnetists  often rely on a set
of two projections of the vectors, one on the horizontal plane and one on the
vertical plane.  These are variously called
\index{Zijderveld diagram}
\index{orthogonal projections}
\index{vector end-point diagram}
\index{Zijderveld, J.D.A.}
Zijderveld diagrams (Zijderveld [1967]),\nocite{zijderveld67} orthogonal
projections, or vector end-point diagrams.  


 In orthogonal projections, the
 $x_1$  component is plotted versus $x_2$ (solid
symbols) in one projection,
  and $x_1$ is replotted versus Down ($x_3$) (open symbols) in
another projection.   In the plots shown in Figure~\ref{fig:zijd}a,c we have rotated the vector such that the $x_1$  component is parallel to the original NRM direction.   The
paleomagnetic convention differs from the usual x-y plotting convention
because  $x_2$ and $x_3$ are on the $-y$ axis.   
The paleomagnetic
conventions make sense if one visualizes the diagram as a map view for
the solid symbols
and a vertical projection for the open symbols. 

In Figure~\ref{fig:zijd}, we show several general types of demagnetization behavior.  In 
Figure~\ref{fig:zijd}a,
the sample has a North-Northwest and downward directed  NRM (see inset of equal area projection  in geographic coordinates.)  The direction does not change during demagnetization and the NRM is 
a single vector. 
The sample in Figure~\ref{fig:zijd}c shows a progressive change in direction from
a Westward and up directed component to a North and down
 direction.  The vector continuously changes direction to the end and no
final ``clean'' direction has been confidently isolated.  
These data are plotted on an equal area projection in the inset along with the trace of the  best-fitting plane (a great circle).
 The most stable
component probably lies somewhere near the best-fitting plane.    This specimen came from the outcrop depicted in Figure~\ref{fig:lightning} in Chapter 7 which had been hit by lightning.  The  presumptive IRM is much ``softer'' on demagnetization (Figure~\ref{fig:zijd}d) than the specimen that had not been hit by lighting (Figure~\ref{fig:zijd}a,b) and  is more than an  order of magnitude stronger.      The median destructive field (from Chapter  8) is illustrated in Figure~\ref{fig:zijd}b.   The behavior in Figure~\ref{fig:zijd}e,f is markedly different in that the intensity, after an initial smooth decrease, begins to climb again at high demagnetizing fields.  The direction is deflected from the origin towards the last axis to be demagnetized.  This behavior is typical of ARM acquisition during demagnetization and can occur even if there is zero bias field - a phenomenon known as gyromagnetic remanence (see Chapter 7).   

When specimens acquire a remanence either along the axis of the oscillating field (an ARM) or orthogonal to it (a GRM as in Figure~\ref{fig:zijd}e) they require a more complicated demagnetization regime than just along the three axes.  In the case of the parallel acquisition, a 
\index{demagnetization!double}
double demagnetization protocol works well.  In double demagnetization (e.g., 
\nocite{tauxe04}
\index{Tauxe, L.}
Tauxe et al., 2004),   a specimen is subjected to demagnetization along the three orthogonal axes, say along +X,+Y,+Z,  and is measured, then demagnetized along -X, -Y, -Z and remeasured.  The two measurements are averaged to give an ARM free vector.  In the case of GRM,  
\index{Stephenson, A.}
Stephenson (1993) 
 \nocite{stephenson93} 
 developed a 
\index{demagnetization!GRM protocol}
triple demagnetization protocol whereby specimens are  demagnetized along +y, +z, +x, measured,
then demagnetized along +y, measured and   finally along +z and measured. These three steps are averaged to give a GRM-free vector.    GRMs have been associated with specimens that have a high anisotropy (e.g., 
\index{Stephenson, A.}
Stephenson, 1993; 
\index{Tauxe, L.}
Tauxe et al., 2004),  or have a greigite magnetic remenance (e.g.,
\nocite{snowball97}
\index{Snowball, I.}
 Snowball, 1997).    




Some people choose to plot the pairs of points ($x_1,x_2$) versus $(H,x_3)$
where $H$ is the horizontal projection of the vector 
given by $\sqrt{x_1^2+x_2^2}$.  In this projection, which is 
sometimes called a {\it component plot}, the two axes do not
correspond to the same vector from point to point. 
Instead, the coordinate system changes
with every demagnetization step because $H$ almost always changes direction,
even if only slightly.  
Plotting $H$ versus $x_3$ is therefore a confusing and misleading practice. 
The primary rationale for doing so is because, in the traditional orthogonal projection, the vertical component 
reveals only an apparent inclination.  
If something close to true inclination is desired, then, instead of plotting $H$ and $x_3$,
one can simply rotate the horizontal axes of the orthogonal plot such that it 
closely parallels the desired declination (Figure~\ref{fig:zijd}a,b).


\section {Vector difference sum}
\label{sect:vds}
 
\index{vector difference sum}% 
An equal area projection may be the most useful way to present demagnetization data from a
sample with several strongly overlapping remanence
 components (such as in
Figures~\ref{fig:zijd}c-d).    In order to represent the vector nature of
paleomagnetic data, it is
necessary to plot intensity information.  Intensity can be
plotted versus demagnetization step in an 
\index{intensity decay curve}%
{intensity decay curve} (Figure~\ref{fig:zijd}b,d).  
However, if there are several components with different
directions, the intensity decay curve cannot be used to determine,
say, the blocking temperature spectrum or mdf, because it is the vector
sum of the two components.  It is therefore
 advantageous to consider the decay curve of
\index{vector difference sum}%
the {\it vector difference sum} (VDS) of
\index{Gee, J.S.}
 Gee et al. (1993). \nocite{gee93}
 The VDS
 ``straightens out'' the various components by summing up the vector differences
at each demagnetization step, so the total magnetization is plotted, as
opposed to the resultant.    
 

\begin{figure}[htb]
\epsfxsize 7cm
\centering \epsffile{EPSfiles/foldtesta.eps}
\epsfxsize 7cm
\centering \epsffile{EPSfiles/foldtestb.eps}
\caption{Sampling units with different bedding attitudes in the ``fold
test''. a) Example of folded beds. (Picture from Dupont-Nivet et al.,  2002.)  b) Hypothetical paleomagnetic directions are shown on equal area projections before and after
adjusting for bedding tilt.  Top pair represents the case in which the grouping of paleomagnetic directions is improved after adjusting for tilt which would argue for  a pre-tilt acquisition of remanence.  Lower pair represents a post-tilt acquisition of remanence in which the grouping is worse after restoring beds to the horizontal position. }
\label{fig:fold}
\end{figure} 

  \nocite{dupont-nivet02}


\section {Best-fit lines and planes}
\label{sect:BFL}

\index{principal component analysis}%
 Orthogonal vector projections aid in   identification
of the various remanence
components in a sample.  Demagnetization data are usually
treated using  what is known as 
\index{principal component analysis}
 {\it principal component analysis} 
\index{Kirschvink, J.}
(Kirschvink, 1980).  \nocite{kirschvink80} 
This is done by calculating the orientation tensor for the set of data and finding its eigenvectors ($V_i$) and eigenvalues ($\tau_i$); see Appendix~\ref{app:eigen} for computational details.  
What comes out of the analysis is a best-fit line through a single component of data  as in 
Figure~\ref{fig:zijd}a,b or a best-fit planes or great circle  through multi-component data as in Figure~\ref{fig:zijd}c,d.   
\index{Kirschvink, J.}
Kirschvink [1980]  also defined the 
\index{maximum angle of deviation}
{\it maximum angle of deviation} or (MAD) for each of these. 

The best-fit line is given by the principal eigenvector $V_1$ its MAD  is given by:  

\begin{equation}
MAD = \tan^{-1}(\sqrt{(\tau_2^2+\tau_3^2})/{\tau_1}).
\label{eq:mad}
\end{equation}


If no unique principal direction can be isolated (as for the sample in
Figure~\ref{fig:zijd}c-d), the eigenvector $\V_3$ associated with the least 
eigenvalue $\tau_3$ can be taken as
the pole to the best-fit plane wherein  the component of interest must lie. 
The MAD angle for the best-fit plane is given by:

\begin{equation}
MAD_{\hbox {plane}} = \tan^{-1} \sqrt { \tau_3/\tau_2 + \tau_3/\tau_1}.
\label{eq:madP}
\end{equation}

   The angle between the best-fitting line through the data and the origin is termed the 
   \index{deviation angle}
   {\it Deviation ANGle} or DANG
   \index{Tauxe, L.}
   \index{Staudigel, H.}
    (Tauxe and Staudigel, 2004; Appendix~\ref{app:pint}). 


\begin{figure}[htb]
\epsfxsize 7cm
\centering \epsffile{EPSfiles/congloma.eps}
\epsfxsize 5cm
\centering \epsffile{EPSfiles/conglomb.eps}
\caption {The paleomagnetic conglomerate test. a) The target lithology was involved in a catastrophic event leading to incorporation into a conglomerate bed.  Samples are taken from individual clasts.  The directions of samples from
the target lithology are shown in b) indicating that it is relatively
homogeneously magnetized.  
c) dirctions from the  conglomerate clasts are also homogeneously
magnetized;
the magnetization must post-date formation of the conglomerate.  
In a positive
conglomerate test d),   the magnetization vectors
of samples from the conglomerate clasts are random.
 }
\label{fig:conglom}
\end{figure}



\begin{figure}[htb]
\epsfxsize 10cm
\centering \epsffile{EPSfiles/baked.eps}
\caption{The baked contact test.  In a positive test, 
zones baked by the intrusion are remagnetized and have directions that
grade from that of the intrusion to that of the host rock.
If all the material is homogeneously magnetized,
then the age of the intrusion places an upper bound on the age of
magnetization.}
\label{fig:baked}
\end{figure}


\section{Field strategies}

In addition to establishing that a given rock unit
 retains a consistent magnetization, it is also of interest
to establish when this magnetization was acquired.  Arguments concerning the age of magnetic remanence can be
built on indirect petrographic evidence as to the relative ages of various magnetic minerals, or by evidence
based on geometric relationships in the field.
  There are two key field tests that require special sampling
strategies: the fold test and the conglomerate test.

The
\index{fold test}
 {\it fold test} relies on the tilting or folding of the 
target geological material.
 If, for example, one wanted to establish the
antiquity of a particular set of directions, one could deliberately
sample units of like lithology, with different present attitudes 
(Figure~\ref{fig:fold}).  If the recovered directions are more tightly
grouped before  adjusting for tilt (as in the
lower left panel), then the
magnetization is likely to have been acquired after tilting.  On the other hand, if
directions become better grouped in the tilt adjusted coordinates (see upper right panel),
 one
has an argument in favor of a pre-tilt age of the magnetization.  
Methods for quantifying the tightness of grouping in various coordinate
systems will be discussed in  later chapters.


In the
\index{conglomerate test}
 {\it conglomerate test}, lithologies that are  desirable for
paleomagnetic purposes must be found in a conglomerate bed
(Figure~\ref{fig:conglom}).  In this
rare and happy circumstance, we can sample them and show that: 1) the
rock magnetic behavior is the same  for the conglomerate samples as
for those being used in the paleomagnetic study, 2) the directions
of the studied lithology are well grouped, (Figure~\ref{fig:conglom}) and
3) the directions from the conglomerate clasts 
are randomly oriented (see Figure~\ref{fig:conglom}).  If the directions of the clasts are not randomly distributed
(Figure~\ref{fig:conglom}), 
then presumably the conglomerate clasts (and, by inference, 
the paleomagnetic samples from the studied lithology as
well) were magnetized after deposition of the conglomerate. 
We will discuss statistical methods for deciding if a set of directions
is random in later chapters.  


\index{baked contact test}%
The {\it baked contact test} is illustrated in Figure~\ref{fig:baked}.  It
is similar to the conglomerate test in that we seek to determine whether the 
lithology in question has undergone pervasive secondary overprinting.  When
an igneous body intrudes into an existing 
\index{host rock}%
{\it host rock}, it heats (or bakes) the contact zone to above the Curie temperature of the host
rock.  The baked contact immediately adjacent to the intrusion should therefore
have the same remanence direction as the intrusive unit.  This
magnetization
may be in an entirely different direction from the pre-existing host rock. The
maximum temperature reached in the baked zone decreases away from the intrusion
and remagnetization is not complete.  Thus the NRM directions of the baked zone
gradually change from that of the intrusion to that of the host rock.  Such
a condition would argue against pervasive overprinting in the  host rock that post-dated the
intrusion,  and the age of the intrusion would provide an 
upper bound on the age of remanence in the host rock.

\vskip .5 in\noindent{\bf 
Suggested Supplemental Reading}

{\obeylines
\parskip 0pt
 \hskip 1em Chapters 8, 9: Collinson (1983) \nocite{collinson83}
}

\clearpage

\section{Problems}


{\parindent 0pt  \parskip 12pt

Before you start, make sure you have the most recent distribution of the PmagPy software (see Appendix~\ref{app:pmagpy}).  
  As always, a help message is generated for all these programs by typing  {\bf progname -h}
on the command line. 
  
{\bf Problem 1: }

The remanence vectors in the Chapter\_9 directory saved in {\it zijd\_example.dat} (see instructions for problems in Chapter 5 for downloading) were measured during the 
thermal demagnetization of a specimen.  The first column is the specimen name. The second is the temperature to which the specimen was heated, then cooled in zero field.  The next colums are intensity, declination and inclination respectively for each   treatment step.  

a) Write a python program to make a Zijderveld diagram using python.  
 
 Then follow these steps:  1) Read in the data.  2) Convert the vectors to $x, y, z$.  3) Plot $x$ versus $-y$ using some symbol and then connect those dots with a line.  This is the horizontal projection  of the vector so $x$ should be on the horizontal axis and $-y$ should be up. (Think about this!   You are plotting a map view and Y is the East direction.  So $+y$ should be to the right of $x$.)   4) Now plot$ x $versus $-z$.  Here again the projection is funny because $+z$ is the down direction. Therefore it should be down.  So, it is  $-z$ that is up.   Use a different symbol for these points and plot them on the same plot.  

b) Now plot the data  using the program {\bf zeq.py}. [Hint:  check the help message to figure out how...] 
Compare your output with that produced by my program {\bf zeq.py}.  Re-write your program until your program is right; you can cheat by looking in zeq.py and the two function modules {\bf pmag.py} and {\bf pmagplotlib.py}   if you have to, but make your program ``your own''.    



c)     Assuming these data have already been converted to geographic coordinates, what is the approximate direction (e.g. NE and up) of the low stability component
of magnetization? The high stability component of magnetization? What is
the most likely remanence carrying mineral in this sample?   Thinking about what you learned about VRM in Chapter 7, for the low stability component to be a VRM acquired over the last million years,  at what temperature would the rock have to have been held to acquire this component viscously over a million years?  


c) Use the {\bf zeq.py} to calculate best-fit lines through the two components and a great circle through all the data (leaving out the NRM step).  Which interpretation makes the most sense?  






{\bf Problem 2: }

Use the program {\bf sundec.py}  to estimate what the drilling azimuth was using the following sun compass information:   You are located at 35$^{\circ}$ N and 33$^{\circ}$ E.  The local
time is three hours ahead of Universal Time.  The shadow angle for the
drilling direction was 68$^{\circ}$ measured at 16:09 on May 23, 1994.

{\bf Problem 3:}
% problem 4.2 from Butler 1992:

The direction of NRM  for these problems is given in geographic coordinates along with the
attitude of dipping strata from which the site was collected.  For each problem, plot the NRM direction on an equal-area
projection (see Appendix~\ref{app:eqarea}). 
Then using the procedures outlined in Appendix~\ref{app:eqtilt} (or slight modifications thereof), determine
the ``structurally corrected'' direction of NRM that results from restoring the strata to horizontal.

a)    $I$ = �2$^{\circ}$, $D$ = 336$^{\circ}$, bedding dip = 41$^{\circ}$, dip direction = 351$^{\circ}$ (strike = 81$^{\circ}$).

b) I = 15$^{\circ}$, D = 227$^{\circ}$, bedding dip = 24$^{\circ}$, dip direction = 209$^{\circ}$ (strike = 299$^{\circ}$).

{\bf Problem 4:}
%Problem 4.3 from Butler 1992.

Now consider a more complex situation in which a paleomagnetic site has been collected from the
limb of a plunging fold. On the east limb of a plunging anticline, a direction of NRM is found to be
I = 33$^{\circ}$, D = 309$^{\circ}$. The bedding attitude of the collection site is dip = 29$^{\circ}$, strike = 210$^{\circ}$ (dip direction = 120$^{\circ}$, and the pole to bedding is azimuth = 300$^{\circ}$, inclination = 61$^{\circ}$). The trend and plunge of the
anticlinal axis are trend = 170$^{\circ}$, plunge = 20$^{\circ}$. Determine the direction of NRM from this site following
structural correction. Hint: First correct the NRM direction (and the pole to bedding) for the
plunge of the anticline. Then complete the structural correction of the NRM direction by restoring
the bedding (corrected for plunge) to horizontal.



{\bf Problem 5:}  

Write a python program  to convert
$D=8.1, I=45.2$  into geographic and tilt adjusted coordinates. Use the geographic coordinates as input to the tilt correction program.   The
orientation of the laboratory arrow on the specimen was: azimuth = 347$^{\circ}$;
plunge = 27$^{\circ}$.  The  strike was 135$^{\circ}$ and the dip was 21$^{\circ}$.
(NB: the convention is that the dip direction is to the ``right'' of 
the strike).   For this it would be handy to use the {\bf numpy} module which allows arrays, instead of simple lists.  To make an array $A$ of elements $ a_{ij}$:

$$
{\pmatrix{
 a_{11}&a_{12}&a_{13}\cr
 a_{21}&a_{22}&a_{23}\cr
 a_{31}&a_{32}&a_{33}\cr
}},
$$
\eject
\noindent the command would be:

\begin{verbatim}
import numpy
A=numpy.array([[a11,a12,a13],[a21,a22,a23],[a31,a32,a33]])
\end{verbatim}

The import command can be put at the beginning of the program as always.    Use your programs to convert direction to cartesion coordinates and back again.  

Compare your answer to the one given by {\bf di\_geo.py} and {\bf di\_tilt.py}.    Rewrite your code until you have it right.    NB: {\bf  di\_tilt.py}  uses dip and dip direction instead of strike and dip.  These are complete interchangeable, but dip and dip direction is unique, while strike and dip has requires some convention like ``dip to right of strike'').   



}

