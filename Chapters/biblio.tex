\documentclass[,plain]{tauxe}
\def\draftnote{\relax}
\usepackage{graphicx,longtable}
\makeatletter
\input audefinition.tex
\input bookfinal.aux
\makeatother
\def\addappheadtotoc{}
\def\bm{}
\let\matrix\array

\markboth{bibliography}{bibliography}
\begin{document}
\setcounter{page}{463}
\sectionrun{Bibliography}


\begin{thebibliography}{}

\bibitem{}%[Abramowitz \& Stegun, 1970]{abramowitz70}
Abramowitz, M., \& Stegun, I.~A., Eds. (1970). {\it Handbook of Mathematical Functions}, volume~55 of {\it Applied Mathematics Series}. Washington, DC: National Bureau of Standards.

\bibitem{}%[Aitken et~al., 1981]{aitken81}
Aitken, M., Alcock, P., Bussel, G., \& Shaw, C. (1981). Archaeomagnetic determination of the past geomagnetic intensity using ancient ceramics: allowance for anisotropy. {\it Archaeometry}, 23, 53--64.

\bibitem{}%[Aitken et~al., 1988]{aitken88}
Aitken, M.~J., Allsop, A.~L., Bussell, G.~D., \& Winter, M.~B. (1988). Determination of the intensity of the Earth's magnetic field during archeological times: reliability of the Thellier technique. {\it Rev. Geophys.}, 26, 3--12.

\bibitem{}%[Alvarez et~al., 1977]{alvarez77}
Alvarez, W., Arthur, M.~A., Fischer, A.~G., Lowrie, W., Napoleone, G., Premoli-Silva, I., \& Roggenthen, W.~M. (1977). Type section for the Late Cretaceous-Paleocene reversal time scale. {\it Geol. Soc. Amer. Bull.}, 88, 383--389.

\bibitem{}%[Anonymous, 1979]{anonymous79}
Anonymous (1979). Magnetostratigraphic polarity units---a supplementary chapter of the {ISSC} International stratigraphic guide. {\it Geology}, 7, 578--583.

\bibitem{}%[Anson \& Kodama, 1987]{anson87}
Anson, G.~L., \& Kodama, K.~P. (1987). Compaction-induced inclination shallowing of the post-depositional remanent magnetization in a synthetic sediment. {\it Geophys. J. Roy. Astr. Soc.}, 88, 673--692.

\bibitem{}%[Aurnou et~al., 2003]{aurnou03}
Aurnou, J., Andreadis, S., Zhu, L., \& Olson, P. (2003). Experiments on convection in Earth's core tangent cylinder. {\it Earth Planet. Sci. Lett.}, 212(1--2), 119--134.

\bibitem{}%[Backus et~al., 1996]{backus96}
Backus, G., Parker, R.~L., \& Constable, C. (1996). {\it Foundations of Geomagnetism}. Cambridge: Cambridge University Press.

\bibitem{}%[Balsley \& Buddington, 1960]{balsley60}
Balsley, J.~R., \& Buddington, A.~F. (1960). Magnetic susceptibility anisotropy and fabric of some Adirondack granites and orthogneisses. {\it Amer. J. Sci.}, 258A, 6--20.

\bibitem{}%
Banerjee, S. K. (1971). New grain size limits for paleomagnetic
stability in hematite. {\it Nature Phys. Sci.}, 232, 15--16.

\bibitem{}%[Banerjee, 1991]{banerjee91}
Banerjee, S.~K. (1991). Magnetic properties of Fe-Ti oxides. In D.~H.
Lindsley (Ed.), {\it Oxide Minerals: Petrologic and Magnetic Significance}, volume~25 of {\it Reviews in Mineralogy} (pp.\ 107--128). Washington, DC: Mineralogical Society of America.

\bibitem{}%[Banerjee et~al., 1981]{banerjee81}
Banerjee, S.~K., King, J., \& Marvin, J. (1981). A rapid method for magnetic granulometry with applications to environmental studies. {\it Geophys. Res. Lett.}, 8, 333--336.

\bibitem{}%[Behrensmeyer \& Tauxe, 1982]{behrensmeyer82}
Behrensmeyer, A.~K., \& Tauxe, L. (1982). Isochronous fluvial systems in Miocene deposits of Northern Pakistan. {\it Sedimentology}, 29, 331--352.

\bibitem{}%[Ben-Yosef et~al., 2008a]{benyosef08}
Ben-Yosef, E., Ron, H., Tauxe, L., Agnon, A., Genevey, A., Levy, T., Avner, U., \& Najjar, M. (2008a). Application of copper slag in geomagnetic archaeointensity research. {\it J. Geophys. Res.}, 113, doi:10.1029/2007JB005235.

\bibitem{}%[Ben-Yosef et~al., 2008b]{benyosef08b}
Ben-Yosef, E., Tauxe, L., Ron, H., Agnon, A., Avner, U., Najjar, M., \& Levy, T. (2008b). A new approach for geomagnetic archeointensity research: insights on ancient matellurgy in the Southern Levant. {\it J. Archael. Sci.}, 35, 2863--2879.

\bibitem{}%[Berggren et~al., 1995]{berggren95}
Berggren, W., Kent, D., Swisher~III, C., \& Aubry, M.-P. (1995). A revised Cenozoic geochronology and chronostratigraphy. In\break W. Berggren, D. Kent, M.-P. Aubry, \& J. Hardenbol (Eds.), {\it Geochronology Time Scales and Global Stratigraphic Correlation}\break (pp.\ 129--212). Tulsa, OK: SEPM.

\bibitem{}%[Besse \& Courtillot, 2002]{besse02}
Besse, J., \& Courtillot, V. (2002). Apparent and true polar wander and the geometry of the geomagnetic field over the last 200 Myr. {\it J. Geophys. Res.}, 107, doi:10.1029/2000JB000050.

\bibitem{}%[Bingham, 1974]{bingham74}
Bingham, C. (1974). An antipodally symmetric distribution on the sphere. {\it Ann. Statist.}, 2, 1201--1225.

\bibitem{}%[Bitter, 1931]{bitter31}
Bitter, F. (1931). On inhomogeneities in the magnetization of ferromagnetic materials. {\it Phys. Rev.}, 38, 1903--1905.

\bibitem{}%[Bol'shakov \& Shcherbakova, 1979]{bolshakov79}
Bol'shakov, A., \& Shcherbakova, V. (1979). A thermomagnetic criterion for determining the domain structure of ferrimagnetics. {\it Izv. Phys. Solid Earth}, 15, 111--117.

\bibitem{}%[Bonhommet \& Z\"ahringer, 1969]{bonhommet69}
Bonhommet, N., \& Z\"ahringer, J. (1969). Paleomagnetism and potassium argon age determinations of the Laschamp geomagnetic polarity event. {\it Earth Planet. Sci. Lett.}, 6, 43--46.

\bibitem{}%[Borradaile, 1988]{borradaile88}
Borradaile, G.~J. (1988). Magnetic susceptibility, petrofabrics and strain. {\it Tectonophysics}, 156, 1--20.

\bibitem{}%[Borradaile, 2003]{borradaile03}
Borradaile, G.~J. (2003).
{\it Statistics of Earth Science Data: Their Distribution in Time, Space, and Orientation}. Berlin: Springer.

\bibitem{}%[Brunhes, 1906]{brunhes06}
Brunhes, B. (1906).
Recherches sur le direction d'aimantation des roches volcaniques.
{\it J. Phys.}, 5, 705--724.

\bibitem{}%[Bullard et~al., 1965]{bullard65}
Bullard, E.~C., Everett, J.~E., \& Smith, A.~G. (1965).
A symposium on continential drift---IV. the fit of the continents around the Atlantic.
{\it Phil. Trans. Roy. Soc.}, 258, 41--51.

\bibitem{}%[Busse, 1983]{busse83}
Busse, F. (1983).
A model of mean zonal flows in the major planets.
{\it Geophys. Astrophys. Fluid Dyn.}, 23, 153--174.

\bibitem{}%[Butler, 1992a]{butler92b}
Butler, R.~F. (1992a).
Comment on ``High-latitude paleomagnetic poles from Middle Jurassic
plutons and Moat volcanics in New England and the controversy regarding Jurassic {APW} for North America'' by {M}. Van Fossen and {D}. {V}. Kent.
{\it J. Geophys. Res.}, 97, 1801--1802.

\bibitem{}%[Butler, 1992b]{butler92}
Butler, R.~F. (1992b).
{\it Paleomagnetism: Magnetic Domains to Geologic Terranes}.
Boston: Blackwell Scientific Publications.

\bibitem{}%[Butler \& Banerjee, 1975]{butler75}
Butler, R.~F., \& Banerjee, S.~K. (1975).
Theoretical single domain grain-size range in magnetite and titanomagnetite.
{\it J. Geophys. Res.}, 80, 4049--4058.

\bibitem{}%[Cande \& Kent, 1992]{cande92}
Cande, S.~C., \& Kent, D.~V. (1992).
A new geomagnetic polarity time scale for the late Cretaceous and Cenozoic.
{\it J. Geophys. Res.}, 97, 13917--13951.

\bibitem{}%[Cande \& Kent, 1995]{cande95}
Cande, S.~C., \& Kent, D.~V. (1995).
Revised calibration of the geomagnetic polarity timescale for the late Cretaceous and Cenozoic.
{\it J. Geophys. Res.}, 100, 6093--6095.

\bibitem{}%[Carter-Stiglitz et~al., 2006]{carterstiglitz06}
Carter-Stiglitz, B., Solheid, P., Egli, R., \& Chen, A. (2006).
Tiva Canyon Tuff (II): near single domain standard reference material available.
{\it The IRM Quarterly}, 16(1), 1.

\bibitem{}%[Cassata et~al., 2008]{cassata08}
Cassata, W., Singer, B., \& Cassidy, J. (2008).
Laschamp and Mono Lake geomagnetic excursions recorded in New Zealand.
{\it Earth Planet. Sci. Lett.}, 268, 76--88.

\bibitem{}%[Cassidy, 2006]{cassidy06}
Cassidy, J. (2006).
Geomagnetic excursion captured by multiple volcanoes in a monogenetic field.
{\it Geophys. Res. Lett.}, 33, L1310, doi:10.1029/2006GL027284.

\bibitem{}%[Channell et~al., 1984]{channell84}
Channell, J. E.~J., Lowrie, W., Pialli, P., \& Venturi, F. (1984).
Jurassic magnetostratigraphy from Umbrian (Italian) land sections.
{\it Earth Planet. Sci. Lett.}, 68, 309--325.

\bibitem{}%[Channell, 1992]{channell92}
Channell, J. E.~T. (1992).
Paleomagnetic data from Umbria (Italy): implications for the rotation of Adria and Mesozoic apparent polar wander paths.
{\it Tectonophysics}, 216, 365--378.

\bibitem{}%[Channell, 2006]{channell06}
Channell, J. E.~T. (2006).
Late Brunhes polarity excursions (Mono Lake, Laschamp, Iceland Basin and Pringle Falls) recorded at ODP Site 919 (Irminger Basin).
{\it Earth Planet. Sci. Lett.}, 244, 378--393.

\bibitem{}%[Channell et~al., 1995]{channell95}
Channell, J. E.~T., Erba, E., Nakanishi, M., \& Tamaki, K. (1995).
Late Jurassic--Early Cretaceous time scales and oceanic magnetic anomaly block models.
In W. Berggren, D. Kent, M. Aubry, \& J. Hardenbol (Eds.), {\it Geochronology, Time Scales and Stratigraphic Correlation}, volume~54 (pp.\ 51--64). SEPM Spec. Pub.

\bibitem{}%[Cisowski, 1981]{cisowski81}
Cisowski, S. (1981).
Interacting vs. non-interacting single domain behavior in natural and synthetic samples.
{\it Phys. Earth Planet. Inter.}, 26, 56--62.

\bibitem{}%[Clement, 1991]{clement91b}
Clement, B.~M. (1991).
Geographical distribution of transitional {VGP}s: evidence for non-zonal equatorial symmetry during the Matuyama-Brunhes geomagnetic reversal.
{\it Earth Planet. Sci. Lett.}, 104, 48--58.

\bibitem{}%[Clement, 2004]{clement04}
Clement, B.~M. (2004).
Dependence of the duration of geomagnetic polarity reversals on site latitude.
{\it Nature}, 428(6983), 637--640.

\bibitem{}%[Clement \& Kent, 1984]{clement84}
Clement, B.~M., \& Kent, D.~V. (1984).
A detailed record of the Lower Jaramillo polarity transition from a southern hemisphere, deep-sea sediment core.
{\it Jour. Geophys. Res.}, 89, 1049--1058.

\bibitem{}%[Coe, 1967]{coe67}
Coe, R.~S. (1967).
The determination of paleo-intensities of the Earth's magnetic field with emphasis on mechanisms which could cause non-ideal behavior in Thellier's method.
{\it J. Geomag. Geoelectr.}, 19, 157--178.

\bibitem{}%[Coe et~al., 1978]{coe78}
Coe, R.~S., Gromm\'e, S., \& Mankinen, E.~A. (1978).
Geomagnetic paleointensities from radiocarbon-dated lava flows on Hawaii and the question of the Pacific nondipole low.
{\it J. Geophys. Res.}, 83, 1740--1756.

\bibitem{}%[Coffey et~al., 1996]{coffey96}
Coffey, W., Kalmykov, Y., \& Waldron, J. (1996).
{\it The Langevin Equation with Applications in Physics, Chemistry and Electrical Engineering}, volume~11 of {\it World Scientific Series in Contemporary Chemical Physics}.
Singapore: World Scientific.

\bibitem{}%[Collinson, 1965]{collinson65}
Collinson, D.~W. (1965).
{DRM} in sediments.
{\it J. Geophys. Res.}, 70, 4663--4668.

\bibitem{}%[Collinson, 1983]{collinson83}
Collinson, D.~W. (1983).
{\it Methods in Rock Magnetism and Paleomagnetism}.
London: Chapman and Hall.

\bibitem{}%[Constable \& Parker, 1988]{constable88}
Constable, C., \& Parker, R.~L. (1988).
Statistics of the geomagnetic secular variation for the past 5 m.y.
{\it J. Geophys. Res.}, 93,\break 11569--11581.

\bibitem{}%[Constable \& Tauxe, 1990]{constable90}
Constable, C., \& Tauxe, L. (1990).
The bootstrap for magnetic susceptibility tensors.
{\it J. Geophys. Res.}, 95, 8383--8395.

\bibitem{}%[Constable, 2003]{constable03}
Constable, C.~G. (2003).
Geomagnetic reversals: rates, timescales, preferred paths, statistical models and simulations.
In C. Jones, A. Soward, \& K. Zhang (Eds.), {\it Earth's Core and Lower Mantle, The Fluid Mechanics of Astrophysics and Geophysics}. London: Taylor and Francis.

\bibitem{}%[Constable et~al., 2000]{constable00}
Constable, C.~G., Johnson, C.~L., \& Lund, S.~P. (2000).
Global geomagnetic field models for the past 3000 years: transient or permanent flux lobes?
{\it Phil. Trans. Roy. Soc. London, Series A}, 358(1768), 991--1008.

\bibitem{}%[Cook, 2001]{cook01}
Cook, A. (2001).
Edmond Halley and the magnetic field of the Earth.
{\it Notes Rec. Roy. Soc. London}, 55, 473--490.

\bibitem{}%[Cox, 1969]{cox69}
Cox, A. (1969).
Research note: Confidence limits for the precision parameter, K.
{\it Geophys. J. Roy. Astron. Soc.}, 17, 545--549.

\bibitem{}%[Cox \& Doell, 1960]{cox60}
Cox, A., \& Doell, R. (1960).
Review of Paleomagnetism.
{\it Geol. Soc. Amer. Bull.}, 71, 645--768.

\bibitem{}%[Cox et~al., 1964]{cox64}
Cox, A., Doell, R.~R., \& Dalrymple, G.~B. (1964).
Reversals of the Earth's magnetic field.
{\it Science}, 144, 1537--1543.

\bibitem{}%[Cox et~al., 1963]{cox63}
Cox, A.~V., Doell, R.~R., \& Dalrymple, G.~B. (1963).
Geomagnetic polarity epochs and Pleistocene geochronometry.
{\it Nature}, 198, 1049--1051.

\bibitem{}%[Creer et~al., 1959]{creer59}
Creer, K., Irving, E., \& Nairn, A. (1959).
Paleomagnetism of the Great Whin Sill.
{\it Geophys. J. Int.}, 2, 306--323.

\bibitem{}%[Creer, 1983]{creer83}
Creer, K.~M. (1983).
Computer synthesis of geomagnetic paleosecular variations.
{\it Nature}, 304, 695--699.

\bibitem{}%[Cronin et~al., 2001]{cronin01}
Cronin, M., Tauxe, L., Constable, C., Selkin, P., \& Pick, T. (2001).
Noise in the quiet zone.
{\it Earth Planet. Sci. Lett.}, 190, 13--30.

\bibitem{}%[Cullity, 1972]{cullity72}
Cullity, B. (1972).
{\it Introduction to Magnetic Materials}.
Reading: Addison-Wesley Publishing Company.

\bibitem{}%[Dankers \& Zijderveld, 1981]{dankers81}
Dankers, P. H.~M., \& Zijderveld, J. D.~A.\break (1981).
Alternating field demagnetization of rocks and the problem of gyromagnetic remanence.
{\it Earth Planet. Sci. Lett.}, 53,\break 89--92.

\bibitem{}%[David, 1904]{david04}
David, P. (1904).
Sur la stabilit\'e de la direction d'aimantation dans quelques roches volcaniques.
{\it C. Roy. Acad. Sci. Paris}, 138, 41--42.

\bibitem{}%[Day et~al., 1977]{day77}
Day, R., Fuller, M.~D., \& Schmidt, V.~A. (1977).
Hysteresis properties of titanomagnetites: grain size and composition dependence.
{\it Phys. Earth Planet. Inter.}, 13, 260--266.

\bibitem{}%[Deamer \& Kodama, 1990]{deamer90}
Deamer, G.~A., \& Kodama, K.~P. (1990).
Compaction-induced inclination shallowing in synthetic and natural clay-rich sediments.
{\it J. Geophys. Res.}, 95, 4511--4529.

\bibitem{}%[Dekkers \& B\"ohnel, 2006]{dekkers06}
Dekkers, M., \& B\"ohnel, H. (2006).
Reliable absolute paleointensities independent of magnetic domain state.
{\it Earth Planet. Sci. Lett.}, 248, 508--517.

\bibitem{}%[Dekkers, 1988]{dekkers88}
Dekkers, M.~J. (1988).
Magnetic properties of natural pyrrhotite {I}. Behaviour of initial susceptibility and saturation magnetization related rock magnetic parameters in a grain-size dependent framework.
{\it Phys. Earth Planet. Inter.}, 52, 376--393.

\bibitem{}%
Dekkers, M. J. (1989a). Magnetic properties of natural goethite I.
Grain size dependence of some low and high field related rock
magnetic parameters measured at room temperature. {\it Geophys. J.},
97, 323--340.

\bibitem{}%
Dekkers, M. J. (1989b). Magnetic properties of natural pyrrhotite II.
High and low temperature behaviors of Jrs and TRM as a function of
grain size. {\it Phys. Earth Planet. Inter.}, 57, 266--283.

\bibitem{}%
Dekkers, M. J., Mattei, J. L., Fillion, G., \& Rochette, P. (1989).
Grain-size dependence of the magnetic behavior of pyrrhotite during
its low temperature transition at 34 K. {\it Geophys. Res. Lett.},
16, 855--858.

\bibitem{}%[DeMets et~al., 1994]{demets94}
DeMets, C., Gordon, R.~G., Argus, D.~F., \& Stein, S. (1994).
Effect of recent revisions to the geomagnetic reversal time scale on estimates of current plate motions.
{\it Geophys. Res. Lett.}, 21, 2191--2194.

\bibitem{}%[Doell \& Dalrymple, 1966]{doell66}
Doell, R., \& Dalrymple, G. (1966).
Geomagnetic polarity epochs: A new polarity event and the age of the Brunhes-Matuyama boundary.
{\it Science}, 152, 1060--1061.

\bibitem{}%[Dunlop, 2002a]{dunlop02b}
Dunlop, D. (2002a).
Theory and application of the Day plot ($M_{rs}/M_s$ versus $H_{cr}/H_c$) 2. Application to data for rocks, sediments, and soils.
{\it J. Geophys. Res.}, 107, doi:10.1029/2001JB000487.

\bibitem{}%[Dunlop \& Argyle, 1997]{dunlop97b}
Dunlop, D., \& Argyle, K. (1997).
Thermoremanence, anhysteretic remanence and susceptibility of submicron magnetites: nonlinear field dependence and variation\break with grain size.
{\it J. Geophys. Res.}, 102, 20199--20210.

\bibitem{}%[Dunlop \& \"Ozdemir, 1997]{dunlop97}
Dunlop, D., \& \"Ozdemir, O. (1997).
{\it Rock Magnetism: Fundamentals and Frontiers}.
Cambridge: Cambridge University Press.

\bibitem{}%[Dunlop \& \"Ozdemir, 2001]{dunlop01}
Dunlop, D., \& \"Ozdemir, O. (2001).
Beyond N\'eel's theories: thermal demagnetization of narrow-band partial thermoremanent magnetization.
{\it Phys. Earth Planet. Int.}, 126, 43--57.

\bibitem{}%[Dunlop, 2002b]{dunlop02a}
Dunlop, D.~J. (2002b).
Theory and application of the Day plot ($M_{rs}/M_s$ versus $H_{cr}/H_c$) 1. Theoretical curves and tests using titanomagnetite data.
{\it J. Geophys. Res.}, 107, doi:10.1029/2001JB000486.

\bibitem{}%[Dunlop \& Xu, 1994]{dunlop94}
Dunlop, D.~J., \& Xu, S. (1994).
Theory of partial thermoremanent magnetization in multidomain grains, 1. Repeated identical barriers to wall motion (single microcoercivity).
{\it J. Geophys. Res.}, 99, 9005--9023.

\bibitem{}%[Dupont-Nivet et~al., 2002]{dupont-nivet02}
Dupont-Nivet, G., Guo, Z., Butler, R., \& Jia, C. (2002).
Discordant paleomagnetic direction in Miocene rocks from the central Tarim Basin: evidence for local deformation and inclination shallowing.
{\it Earth Planet. Sci. Lett.}, 199, 473--482.

\bibitem{}%[Efron \& Tibshirani, 1993]{efron93}
Efron, B., \& Tibshirani, R. (1993).
{\it An Introduction to the Bootstrap}, volume~57 of {\it Monographs on Statistics and Applied Probability}.
New York: Chapman and Hall.

\bibitem{}%[Egli, 2003]{egli03}
Egli, R. (2003).
Analysis of the field dependence of remanent magnetization curves.
{\it J. Geophy. Res.}, 108(B2).

\bibitem{}%[Elsasser, 1958]{elsasser58}
Elsasser, W. (1958).
The Earth as a dynamo.
{\it Sci. Am.}, 198, 44--48.



\bibitem{}%[Evans \& Heller, 2003]{evans03}
Evans, M., \& Heller, F. (2003). {\it Environmental Magnetism:
Principles and Applications of Enviromagnetics}. San Diego:
Academic Press.

\bibitem{}%[Evans \& McElhinny, 1969]{evans69}
Evans, M.~E., \& McElhinny, M.~W. (1969).
An investigation of the origin of stable remanence in magnetite-bearing igneous rocks.
{\it J. Geomag. Geoelectr.}, 21, 757--773.

\bibitem{}%[Fabian, 2003]{fabian03}
Fabian, K. (2003).
Some additional parameters to estimate domain state from isothermal magnetization measurements.
{\it Earth Planet. Sci. Lett.}, 213(3--4), 337--345.

\bibitem{}%[Fabian et~al., 1996]{fabian96}
Fabian, K., Andreas, K., Williams, W., Heider, F., Leibl, T., \& Huber, A. (1996).
Three-dimensional micromagnetic calculations for magnetite using FFT.
{\it Geophys. J. Int.}, 124, 89--104.

\bibitem{}%[Feinberg et~al., 2005]{feinberg05}
Feinberg, J., Scott, G., Renne, P., \& Wenk, H.-R. (2005).
Exsolved magnetite inclusions in silicates: features determining their remanence behavior.
{\it Geology}, 33, 513--516: doi:10.1130/G21290.1.

\bibitem{}%[Fisher et~al., 1987]{fisher87}
Fisher, N.~I., Lewis, T., \& Embleton, B. J.~J. (1987).
{\it Statistical Analysis of Spherical Data}.
Cambridge: Cambridge University Press.

\bibitem{}%[Fisher, 1953]{fisher53}
Fisher, R.~A. (1953).
Dispersion on a sphere.
{\it Proc. Roy. Soc. London, Series A}, 217, 295--305.

\bibitem{}%[Fletcher \& O'Reilly, 1974]{fletcher74}
Fletcher, E., \& O'Reilly, O. (1974).
Contribution of Fe$^{2+}$ ions to the magnetocrystalline anisotropy constant K$_1$ of Fe$_{(3-x)}$Ti$_x$O$_4 (0<x<0.1)$.
{\it J. Phys. C: Sol. State Phys.}, 7, 171--178.

\bibitem{}%[Flinn, 1962]{flinn62}
Flinn, D. (1962).
On folding during three-dimensional progressive deformation.
{\it Geol. Soc. London Quart. J.}, 118,\break 385--433.

\bibitem{}%[Folgheraiter, 1899]{folgheraiter1899}
Folgheraiter, G. (1899).
Sur les variations s\'eculaires de l'inclinaison magn\'etique dans l'antiquit\'e.
{\it J. de Phys.}, 5, 660--667.

\bibitem{}%[Forsythe \& Chisholm, 1994]{forsythe94}
Forsythe, R., \& Chisholm, L. (1994).
Paleomagnetic and structural contraints on rotations in the North Chilean Coast Ranges.
{\it J. South Am. Earth Sci.}, 7, 279--294.

\bibitem{}%[Frost \& Lindsley, 1991]{frost91}
Frost, B., \& Lindsley, D. (1991).
The occurrence of Fe-Ti oxides in igneous rocks.
In D. Lindsley (Ed.), {\it Oxide Minerals: Petrologic and Magnetic
Significance}, volume~25 of {\it Reviews in Mineralogy} (pp.\ 433--486). Chantilly: Mineralogical Society of America.

\bibitem{}%[Galbrun, 1985]{galbrun85}
Galbrun, B. (1985).
Magnetostratigraphy of the Berriasian stratotype section (Berrias, France).
{\it Earth Planet. Sci. Lett.}, 74, 130--136.

\bibitem{}%[Gapeyev \& Tsel'movich, 1988]{gapeyev88}
Gapeyev, A., \& Tsel'movich, V. (1988).
Stages of oxidation of titanomagnetite grains in igneous rocks (in Russian).
{\it Viniti N. Moscow}, 1331-B89, 3--8.

\bibitem{}%[Gee et~al., 1993]{gee93}
Gee, J., Staudigel, H., Tauxe, L., Pick, T., \& Gallet, Y. (1993).
Magnetization of the La Palma seamount series: implications for seamount paleopoles.
{\it J. Geophys. Res.}, 98, 11743--11768.

\bibitem{}%[Gee \& Kent, 2007]{gee07}
Gee, J.~S., \& Kent, D.~V. (2007).
Source of oceanic magnetic anomalies and the geomagetic polarity timescale.
In M. Kono (Ed.), {\it Geomagnetism}, volume~5 of {\it Treatise on
Geophysics} (pp.\ 455--507). Amsterdam: Elsevier.

\bibitem{}%[Gee et~al., 2008]{gee08}
Gee, J.~S., Tauxe, L., \& Constable, C. (2008).
AMSSpin---A LabVIEW program for measuring the anisotropy of magnetic susceptibility (AMS) with the Kappabridge KLY-4S.
{\it Geochem. Geophys. Geosyst.}, 9, Q08Y02,doi:10.1029/2008GC001976.

\bibitem{}%[Genevey \& Gallet, 2003]{genevey03}
Genevey, A., \& Gallet, Y. (2003).
Eight thousand years of geomagnetic field intensity variations in the eastern Mediterranean.
{\it J. Geophys. Res.}, 108, doi:10.1029/2001JB001612.

\bibitem{}%[Genevey et~al., 2008]{genevey08}
Genevey, A., Gallet, Y., Constable, C.~G., Korte, M., \& Hulot, G. (2008).
ArcheoInt: an upgraded compilation of geomagnetic field intensity data for the past ten millennia and its application to the recovery of the past dipole moment.
{\it Geochem. Geophys. Geosyst.}, 9, doi:10.1029/2007GC001881.

\bibitem{}%[Gibbs, 1985]{gibbs85}
Gibbs, R. (1985).
Estuarine flocs: their size, settling velocity and density.
{\it J. Geophys. Res.}, 90, 3249--3251.

\bibitem{}%[Gilder et~al., 2001]{gilder01}
Gilder, S., Chen, Y., \& Sen, S. (2001).
Oligo-Miocene magnetostratigarphy and rock magnetism of the Xishuigou section, Subei (Gansu Province, western China) and implications for shallow inclinations in central Asia.
{\it J. Geophys. Res.}, 106, 30505--30521.

\bibitem{}%[Glatzmaier \& Roberts, 1995]{glatzmaier95}
Glatzmaier, G., \& Roberts, P. (1995).
A three-dimensional self-consistent computer simulation of a geomagnetic field reversal.
{\it Nature}, 377, 203--209.

\bibitem{}%[Glatzmaier \& Roberts, 1996]{glatzmaier96}
Glatzmaier, G., \& Roberts, P. (1996).
Rotation and magnetism of Earth's inner core.
{\it Science}, 274, 1887--1891.

\bibitem{}%[Glatzmaier et~al., 1999]{glatzmaier99}
Glatzmaier, G.~A., Coe, R.~S., Hongre, L., \& Roberts, P.~H. (1999).
The role of the Earth's mantle in controlling the frequency of geomagnetic reversals.
{\it Nature}, 401(6756), 885--890.

\bibitem{}%[Glen, 1982]{glen02}
Glen, W. (1982).
{\it The Road to Jaramillo}.
Stanford: Stanford University Press.

\bibitem{}%[Gordon et~al., 1984]{gordon84}
Gordon, R.~G., Cox, A., \& Hare, S.~O. (1984).
Paleomagnetic euler poles and the apparent polar wander and absolute motion of North America since the Carboniferous.
{\it Tectonics}, 3, 499--537.

\bibitem{}%[Gradstein et~al., 1995]{gradstein95}
Gradstein, F., Agterberg, F., Ogg, J., Hardenbol, J., Van~Veen, P., Thierry, J., \& Huang, Z. (1995).
A Triassic, Jurassic and Cretaceous time scale.
In W. Berggren, D. Kent, M.-P. Aubry, \& J. Hardenbol (Eds.), {\it Geochronology Time Scales and Global Stratigraphic Correlation} (pp.\ 95--126). Tulsa, OK:\break SEPM.

\bibitem{}%[Gradstein et~al., 2004]{gradstein04}
Gradstein, F., Ogg, J., \& Smith, A. (2004).
{\it Geologic Time Scale 2004}.
Cambridge: Cambridge University Press.

\bibitem{}%[Graham, 1949]{graham49}
Graham, J.~W. (1949).
The stability and significance of magnetism in sedimentary rocks.
{\it J. Geophys. Res.}, 54, 131--167.

\bibitem{}%[Gregor et~al., 1974]{gregor74}
Gregor, C., Mertzman, S., Nairn, A., \& Negendank, J. (1974).
The paleomagnetism of some Mesozoic and Cenozoic volcanic rocks from the Lebanon.
{\it Tectonophysics}, 21, 375--395.

\bibitem{}%[Gromm\'e et~al., 1969]{gromme69}
Gromm\'e, C.~S., Wright, T.~L., \& Peak, D.~L. (1969).
Magnetic properties and oxidation of iron-titanium oxide minerals in Alae and Makaopulhi Lava Lakes, Hawaii.
{\it J. Geophys. Res.}, 74, 5277--5293.

\bibitem{}%[Gubbins \& Herrero-Bervera, 2007]{gubbins07}
Gubbins, D., \& Herrero-Bervera, E. (2007).
{\it Encyclopedia of Geomagnetism and Paleomagnetism}.
Encyclopedia of Earth Sciences. Heidelberg:  Springer.

\bibitem{}%[Guyodo \& Valet, 1999]{guyodo99}
Guyodo, Y., \& Valet, J.~P. (1999).
Global changes in intensity of the Earth's magnetic field during the past 800 kyr.
{\it Nature}, 399(6733), 249--252.

\bibitem{}
Halgedahl, S., Day, R., \& Fuller, M. (1980). The effect of cooling
rate on the intensity of weak-field TRM in single-domain magnetite.
{\it J. Geophys. Res.}, 85, 3690--3698.


%\bibitem{}%[Halgedahl \& Fuller, 1980]{halgedahl80}
%Halgedahl, S., \& Fuller, M. (1980).
%Magnetic domain observations of nucleation processes in fine particles of intermediate titanomagnetite.
%{\it Nature}, 288, 70--72.

\bibitem{}%[Halgedahl \& Fuller, 1983]{halgedahl83}
Halgedahl, S., \& Fuller, M. (1983).
The dependence of magnetic domain structure upon magnetization state with emphasis upon nucleation as a mechanism for pseudo-single domain behavior.
{\it J. Geophys. Res.}, 88, 6505--6522.

\bibitem{}%[Hargraves, 1991]{hargraves91}
Hargraves, R.~B. (1991).
Distribution anisotropy: the cause of {AMS} in igneous rocks?
{\it Geophys. Res. Lett.}, 18, 2193--2196.

\bibitem{}%[Hargraves \& Onstott, 1980]{hargraves80}
Hargraves, R.~B., \& Onstott, T.~C. (1980).
Paleomagnetic results from some southern African kimberlites and their tectonic significance.
{\it J. Geophys. Res.}, 85, 3587--3596.

\bibitem{}%[Harrison, 1966]{harrison66}
Harrison, C. G.~A. (1966).
The paleomagnetism of deep sea sediments.
{\it J. Geophys. Res.}, 71, 3033--3043.

\bibitem{}%[Harrison \& Feinberg, 2008]{harrison08}
Harrison, R., \& Feinberg, J. (2008).
FORCinel: An improved algorithm for calculating first-order reversal curve (FORC) distributions using locally-weighted regression smoothing.
{\it Geochem. Geophys. Geosyst.}, doi:10.1029/2008GC001987.

\bibitem{}%[Hatakeyama \& Kono, 2002]{hatakeyama02}
Hatakeyama, T., \& Kono, M. (2002).
Geomagnetic field model for the last 5 My: time-averaged field and secular variation.
{\it Phys. Earth Planet. Int.}, 133, 181--215.

\bibitem{}%[Hays et~al., 1976]{hays76}
Hays, J.~D., Imbrie, J., \& Shackleton, N.~J. (1976).
Variations in the Earth's orbit: pacemaker of the ice ages.
{\it Science}, 194, 1121--1132.

\bibitem{}%[He et~al., 2008]{he08}
He, H., Pan, Y.~X., Tauxe, L., \& Qin, H. (2008).
Toward age determination of the Barremian-Aptian boundary M0r of the Early Cretaceous.
{\it Phys. Earth Planet. Int.}, 169, 41--48.

\bibitem{}%[Heider \& Hoffmann, 1992]{heider92}
Heider, F., \& Hoffmann, V. (1992).
Magneto-optical Kerr effect on magnetite crystals with externally applied magnetic fields.
{\it Earth Planet. Sci. Lett.}, 108, 131--138.

\bibitem{}%[Heider et~al., 1996]{heider96}
Heider, F., Zitzelsberger, A., \& Fabian, K. (1996).
Magnetic susceptibility and remanent coercive force in grown magnetite crystals from 0.1 $\mu$m to 6mm.
{\it Phys. Earth Planet. Inter.}, 93, 239--256.

\bibitem{}%[Heirtzler et~al., 1968]{heirtzler68}
Heirtzler, J.~R., Dickson, G.~O., Herron, E.~M., Pitman, W. C.~I., \& LePichon, X. (1968).
Marine magnetic anomalies geomagnetic\break field reversals, and motions of the ocean floor and continents.
{\it J. Geophys. Res.}, 73, 2119--2136.

\bibitem{}%[Helsley \& Steiner, 1969]{helsley69}
Helsley, C., \& Steiner, M. (1969).
Evidence for long intervals of normal polarity during the Cretaceous period.
{\it Earth Planet. Sci. Lett.}, 5, 325--332.

\bibitem{}%[Hext, 1963]{hext63}
Hext, G.~R. (1963).
The estimation of second-order tensors, with related tests and designs.
{\it Biometrika}, 50, 353--357.

\bibitem{}%[Hilgen, 1991]{hilgen91}
Hilgen, F.~J. (1991).
Astronomical calibration of Gauss to Matuyama sapropels in the Mediterranean and implication for the Geomagnetic Polarity Time Scale.
{\it Earth Planet. Sci. Lett.}, 104, 226--244.

\bibitem{}%[Hill et~al., 2005]{hill05}
Hill, M., Shaw, J., \& Herrero-Bervera, E. (2005).
Paleointensity record through the Lower Mammoth reversal from the Waianae volcano, Hawaii.
{\it Earth Planet. Sci. Lett.}, 230, 255--272.

\bibitem{}%[Hoffman \& Biggin, 2005]{hoffman05}
Hoffman, K.~A., \& Biggin, A.~J. (2005).
A rapid multi-sample approach to the determination of absolute paleointensity.
{\it J. Geophys. Res.}, 110, B12108, doi:10.1029/2005JB003646.

\bibitem{}%[Hoffman et~al., 1989]{hoffman89}
Hoffman, K.~A., Constantine, V.~L., \& Morse, D.~L. (1989).
Determinaton of absolute palaeointensity using a multi-specimen procedure.
{\it Nature}, 339, 295--297.

\bibitem{}%[Hoffmann et~al., 1999]{hoffmann99}
Hoffmann, V., Knab, M., \& Appel, E. (1999).
Magnetic susceptibility mapping of roadside pollution.
{\it J. Geochem. Explor.}, 66, 313--326.

\bibitem{}%[Hospers, 1955]{hospers55}
Hospers, J. (1955).
Rock magnetism and polar wandering.
{\it J. Geol.}, 63, 59--74.

\bibitem{}%[Hughen et~al., 2004]{hughen04}
Hughen, K., Lehman, S., Southon, J., Overpeck, J., Marchal, O., Herring, C., \& Turnbull, J. (2004).
C-14 activity and global carbon cycle changes over the past 50,000 years.
{\it Science}, 303(5655), 202--207.

\bibitem{}%[Hulot et~al., 2002]{hulot02}
Hulot, G., Eymin, C., Langlais, B., Mandea, M., \& Olsen, N. (2002).
Small-scale structure of the geodynamo inferred from Oersted and Magsat satellite data.
{\it Nature}, 416, 620--623.

\bibitem{}%[Irving, 1958]{irving58}
Irving, E. (1958).
Paleogeographic reconstruction from paleomagnetism.
{\it Geophys. J. Roy. Astr. Soc.}, 1, 224--237.

\bibitem{}%[Irving, 1964]{irving64}
Irving, E. (1964).
{\it Paleomagnetism and Its Application to Geological and Geophysical Problems}.
New York: John Wiley and Sons, Inc.

\bibitem{}%[Irving, 1979]{irving79}
Irving, E. (1979).
Paleopoles and paleolatitudes of North America and speculations about displaced terrains.
{\it Can. J. Earth Sci.}, 16, 669--694.

\bibitem{}%[Irving \& Ward, 1963]{irving63}
Irving, E., \& Ward, M. (1963).
A statistical model of the geomagnetic field.
{\it Pure Appl. Geophys.}, 57, 47--52.

\bibitem{}%[Irwin, 1987]{irwin87}
Irwin, J. (1987).
Some paleomagnetic constraints on the tectonic evolution of the coastal cordillera of central Chile.
{\it J. Geophys. Res.}, 92, 3603--3614.

\bibitem{}%[Jackson et~al., 2000]{jackson00}
Jackson, A., Jonkers, A. R.~T., \& Walker, M.~R. (2000).
Four centuries of geomagnetic secular variation from historical records.
{\it Phil. Trans. Roy. Soc. London, Series A}, 358(1768), 957--990.

\bibitem{}%[Jackson et~al., 2006]{jackson06}
Jackson, M., Carter-Stiglitz, B., Egli, R., \& Solheid, P. (2006).
Characterizing the superparamagnetic grain distribution f(V, Hk) by thermal fluctuation tomography.
{\it J. Geophys. Res.}, 111, B12S07, doi:10.1029/2006JB004514.

\bibitem{}%[Jackson et~al., 1990]{jackson90}
Jackson, M., Worm, H.~U., \& Banerjee, S.~K. (1990).
Fourier analysis of digital hysteresis data: rock magnetic applications.
{\it Phys. Earth Planet. Inter.}, 65, 78--87.

\bibitem{}%[Jackson et~al., 1991]{jackson91}
Jackson, M.~J., Banerjee, S.~K., Marvin, J.~A., Lu, R., \& Gruber, W. (1991).
Detrital remanence, inclination errors and anhysteretic remanence anisotropy: quantitative model and experimental results.
{\it Geophys. J. Int.}, 104, 95--103.

\bibitem{}%[Jelinek, 1978]{jelinek78}
Jelinek, V. (1978).
Statistical processing of anisotropy of magnetic susceptibility measured on groups of specimens.
{\it Studia Geophys. et Geol.}, 22, 50--62.

\bibitem{}%[Jelinek, 1981]{jelinek81}
Jelinek, V. (1981).
Characterization to the magnetic fabric of rocks.
{\it Tectonophysics}, 79, T63--T67.

\bibitem{}%[Jiles, 1998]{jiles98}
Jiles, D. (1991).
{\it Introduction to Magnetism and Magnetic Materials, Second Edition}.
Boca Raton: Taylor and Francis.

\bibitem{}%[Joffe \& Heuberger, 1974]{joffe74}
Joffe, I., \& Heuberger, R. (1974).
Hysteresis properties of distributions of cubic single-domain ferromagnetic particles.
{\it Phil. Mag.}, 314, 1051--1059.

\bibitem{}%[Johnson et~al., 2008]{johnson08}
Johnson, C.~L., Constable, C.~G., Tauxe, L., Barendregt, R., Brown, L., Coe, R., Layer, P., Mejia, V., Opdyke, N., Singer, B., Staudigel, H., \& Stone, D. (2008).
Recent investigations of the 0--5 Ma geomagnetic field recorded in lava flows.
{\it Geochem. Geophys. Geosyst.}, 9, Q04032, doi:10.1029/2007GC001696.

\bibitem{}%[Johnson et~al., 1948]{johnson48}
Johnson, E.~A., Murphy, T., \& Torreson, O.~W. (1948).
Pre-history of the Earth's magnetic field.
{\it Terr. Magn. Atmos. Elect.}, 53,\break 349--372.

\bibitem{}%[Johnson et~al., 1984]{johnson84}
Johnson, R., van~der Voo, R., \& Lowrie, W. (1984).
Paleomagnetism and late diagenesis of Jurassic carbonates from the Jura Mountains, Switzerland and France.
{\it Geol. Soc. Amer. Bull.}, 95, 478--488.

\bibitem{}%[Jupp \& Kent, 1987]{jupp87}
Jupp, P., \& Kent, J. (1987).
Fitting smooth paths to spherical data.
{\it Appl. Statist.}, 36, 34--46.

\bibitem{}%[Katari \& Bloxham, 2001]{katari01}
Katari, K., \& Bloxham, J. (2001).
Effects of sediment aggregate size on DRM intensity: a new theory.
{\it Earth Planet. Sci. Lett.}, 186(1), 113--122.

\bibitem{}%[Katari \& Tauxe, 2000]{katari00}
Katari, K., \& Tauxe, L. (2000).
Effects of surface chemistry and flocculation on the intensity of magnetization in redeposited sediments.
{\it Earth Planet. Sci. Lett.}, 181, 489--496.

\bibitem{}%[Kent et~al., 2002]{kent02}
Kent, D., Hemming, S., \& Turrin, B. (2002).
Laschamp excursion at Mono Lake?
{\it Earth Planet. Sci. Lett.}, 197, 151--164.

\bibitem{}%[Kent \& Smethurst, 1998]{kent98}
Kent, D., \& Smethurst, M. (1998).
Shallow bias of paleomagnetic inclinations in the Paleozoic and Precambrian.
{\it Earth Planet. Sci. Lett.}, 160, 391--402.

\bibitem{}%[Kent \& Olsen, 1999]{kent99b}
Kent, D.~V. \& Olsen, P. (1999).
Astronomically tuned geomagnetic polarity time scale for the Late Triassic.
{\it J. Geophys. Res.}, 104, 12831--12841.

\bibitem{}%[Kent et~al., 1995]{kent95}
Kent, D.~V., Olsen, P.~E., \& Witte, W.~K. (1995).
Late Triassic-earliest Jurassic geomagnetic polarity sequence and paleolatitudes from drill cores in the Newark rift basin, eastern North America.
{\it J. Geophys. Res.}, 100, 14965--14998.

\bibitem{}%[Kent, 1982]{kent82}
Kent, J.~T. (1982).
The Fisher-Bingham distribution on the sphere.
{\it J. R. Statist. Soc. B.}, 44, 71--80.

\bibitem{}%[King et~al., 1982]{king82}
King, J., Banerjee, S.~K., Marvin, J., \& Ozdemir, O. (1982).
A comparison of different magnetic methods for determining the relative grain size of magnetite in natural materials: some results from lake sediments.
{\it Earth Planet. Sci. Lett.}, 59, 404--419.

\bibitem{}%[King et~al., 1983]{king83}
King, J.~W., Banerjee, S.~K., \& Marvin, J. (1983).
A new rock magnetic approach to selecting sediments for geomagnetic paleointensity studies: application to paleointensity for the last 4000 years.
{\it J. Geophys. Res.}, 88, 5911--5921.

\bibitem{}%[King, 1955]{king55}
King, R.~F. (1955).
The remanent magnetism of artificially deposited sediments.
{\it Mon. Nat. Roy. Astr. Soc., Geophys. Suppl.}, 7, 115--134.

\bibitem{}%[Kirschvink, 1980]{kirschvink80}
Kirschvink, J.~L. (1980).
The least-squares line and plane and the analysis of paleomagnetic data.
{\it Geophys. J. Roy. Astron. Soc.}, 62, 699--718.

\bibitem{}%[Kirschvink et~al., 1997]{kirschvink97b}
Kirschvink, J.~L., Ripperdan, R., \& Evans, D. (1997).
Evidence for a large-scale reorganization of early Cambrian continental masses by inertial interchange true polar wander.
{\it Science}, 277, 541--545.

\bibitem{}%[Kluth et~al., 1982]{kluth82}
Kluth, C., Butler, R., Harding, L., Shafiqullah, M., \& Damon, P. (1982).
Paleomagnetism of Late Jurassic rocks in the Northern Canelo Hills, southeastern Arizona.
{\it J. Geophy. Res.}, 87, 7079--7086.

\bibitem{}%[Knight \& Walker, 1988]{knight88}
Knight, M.~D., \& Walker, G. P.~L. (1988).
Magma flow directions in dikes of the Koolau Comples, Oahu, determined from magnetic fabric studies.
{\it J. Geophys. Res.}, 93, 4301--4319.

\bibitem{}%[K\"onigsberger, 1938]{koenigsberger38}
K\"onigsberger, J. (1938).
Natural residual magnetism of eruptive rocks, Pt I, Pt II.
{\it Terr. Magn. and Atmos. Electr.}, 43, 119--127, 299--320.



\bibitem{}%[Kono, 1974]{kono74}
Kono, M. (1974).
Intensities of the Earth's magnetic field about 60 my ago determined from the Deccan Trap basalts, India.
{\it J. Geophys. Res.}, 79, 1135--1141.

\bibitem{}%[Kono, 2007a]{kono07}
Kono, M. (2007a).
Geomagnetism in Perspective.\break
In M. Kono (Ed.), {\it Geomagnetism}, volume~5 of\break {\it Treatise on
Geophysics} (pp.\ 1--30). Amsterdam:\hspace*{-3pt}\break Elsevier.

\bibitem{}%[Kono, 2007b]{kono07b}
Kono, M. (2007b).
{\it Treatise in Geophysics, vol. 5}. Amsterdam: 
Elsevier.

\bibitem{}%[Kono \& Ueno, 1977]{kono77}
Kono, M., \& Ueno, N. (1977).
Paleointensity determination by a modified Thellier method.
{\it Phys. Earth Planet. Inter.}, 13, 305--314.

\bibitem{}%[Kopp \& Kirschvink, 2008]{kopp08}
Kopp, R.~E., \& Kirschvink, J.~L. (2008).
The identification and biogeochemical interpretation of fossil magnetotactic bacteria.
{\it Earth-Sci. Rev.}, 86(1--4), 42--61.

\bibitem{}%[Korhonen et~al., 2008]{korhonen08}
Korhonen, K., Donadini, F., Riisager, P., \& Pesonen, L.~J. (2008).
GEOMAGIA50: An archeointensity database with PHP and MySQL.
{\it Geochem. Geophys. Geosyst.}, 9(Q04029), doi:10.1029/2007GC001893.

\bibitem{}%[Korte \& Constable, 2005]{korte05b}
Korte, M., \& Constable, C. (2005).\break
Continuous geomagnetic field models\break for the past 7 millennia: 2. CALS7K.\break
{\it Geochem. Geophys. Geosyst.}, 6, Q02H16,\break doi 10.1029/2004GC000801.

\bibitem{}%[Korte \& Constable, 2003]{korte03}
Korte, M., \& Constable, C.~G. (2003).
Continuous global geomagnetic field models for the past 3000 years.
{\it Phys. Earth Planet. Inter.}, 140, 73--89.

\bibitem{}%[Korte \& Constable, 2008]{korte08}
Korte, M., \& Constable, C.~G. (2008).
Spatial and temporal resolution of millennial scale geomagnetic field models.
{\it Advances in Space Research}, 41(1), 57--69.

\bibitem{}%[Korte et~al., 2005]{korte05}
Korte, M., Genevey, A., Constable, C., Frank, U., \& Schnepp, E. (2005).
Continuous geomagnetic field models for the past 7 millennia: 1. A new global data compilation.
{\it Geochem., Geophys., Geosyst.}, 6(Q02H15), Q02H15; DOI 10.1029/2004GC000800.

\bibitem{}%[Kowallis et~al., 1998]{kowallis98}
Kowallis, B., Christiansen, E., Deino, A., Peterson, F., Turner, C., Kunk, M., \& Obradovich, J. (1998).
The age of the Morrison Formation.
{\it Modern Geology}, 22, 235--260.

\bibitem{}%[Kruiver et~al., 2001]{kruiver01}
Kruiver, P., Dekkers, M., \& Heslop, D. (2001).
Quantification of magnetic coercivity components by the analysis of acquisition curves of isothermal remanent magnetisation.
{\it Earth Planet. Sci. Lett.}, 189(3-4),\break 269--276.

\bibitem{}%[LaBrecque et~al., 1977]{labrecque77}
LaBrecque, J.~L., Kent, D.~V., \& Cande, S.~C. (1977).
Revised magnetic polarity time scale for Late Cretaceous and Cenozoic time.
{\it Geology}, 5, 330--335.

\bibitem{}%[Laj \& Channell, 2007]{laj07}
Laj, C., \& Channell, J. E.~T. (2007).
Geomagnetic Excursions.
In M. Kono (Ed.), {\it Geomagnetism}, volume~5 of {\it Treatise on
Geophysics} (pp.\ 373--407). Amsterdam: Elsevier.

\bibitem{}%[Laj et~al., 2002]{laj02}
Laj, C., Kissel, C., Mazaud, A., Michel, E., Muscheler, R., \& Beer, J. (2002).
Geomagnetic field intensity, North Atlantic Deep Water circulation and atmospheric $\Delta^{14}$C during the last 50 kyr.
{\it Earth Planet. Sci. Lett.}, 200, 177--190.

\bibitem{}%[Langel, 1987]{langel87}
Langel, R. (1987).
The main geomagnetic field.
In J. Jacobs (Ed.), {\it Geomagnetism} (pp.\ 249--512). New York: Academic Press.

\bibitem{}%[Lanos et~al., 2005]{lanos05}
Lanos, P., LeGoff, M., Kovacheva, M., \& Schnepp, E. (2005).
Hierarchical modelling of archaeomagnetic data and curve estimation by moving average technique.
{\it Geophys. J. Int.}, 160, 440--476.

\bibitem{}%[Larson \& Pitman, 1972]{larson72}
Larson, R.~L., \& Pitman, W. C.~I. (1972).
World-wide correlation of Mesozoic magnetic anomalies, and its implications.
{\it Geol. Soc. Amer. Bull.}, 83, 3645--3662.

\bibitem{}%[Laskar et~al., 2004]{laskar04}
Laskar, J., Robutel, P., Joutel, F., Gastineau, M., Correia, A., \& Levrard, B. (2004).
A long-term numerical solution for the insolation quantities of the Earth.
{\it Astron. Astrophys.}, 428(doi:10.1051/0004-6361), 261--285.

\bibitem{}%[Lawrence et~al., 2009]{lawrence08}
Lawrence, K.~P., Tauxe, L., Staudigel, H., Constable, C., Koppers, A., McIntosh, W.~C., \& Johnson, C.~L. (2009).
Paleomagnetic field properties at high southern latitude.
{\it Geochem. Geophys. Geosyst.}, 10, doi:10.1029/2008GC002072.

\bibitem{}%[LeGoff et~al., 1992]{legoff92}
LeGoff, M., Henry, B., \& Daly, L. (1992).
Practical method for drawing a VGP path.
{\it Phys. Earth Planet. Inter.}, 70, 201--204.

\bibitem{}%[Levi \& Banerjee, 1976]{levi76}
Levi, S., \& Banerjee, S.~K. (1976).
On the possibility of obtaining relative paleointensities from lake sediments.
{\it Earth Planet. Sci. Lett.}, 29, 219--226.

\bibitem{}%[Love \& Constable, 2003]{love03}
Love, J., \& Constable, C.~G. (2003).
Gaussian statistics for paleomagnetic vectors.
{\it Geophys. J. Int.}, 152, 515--565.

\bibitem{}%[Lowes, 1974]{lowes74}
Lowes, F. (1974).
Spatial power spectum of the main geomagnetic field and extrapolation to the core.
{\it Geophys. J. Roy. Astron. Soc.}, 36, 717--730.

\bibitem{}%[Lowrie, 1990]{lowrie90}
Lowrie, W. (1990).
Identification of ferromagnetic minerals in a rock by coercivity and unblocking temperature properties.
{\it Geophys. Res. Lett.}, 17, 159--162.

\bibitem{}%[Lowrie \& Kent, 2004]{lowrie04}
Lowrie, W., \& Kent, D.~V. (2004).
Geomagnetic polarity timescales and reversal frequency regimes.
In J. Channell, D. Kent, W. Lowrie, \& J. Meert (Eds.), {\it Timescales of the Paleomagnetic Field}, volume 145 (pp.\ 117--129). Washington, DC: American Geophysical Union.

\bibitem{}%[Lund et~al., 1988]{lund88}
Lund, S.~P., Liddicoat, J., Lajoie, T. L.~K., \& Henyey, T.~L. (1988).
Paleomagnetic evidence for long-term (10$^4$ year) memory and periodic behavior in the Earth's core dynamo process.
{\it Geophys. Res. Lett.}, 15, 1101--1104.

\bibitem{}%[Maher \& Thompson, 1999]{maher99}
Maher, B.~A., \& Thompson, R., Eds. (1999).
{\it Quaternary Climates, Environments and Magnetism}.
Cambridge: Cambridge University Press.

\bibitem{}%[Mardia \& Zemrock, 1977]{mardia77}
Mardia, K.~V., \& Zemrock, P.~J. (1977).
Table of maximum likelihood estimates for the Bingham distribution.
{\it J. Statist. Comput. Simul.}, 6, 29--34.

\bibitem{}%[Masarik \& Beer, 1999]{masarik99}
Masarik, J., \& Beer, J. (1999).
Simulation of particle fluxes and cosmogenic nuclide production in the Earth's atmosphere.
{\it J. Geophys. Res.}, 104, 12099--12110.

\bibitem{}%[Mason \& Raff, 1961]{mason61}
Mason, R., \& Raff, A. (1961).
Magnetic survey off the west coast of North America, 40 degrees N. latitude to 52 degrees N. latitude.
{\it Geol. Soc. Amer. Bull.}, 72, 1267--1270.

\bibitem{}%[Masters et~al., 2000]{masters00}
Masters, G., Laske, G., Bolton, H., \& Dziewonski, A.~M. (2000).
The relative behavior of shear velocity, bulk sound speed, and compressional velocity in the mantle: implications for chemical and thermal structure.
In S. Karato, R. Forte, G. Liebermann, G. Masters, \& L. Stixrude (Eds.), {\it Earth's Deep Interior}, volume 117 of {\it AGU Monograph}. Washington, D.C.: American Geophysical Union.

\bibitem{}%[Matuyama, 1929]{matuyama29}
Matuyama, M. (1929).
On the direction of magnetisation of basalt in Japan,\break Tyosen and Manchuria.
{\it Proc. Imp. Acad. Jap.}, 5, 203--205.

\bibitem{}%[May \& Butler, 1986]{may86}
May, S., \& Butler, R. (1986).
North American Jurassic apparent polar wander: implications for plate motion, paleogeography and cordilleran tectonics.
{\it J. Geophys. Res.}, 91, 11519--11544.

\bibitem{}%[Mayergoyz, 1986]{mayergoyz86}
Mayergoyz, I. (1986).
Mathematical models of hysteresis.
{\it IEEE Trans. Magn.}, MAG-22, 603--608.

\bibitem{}%[McCabe et~al., 1983]{mccabe83}
McCabe, C., van~der Voo, R., Peacor, C.~R., Scotese, C.~R., \& Freeman, R. (1983).
Diagenetic magnetite carries ancient yet secondary remanence in some Paleozoic carbonates.
{\it Geology}, 11, 221--223.

\bibitem{}%[McDougall \& Tarling, 1963]{mcdougall63}
McDougall, I., \& Tarling, D. (1963).
Dating reversals of the Earth's magnetic fields.
{\it Nature}, 198, 1012--1013.

\bibitem{}%[McElhinnhy \& Lock, 1996]{mcelhinny96}
McElhinnhy, M., \& Lock, J. (1996).
IAGA paleomagnetic databases with Access.
{\it Surv. Geophys.}, 17, 575--591.

%\bibitem{}%[McElhinny \& Lock, 1996]{mcelhinny96b}
%McElhinny, M. \& Lock, J. (1996).
%IAGA paleomagnetic databases with Access.
%{\it Surv. in Geophysics}, 17, 575--591.

\bibitem{}%[McElhinny \& McFadden, 2000]{mcelhinny00}
McElhinny, M., \& McFadden, P. (2000).
{\it Paleomagnetism: Continents and Oceans}.
San Diego: Academic Press.

\bibitem{}%[McElhinny, 1964]{mcelhinny64}
McElhinny, M.~W. (1964).
Statistical significance of the fold test in paleomagnetism.
{\it Geophys. J. Roy. Astro. Soc.}, 8, 338--340.

\bibitem{}%[McElhinny \& McFadden, 1997]{mcelhinny97}
McElhinny, M.~W., \& McFadden, P.~L. (1997).
Palaeosecular variation over the past 5 Myr based on a new generalized database.
{\it Geophys. J. Int.}, 131(2), 240--252.

\bibitem{}%[McFadden \& Jones, 1981]{mcfadden81}
McFadden, P.~L., \& Jones, D.~L. (1981).
The fold test in paleomagnetism.
{\it Geophys. J. Roy. Astr. Soc.}, 67, 53--58.

\bibitem{}%[McFadden \& McElhinny, 1988]{mcfadden88}
McFadden, P.~L., \& McElhinny, M.~W. (1988).
The combined analysis of remagnetization circles and direct observations in paleomagnetism.
{\it Earth Planet. Sci. Lett.}, 87, 161--172.

\bibitem{}%[McFadden \& Reid, 1982]{mcfadden82}
McFadden, P.~L., \& Reid, A.~B. (1982).
Analysis of paleomagnetic inclination data.
{\it Geophys. J. Roy. Astr. Soc.}, 69, 307--319.

\bibitem{}%[Means, 1976]{means76}
Means, W. (1976).
{\it Stress and Strain: Basic Concepts of Continuum Mechanics for Geologists}.
Heidelberg:	Springer-Verlag.

\bibitem{}%[Mercanton, 1926]{mercanton26}
Mercanton, P. (1926).
Inversion de\break l'inclinaison magn\'etique terrestre aux ages g\'eologiques.
{\it Terr. Magn. Atmosph. Elec.}, 31, 187--190.

\bibitem{}%[Merrill et~al., 1996]{merrill96}
Merrill, R.~T., McElhinny, M.~W., \& McFadden, P.~L. (1996).
{\it The Magnetic Field of the Earth: Paleomagnetism, the Core, and the Deep Mantle}.
San Diego: Academic Press.

\bibitem{}%[Mochizuki et~al., 2006]{mochizuki06}
Mochizuki, N., Tsunakawa, H., Shibuya, H., Cassidy, J., \& Smith, I. (2006).
Paleointensities of the Auckland geomagnetic excursions by the LTD-DHT Shaw method.
{\it Phys. Earth Planet. Int.}, 154, 168--179.

\bibitem{}%[Morel \& Irving, 1981]{morel81}
Morel, P., \& Irving, E. (1981).
Paleomagnetism and the evolution of Pangea.
{\it J. Geophys. Res.}, 86, 1858--1872.

\bibitem{}%[Moskowitz et~al., 2008]{moskowitz08}
Moskowitz, B., Bazylinski, D., Egli, R., Frankel, R., \& Edwards, K. (2008).
Magnetic properties of marine magnetotactic bacteria in a seasonally stratified coastal pond (Salt Pond, MA, USA).
{\it Geophys. J. Int.}, 174, 75--92.

\bibitem{}%[Moskowitz, 1993]{moskowitz93b}
Moskowitz, B.~M. (1993).
High-temperature magnetostriction of magnetite and titanomagnetites.
{\it J. Geophy. Res.}, 98, 359--371.

\bibitem{}%[Moskowitz \& Banerjee, 1981]{moskowitz81}
Moskowitz, B.~M., \& Banerjee, S.~K. (1981).\break
A comparison of the magnetic properties\break of synthetic titanomaghemites and some\break ocean basalts.
{\it J. Geophys. Res.}, 86, 11869--\break 11882.

\bibitem{}%[Moskowitz et~al., 1993]{moskowitz93}
Moskowitz, B.~M., Frankel, R.~B., \& Bazylinski, D.~A. (1993).
Rock magnetic criteria for the detection of biogenic magnetite.
{\it Earth Planet. Sci. Lett.}, 120(3--4), 283--300.

\bibitem{}%[Muscheler et~al., 2005]{muscheler05}
Muscheler, R., Beer, J., Kubik, P., \& Synal, H.-A. (2005).
Geomagnetic field intensity during the last 60,000 years based on $^{10}$Be and $^{36}$Cl from the Summit ice cores and $^{14}$C.
{\it Quat. Sci. Rev.}, 24, 1849--1860.

\bibitem{}%[Muttoni et~al., 2003]{muttoni03}
Muttoni, G., Kent, D.~V., Garzanti, E., Brack, P., Abrahamsen, N., \& Gaetani, M. (2003).
Early Permian Pangea ``B'' to Late permian Pangea ``A''.
{\it Earth Planet. Sci. Lett.}, 215, 379--394.

\bibitem{}%[Nagata, 1961]{nagata61}
Nagata, T. (1961).
{\it Rock Magnetism}.
Tokyo: Maruzen.

\bibitem{}%[Nagata et~al., 1963]{nagata63}
Nagata, T., Arai, Y., \& Momose, K. (1963).
Secular variation of the geomagnetic total force during the last 5000 years.
{\it J. Geophys. Res.}, 68, 5277--5282.

\bibitem{}%[Needham, 1962]{needham62}
Needham, J. (1962).
Science and civilisation in China.
In {\it Physics and Physical Technology, Part 1 Physics}, volume~4. Cambridge: Cambridge University Press.

\bibitem{}%[N\'eel, 1949]{neel49}
N\'eel, L. (1949).
Th\' eorie du trainage magn\' etique des ferromagn\' etiques en grains fines avec applications aux terres cuites.
{\it Ann. Geophys.}, 5, 99--136.

\bibitem{}%[N\'eel, 1955]{neel55}
N\'eel, L. (1955).
Some theoretical aspects of rock-magnetism.
{\it Adv. Phys.}, 4, 191--243.

\bibitem{}%[Newell et~al., 1990]{newell90}
Newell, A.~J., Dunlop, D.~J., \& Enkin, R.~J. (1990).
Temperature dependence of critical sizes, wall widths and moments in two-domain magnetite grains.
{\it Phys. Earth Planet. Inter.}, 65, 165--176.

\bibitem{}%[Nye, 1957]{nye57}
Nye, J.~F. (1957).
{\it Physical Properties of Crystals}.
Oxford: Oxford University Press.

\bibitem{}%[Ogg et~al., 1984]{ogg84}
Ogg, J.~G., Steiner, M.~B., Oloriz, F., \& Tavera, J.~M. (1984).
Jurassic magnetostratigraphy, 1. Kimmeridgian-Tithonian of Sierra Gorda and Cacabuey, southern Spain.
{\it Earth Planet Sci. Lett.}, 71, 147--162.

\bibitem{}%[Onstott, 1980]{onstott80}
Onstott, T.~C. (1980).
Application of the Bingham Distribution Function in paleomagnetic studies.
{\it J. Geophys. Res.}, 85, 1500--1510.

\bibitem{}%[Opdyke \& Channell, 1996]{opdyke96}
Opdyke, N.~D., \& Channell, J. E.~T. (1996).
{\it Magnetic Stratigraphy}. San Diego:
Academic Press.

\bibitem{}%[Opdyke et~al., 1966]{opdyke66}
Opdyke, N.~D., Glass, B., Hays, J.~D., \& Foster, J. (1966).
Paleomagnetic study of Antarctic deep-sea cores.
{\it Science}, 154, 349--357.

\bibitem{}%[O'Reilly, 1984]{oreilly84}
O'Reilly, W. (1984).
{\it Rock and Mineral Magnetism}. Glasgow:
Blackie.

\bibitem{}%[Oreskes, 2001]{oreskes01}
Oreskes, N. (2001).
{\it Plate Tectonics: An Insider's History of the Modern Theory of the Earth}.
Boulder, CO: Westview Press.

\bibitem{}%[Owens, 1974]{owens74}
Owens, W.~H. (1974).
Mathematical model studies on factors affecting the magnetic anisotropy of deformed rocks.
{\it Tectonophysics}, 24, 115--131.

\bibitem{}%[\"Ozdemir et~al., 1993]{ozdemir93}
\"Ozdemir, O., Dunlop, D.~J., \& Moskowitz, B.~M. (1993).
The effect of oxidation on the Verwey transition in magnetite.
{\it Geophys. Res. Lett.}, 20, 1671--1674.

\bibitem{}%[\"Ozdemir et~al., 1995]{ozdemir95}
\"Ozdemir, O., Xu, S., \& Dunlop, D.~J. (1995).
Closure domains in magnetite.
{\it J. Geophys. Res.}, 100, 2193--2209.

\bibitem{}%[Paquereau-Lebti et~al., 2008]{lebti08}
Paquereau-Lebti, P., Fornari, M., Roperch, P., Thouret, J.~C., \& Macedo, O. (2008).
Paleomagnetism, magnetic fabric, and Ar-40/Ar-39 dating of Pliocene and Quaternary ignimbrites in the Arequipa area, southern Peru.
{\it Bull. Volcanol.}, 70(8),\break 977--997.

\bibitem{}%[Pauthenet \& Bochinrol, 1951]{pauthenet51}
Pauthenet, R., \& Bochinrol, L. (1951).
Aimantation spontan\'ee des ferrites.
{\it J. Physique Radium}, 12, 249--251.

\bibitem{}%[Perrin \& Schnepp, 2004]{perrin04}
Perrin, M., \& Schnepp, E. (2004).
IAGA paleointensity database: distribution and quality of the data set.
{\it Phys. Earth Planet. Int.}, 147(2--3), 255--267.

\bibitem{}%[Petrovsky et~al., 2000]{petrovsky00}
Petrovsky, E., Kapicka, A., Jordanova, N., Knab, M., \& Hoffmann, V. (2000).\break
Low-field magnetic susceptibility: a proxy method of estimating increased pollution of different environmental systems.
{\it Environ. Geo.}, 39, doi:10.1007/s002540050010,\break 312--318.

\bibitem{}%[Pick \& Tauxe, 1993]{pick93}
Pick, T., \& Tauxe, L. (1993).
Geomagnetic paleointensities during the Cretaceous normal superchron measured using submarine basaltic glass.
{\it Nature}, 366, 238--242.

\bibitem{}%[Pick \& Tauxe, 1994]{pick94}
Pick, T., \& Tauxe, L. (1994).
Characteristics of magnetite in submarine basaltic glass.
{\it Geophys. J. Int.}, 119, 116--128.

\bibitem{}%[Pitman \& Heirtzler, 1966]{pitman66}
Pitman, W. C.~I., \& Heirtzler, J.~R. (1966).
Magnetic anomalies over the Pacific Antarctic ridge.
{\it Science}, 154, 1164--1171.

\bibitem{}%[Plenier et~al., 2002]{plenier02}
Plenier, G., Camps, P., Henry, B., \& Nicolaysen, K. (2002).
Palaeomagnetic study of Oligocene (24--30 Ma) lava flows from the Kerguelen Archipelago (southern Indian Ocean): directional analysis and magnetostratigraphy.
{\it Phys. Earth Planet. Int.}, 133,\break 127--146.

\bibitem{}%[Plenier et~al., 2007]{plenier07}
Plenier, G., Valet, J.~P., Gu\'erin, G., Lef\`evre, J.-C., LeGoff, M., \& Carter-Stiglitz, B. (2007).
Origin and age of the directions recorded during the Laschamp even in the Cha\^ine des Puys (France).
{\it Earth Planet. Sci. Lett.}, 259, 424--431.

\bibitem{}%[Pokorny et~al., 2004]{pokorny04}
Pokorny, J., Suza, P., \& Hrouda, F. (2004).
Anisotropy of magnetic susceptibility of rocks measured in variable weak magnetic fields using the KLY-4S Kappabridge.
In M. Hern{\'{a}}ndez et al. (Ed.), {\it Magnetic Fabric: Methods and
Applications}, volume 238 (pp.~69--76). Denver: Geology Society Special Publication.

\bibitem{}%[Potter \& Stephenson, 2005]{potter05}
Potter, D., \& Stephenson, A. (2005).
New observations and theory of single-domain magnetic moments.
{\it J. Physics: Conf. Series}, 17, 168--173.

\bibitem{}%[Pr\'evot \& Camps, 1993]{prevot93}
Pr\'evot, M., \& Camps, P. (1993).
Absence of preferred longitude sectors for poles from volcanic records of geomagnetic reversals.
{\it Nature}, 366, 53--57.

\bibitem{}%[Pr\'evot et~al., 1990]{prevot90}
Pr\'evot, M., Derder, M. E.~M., McWilliams, M., \& Thompson, J. (1990).
Intensity of the Earth's magnetic field: evidence for a Mesozoic dipole low.
{\it Earth Planet. Sci. Lett.}, 97, 129--139.

\bibitem{}%[Pullaiah et~al., 1975]{pullaiah75}
Pullaiah, G., Irving, E., Buchan, K., \& Dunlop, D. (1975).
Magnetization changes caused by burial and uplift.
{\it Earth Planet. Sci. Lett.}, 28, 133--143.

\bibitem{}%[Ramsay, 1967]{ramsay67}
Ramsay, J.~G. (1967).
{\it Folding and Fracturing of Rocks}.
New York: McGraw Hill.

\bibitem{}%[Randall \& Taylor, 1996]{randall96}
Randall, D., \& Taylor, G. (1996).
Major crustal rotations in the Andean margin: paleomagnetic results from the Coastal Cordillera of northern Chile.
{\it J. Geophys. Res.}, 101, 15783--15798.

\bibitem{}%[Reeves, 1918]{reeves18}
Reeves, E. (1918).
Halley's magnetic variation charts.
{\it Geograph. J.}, 51, 237--240.

\bibitem{}%[Reynolds et~al., 1985]{reynolds85}
Reynolds, R., Hudson, M., Fishman, N., \& Campbell, J. (1985).
Paleomagnetic and petrologic evidence bearing on the age and origin of uranium deposits in the Permian Cutler Formation, Lisbon Valley, Utah.
{\it Bull. Geol. Soc. Amer.}, 96, 719--730.

\bibitem{}%[Riisager et~al., 2005]{riisager05}
Riisager, P., Knight, K., Baker, J., Peate, I., Al-Kadasi, M., Al-Subbary, A., \& Renne, P. (2005).
Paleomagnetism and $^{40}$Ar/$^{39}$Ar geochronology of Yemeni Oligocene volcanics: implications for timing and duration of Afro-Arabian traps and geometry of the Oligocene paleomagnetic field.
{\it Earth Planet. Sci. Lett.}, 237, 647--672.

\bibitem{}%[Riisager \& Riisager, 2001]{riisager01}
Riisager, P., \& Riisager, J. (2001).
Detecting multidomain magnetic grains in Thellier palaeointensity experiments.
{\it Phys. Earth Planet. Inter.}, 125(1--4), 111--117.

\bibitem{}%[Riisager et~al., 2002]{riisager02}
Riisager, P., Riisager, J., Abrahamsen, N., \& Waagstein, R. (2002).
Thellier palaeointensity experiments on Faroes flood basalts: technical aspects and geomagnetic implications.
{\it Phys. Earth Planet. Inter.}, 131(2), 91--100.

\bibitem{}%[Roberts, 1995]{roberts95}
Roberts, A.~P. (1995).
Magnetic properties of sedimentary greigite (Fe$_3$S$_4$).
{\it Earth Planet. Sci. Lett.}, 134, 227--236.

\bibitem{}%[Roberts \& Stix, 1972]{roberts72}
Roberts, P., \& Stix, M. (1972).
$\alpha$-effect dynamos, by the Bullard-Gellman formalism.
{\it Astron. Astrophys.}, 18, 453--466.

\bibitem{}%[Robertson \& France, 1994]{robertson94}
Robertson, D.~J., \& France, D.~E. (1994).
Discrimination of remanence-carrying minerals in mixtures, using isothermal remanent magnetisation acquisition curves.
{\it Phys. Earth Planet. Int.}, 82(3--4), 223--234.

\bibitem{}%[Robinson et~al., 2004]{robinson04}
Robinson, P., Harrison, R., McEnroe, S., \& Hargraves, R. (2004).
Nature and origin of lamellar magnetism in the hematite-ilmenite series.
{\it Amer. Min.}, 89, 725--747.

\bibitem{}%[Robinson et~al., 2002]{robinson02}
Robinson, P., Harrison, R., McEnroe, S., \& Hargraves, R.~B. (2002).
Lamellar magnetism in the hematite-ilmenite series as an explanation for strong remanent magnetization.
{\it Nature}, 418, 517--520.

\bibitem{}%
Rochette, P., Fillion, G., Matt{\'{e}}i, J. L., \& Dekkers, M.
J. (1990). Magnetic transition at 30--34 Kelvin in pyrrhotite:
insight into a widespread occurrence of this mineral in rocks. {\it
Earth Planet. Sci. Lett.}, 98, 319--328.

\bibitem{}%[Rosenbaum et~al., 2002]{rosenbaum02}
Rosenbaum, G., Lister, G., \& Duboz, C. (2002).
Relative motions of Africa, Iberia and Europe during Alpine orogeny.
{\it Tectonophysics}, 359, 117--129.

\bibitem{}%[Rosenbaum et~al., 1996]{rosenbaum96}
Rosenbaum, J., Reynolds, R., Adam, D., Drexler, J., Sarna-Wojcicki, A., \& Whitney, G. (1996).
A middle Pleistocene climate record from Buck Lake, Cascade Range, southern Oregon---evidence from sediment magnetism, trace-element geochemistry, and pollen.
{\it Geol. Soc. Amer. Bull.}, 108, 1328--1341.

\bibitem{}%[Sbarbori et~al., 2008]{sbarbori08}
Sbarbori, E., Tauxe, L., Goguitchaichvili, A., Urrutia-Fucugauchi,
J., \& Bohrson, W.  (2007).
Paleomagnetic behavior of volcanic rocks from Isla Socorro, Mexico.
{\it Earth Planets and Space}, 61, 191--204.

\bibitem{}%[Schabes \& Bertram, 1988]{schabes88}
Schabes, M.~E., \& Bertram, H.~N. (1988).
Magnetization processes in ferromagnetic cubes.
{\it J. Appl. Phys.}, 64, 1347--1357.

\bibitem{}%[Scheidegger, 1965]{scheidegger65}
Scheidegger, A.~E. (1965).
On the statistics of the orientation of bedding planes, grain axes, and similar sedimentological data.
{\it U.S. Geol. Surv. Prof. Pap.}, 525-C, 164--167.

\bibitem{}%[Schlinger et~al., 1991]{schlinger91}
Schlinger, C., Veblen, D., \& Rosenbaum, J. (1991).
Magnetism and magnetic mineralogy of ash flow tuffs from Yucca Mountain, Nevada.
{\it J. Geophys. Res.}, 96, 6035--6052.

\bibitem{}%[Schnepp et~al., 2008]{schnepp08}
Schnepp, E., Worm, K., \& Scholger, R. (2008).
Improved sampling techniques for baked clay and soft sediments.
{\it Phys. Chem. Earth}, 33(6--7), 407--413.

\bibitem{}%[Schwehr \& Tauxe, 2003]{schwehr03}
Schwehr, K., \& Tauxe, L. (2003).
Characterization of soft-sediment deformation: detection of cryptoslumps\break using magnetic methods.
{\it Geology}, 31(3), 203--206.

\bibitem{}%[Selkin et~al., 2000]{selkin00b}
Selkin, P., Gee, J., Tauxe, L., Meurer, W., \& Newell, A. (2000).
The effect of remanence anisotropy on paleointensity estimates:\break a case study from the Archean Stillwater complex.
{\it Earth Planet. Sci. Lett.}, 182, 403--416.

\bibitem{}%[Selkin et~al., 2007]{selkin07}
Selkin, P., Gee, J.~S., \& Tauxe, L. (2007).
Nonlinear thermoremanence acquisition\break and implications for paleointensity data.\break
{\it Earth Planet. Sci. Lett.}, 256, 81--89.

\bibitem{}%[Selkin \& Tauxe, 2000]{selkin00}
Selkin, P., \& Tauxe, L. (2000).
Long-term variations in paleointensity.
{\it Phil. Trans. Roy. Soc. London}, 358, 1065--1088.

\bibitem{}%[Shackleton et~al., 1990]{shackleton90}
Shackleton, N.~J., Berger, A., \& Peltier, W.~R. (1990).
An alternative astronomical calibration of the lower Pleistocene timescale based on {ODP} Site 677.
{\it Trans. Roy. Soc. Edinburgh: Earth Sci.}, 81, 251--261.

\bibitem{}%[Shaw, 1974]{shaw74}
Shaw, J. (1974).
A new method of determining the magnitude of the paleomagnetic field application to five historic lavas and five archeological samples.
{\it Geophys. J. Roy. Astr. Soc.}, 39, 133--141.

\bibitem{}%[Shcherbakov \& Shcherbakova, 1983]{shcherbakov83}
Shcherbakov, V., \& Shcherbakova, V. (1983).
On the theory of depositional remanent magnetization in sedimentary rocks.
{\it Geophys. Surv.}, 5, 369--380.

\bibitem{}%[Shibuya et~al., 1992]{shibuya92}
Shibuya, H., Cassidy, J., Smith, I., \& Itaya, T. (1992).
Geomagnetic excursion in the Brunhes Epoch recorded in New Zealand basalts.
{\it Earth Planet. Sci. Lett.}, 111, 10--48.

\bibitem{}%[Si \& van~der Voo, 2001]{si01}
Si, J. \& van~der Voo, R. (2001).
Too-low magnetic inclinations in central Asia: an indication of a long-term Tertiary non-dipole field?
{\it Terra Nova}, 13, 471--478.

\bibitem{}%[Smith \& Hallam, 1970]{smith70}
Smith, A., \& Hallam, A. (1970).
The fit of the southern continents.
{\it Nature}, 225, 139--144.

\bibitem{}%[Smith, 1967]{smith67}
Smith, P.~J. (1967).
The intensity of the ancient geomagnetic field: a review and analysis.
{\it Geophys. J. Roy. Astr. Soc.}, 12,\break 321--362.

\bibitem{}%[Snowball, 1997]{snowball97}
Snowball, I. (1997).
Gyroremanent magnetization and the magnetic properties of greigite-bearing clays in southern Sweden.
{\it Geophys. J. Int.}, 129, 624--636.

\bibitem{}%
Snowball, I., \& Thompson, R. (1990). A stable chemical remanence in
Holocene sediments. {\it J. Geophys. Res.}, 95, 4471--4479.

\bibitem{}%[Snowball \& Torii, 1999]{snowball99}
Snowball, I., \& Torii, M. (1999).
Incidence and significance of magnetic iron sulphides in Quaternary sediments and soil.
In B. Maher \& R. Thompson (Eds.), {\it Quaternary Climates, Environments and Magnetism} (pp.\ 199--230). Cambridge: Cambridge University Press.

\bibitem{}%[Song \& Richards, 1996]{song96}
Song, X., \& Richards, P.~G. (1996).
Seismological evidence for differential rotation of the Earth's inner core.
{\it Nature}, 382, 221--224.

\bibitem{}%
Spender, M. R., Coey, J. M. D., \& Morrish, A. H. (1972). The
magnetic properties and Mossbauer spectra of synthetic samples of
Fe$_3$S$_4$. {\it Can. J. Phys.}, 50, 2313--2326.

\bibitem{}%[Stacey \& Banerjee, 1974]{stacey74}
Stacey, F.~D., \& Banerjee, S.~K. (1974).
{\it The Physical Principles of Rock Magnetism}, volume~5 of {\it
Developments in Solid Earth Geophysics}.
Amsterdam:
Elsevier Science Publishing Co.

\bibitem{}%[Stacey et~al., 1960]{stacey60}
Stacey, F.~D., Joplin, G., \& Lindsay, J. (1960).
Magnetic anisotropy and fabric of some foliated rocks from S.E. Australia.
{\it Geophysica Pura Appl.}, 47, 30--40.

\bibitem{}%[Stacey et~al., 1961]{stacey61}
Stacey, F.~D., Lovering, J.~F., \& Parry, L.~G. (1961).
Thermomagnetic properties, natural magnetic moments, and magnetic anisotropies of some chondritic meteorites.
{\it J. Geophys. Res.}, 66, 1523--1534.

\bibitem{}%[Steiner \& Helsley, 1975]{steiner75}
Steiner, M., \& Helsley, C. (1975).
Reversal pattern and apparent polar wander for the Late Jurassic.
{\it Geol. Soc. Amer. Bull.}, 68, 1537--1543.

\bibitem{}%[Stephenson, 1981]{stephenson81}
Stephenson, A. (1981).
Gyromagnetic remanence and anisotropy in single-domain particles, rocks, and magnetic recording tape.
{\it Phil. Mag.}, B44, 635--664.

\bibitem{}%[Stephenson, 1993]{stephenson93}
Stephenson, A. (1993).
Three-axis static alternating field demagnetization of rocks and the identification of NRM, gyroremanent magnetization, and anisotropy.
{\it J. Geophys. Res.}, 98, 373--381.

\bibitem{}%[Stephenson et~al., 1986]{stephenson86}
Stephenson, A., Sadikern, S., \& Potter, D.~K. (1986).
A theoretical and experimental comparison of the susceptibility and remanence in rocks and minerals.
{\it Geophys. J. Roy. Astr. Soc.}, 84, 185--200.

\bibitem{}%[Stokking \& Tauxe, 990b]{stokking90b}
Stokking, L., \& Tauxe, L. (1990b).
Multi-component magnetization in synthetic hematite.
{\it Phys. Earth Planet. Inter.}, 65, 109--124.

\bibitem{}%[Stokking \& Tauxe, 990a]{stokking90}
Stokking, L., \& Tauxe, L.~B. (1990a).
Properties of chemical remanence in synthetic hematite: testing theoretical predictions.
{\it J. Geophys. Res.}, 95, 12639--12652.

\bibitem{}%[Stoner \& Wohlfarth, 1948]{stoner48}
Stoner, E, C., \& Wohlfarth, W.~P. (1948).
A mechanism of magnetic hysteresis in heterogeneous alloys.
{\it Phil. Trans. Roy. Soc. London}, A240, 599--642.

\bibitem{}%[Strik et~al., 2003]{strik03}
Strik, G., Blake, T.~S., Zegers, T.~E., White, S.~H., \& Langereis, C.~G. (2003).
Palaeomagnetism of flood basalts in the Pilbara Craton, Western Australia: Late Archaean continental drift and the oldest known reversal of the geomagnetic field.
{\it J. Geophys. Res.}, 108, doi:10.1029/2003JB002475.

\bibitem{}%[Sugiura, 1979]{sugiura79}
Sugiura, N. (1979).
{ARM}, {TRM} and magnetic interactions: concentration dependence.
{\it Earth Planet. Sci. Lett.}, 42, 451--455.

\bibitem{}%[Syono \& Ishikawa, 1963]{syono63}
Syono, Y., \& Ishikawa, Y. (1963).
Magnetocrystalline anisotropy of xFe$_2$TiO$_4\cdot(1-x)$Fe$_3$O$_4$.
{\it J. Phys. Soc. Japan}, 18, 1230--1231.

\bibitem{}%[Tan et~al., 2003]{tan03}
Tan, X.~D., Kodama, K.~P., Chen, H.~L., Fang, D.~J., Sun, D.~J., \& Li, Y.~A. (2003).
Paleomagnetism and magnetic anisotropy of Cretaceous red beds from the Tarim basin, northwest China: Evidence for a rock magnetic cause of anomalously shallow paleomagnetic inclinations from central\break Asia.
{\it J. Geophys. Res.}, 108(B2), doi:10.1029/2001JB001608.

\bibitem{}%[Tanaka, 1999]{tanaka99}
Tanaka, H. (1999).
Circular asymmetry of the paleomagnetic directions observed at low latitude volcanic sites.
{\it Earth Planets Space}, 51, 1279--1286.

\bibitem{}%[Tarduno et~al., 2006]{tarduno06}
Tarduno, J., Cottrell, R., \& Smirnov, A. (2006).
The paleomagnetism of single silicate crystals: recording geomagnetic field strength during mixed polarity intervals, superchrons, and inner core growth.
{\it Rev. Geophys.}, 44, RG1002, doi:10.1029/2005RG000189.

\bibitem{}%[Tarling \& Hrouda, 1993]{tarling93}
Tarling, D.~H., \& Hrouda, F. (1993).
{\it The Magnetic Anisotropy of Rocks}.
Heidelberg: Springer.

\bibitem{}%[Tauxe, 1993]{tauxe93}
Tauxe, L. (1993).
Sedimentary records of\break relative paleointensity of the geomagnetic field: theory and practice.
{\it Rev. Geophys.}, 31, 319--354.

\bibitem{}%[Tauxe, 1998]{tauxe98}
Tauxe, L. (1998).
{\it Paleomagnetic Principles and Practice}.
Dordrecht: Kluwer Academic Publishers.

\bibitem{}%[Tauxe, 2006a]{tauxe06}
Tauxe, L. (2006a).
Depositional remanent magnetization: toward an improved theoretical and experimental foundation.
{\it Earth Planet. Sci. Lett.}, 244, 515--529.

\bibitem{}%[Tauxe, 2006b]{tauxe06b}
Tauxe, L. (2006b).
Long-term trends in paleointensity: the contribution of DSDP/ODP submarine basaltic glass collections.
{\it Phys. Earth Planet. Int.}, 156(3--4), 223--241.

\bibitem{}%[Tauxe et~al., 2002]{tauxe02}
Tauxe, L., Bertram, H., \& Seberino, C. (2002).
Physical interpretation of hysteresis loops: micromagnetic modelling of fine particle magnetite.
{\it Geochem. Geophys. Geosyst.}, 3, doi:10.1029/2001GC000280.

\bibitem{}%[Tauxe et~al., 1983a]{tauxe83b}
Tauxe, L., Besse, J., \& LaBrecque, J.~L. (1983a).
Paleolatitudes from DSDP Leg 73 sediment cores and implications for the APWP for Africa.
{\it Geophys. J. Roy. Astr. Soc.}, 73, 315--324.

\bibitem{}%[Tauxe et~al., 2003]{tauxe03b}
Tauxe, L., Constable, C., Johnson, C., Miller, W., \& Staudigel, H. (2003).
Paleomagnetism of the Southwestern U.S.A. recorded by 0--5~Ma igneous rocks.
{\it Geochem. Geophys. Geosyst.}, doi:10.1029/2002GC000343.

\bibitem{}%[Tauxe et~al., 1990]{tauxe90}
Tauxe, L., Constable, C.~G., Stokking, L.~B., \& Badgley, C. (1990).
The use of anisotropy to determine the origin of characteristic remanence in the Siwalik red beds of\break northern Pakistan.
{\it J. Geophys. Res.}, 95, 4391--4404.

\bibitem{}%[Tauxe et~al., 1998]{tauxe98b}
Tauxe, L., Gee, J., \& Staudigel, H. (1998).
Flow directions in dikes from anisotropy of magnetic susceptibility data: the bootstrap way.
{\it J. Geophys. Res.}, 103(B8), 17775--17790.

\bibitem{}%[Tauxe \& Hartl, 1997]{tauxe97}
Tauxe, L., \& Hartl, P. (1997).
11 million years of Oligocene geomagnetic field behaviour.
{\it Geophys. J. Int.}, 128, 217--229.

\bibitem{}%[Tauxe et~al., 1996a]{tauxe96}
Tauxe, L., Herbert, T., Shackleton, N.~J., \& Kok, Y.~S. (1996a).
Astronomical calibration of the Matuyama Brunhes Boundary: consequences for magnetic remanence acquisition in\break marine carbonates and the Asian loess sequences.
{\it Earth Planet. Sci. Lett.}, 140,\break 133--146.

\bibitem{}%[Tauxe \& Kent, 1984]{tauxe84}
Tauxe, L., \& Kent, D.~V. (1984).
Properties of a detrital remanence carried by hematite from study of modern river deposits and laboratory redeposition experiments.
{\it Geophys. J. Roy. Astr. Soc.}, 76, 543--561.

\bibitem{}%[Tauxe \& Kent, 2004]{tauxe04d}
Tauxe, L., \& Kent, D.~V. (2004).
A simplified statistical model for the geomagnetic field and the detection of shallow bias in paleomagnetic inclinations: was the ancient magnetic field dipolar?
In J. Channell, D. Kent, W. Lowrie, \& J. Meert (Eds.), {\it Timescales of the Paleomagnetic Field}, volume 145\break (pp.\ 101--116). Washington, D.C.: American Geophysical Union.

\bibitem{}%[Tauxe et~al., 2008]{tauxe08}
Tauxe, L., Kodama, K., \& Kent, D.~V. (2008).
Testing corrections for paleomagnetic inclination error in sedimentary rocks: a comparative approach.
{\it Phys. Earth Planet. Int.}, 169, 152--165.

\bibitem{}%[Tauxe et~al., 1991]{tauxe91}
Tauxe, L., Kylstra, N., \& Constable, C. (1991).
Bootstrap statistics for paleomagnetic data.
{\it J. Geophys. Res.}, 96, 11723--11740.

\bibitem{}%[Tauxe et~al., 2004]{tauxe04b}
Tauxe, L., Luskin, C., Selkin, P., Gans, P.~B., \& Calvert, A. (2004).
Paleomagnetic results from the Snake River Plain: contribution to the global time averaged field database.
{\it Geochem. Geophys. Geosyst.}, Q08H13, doi:10.1029/2003GC000661.

\bibitem{}%[Tauxe et~al., 1996b]{tauxe96b}
Tauxe, L., Mullender, T. A.~T., \& Pick, T. (1996b).
Potbellies, wasp-waists, and superparamagnetism in magnetic hysteresis.
{\it J. Geophys. Res.}, 101, 571--583.

\bibitem{}%[Tauxe \& Staudigel, 2004]{tauxe04}
Tauxe, L., \& Staudigel, H. (2004).
Strength of the geomagnetic field in the Cretaceous Normal Superchron: new data from submarine basaltic glass of the Troodos Ophiolite.
{\it Geochem. Geophys. Geosyst.}, 5(2), Q02H06, doi:10.1029/2003GC000635.

\bibitem{}%[Tauxe et~al., 1983b]{tauxe83}
Tauxe, L., Tucker, P., Petersen, N., \& LaBrecque, J. (1983b).
The magnetostratigraphy of Leg 73 sediments.
{\it Palaeogeogr. Palaeoclimat. Palaeoecol.}, 42, 65--90.

\bibitem{}%[Tauxe \& Watson, 1994]{tauxe94}
Tauxe, L., \& Watson, G.~S. (1994).
The fold test: an eigen analysis approach.
{\it Earth Planet. Sci. Lett.}, 122, 331--341.

\bibitem{}%[Tauxe \& Yamazaki, 2007]{tauxe07}
Tauxe, L., \& Yamazaki, T. (2007).
Paleointensities.
In M. Kono (Ed.), {\it Geomagnetism}, volume~5 of {\it Treatise on
Geophysics} (pp.\ 509--563, doi:10.1016/ B978--044452748--6/00098--5). Amsterdam: Elsevier.

\bibitem{}%[Taylor, 1982]{taylor82}
Taylor, J. (1982).
{\it An Introduction to Error Analysis: The Study of Uncertainties in Physical Measurements}.
Mill Valley, CA: University Science Books.

\bibitem{}%[Thellier \& Thellier, 1959]{thellier59}
Thellier, E., \& Thellier, O. (1959).
Sur l'intensit\' e du champ magn\' etique terrestre dans le pass\' e historique et g\' eologique.
{\it Ann. Geophys.}, 15, 285--378.

\bibitem{}%[Tipler, 1999]{tipler99}
Tipler, P. (1999).
{\it Physics for Scientists and Engineers}.
New York: W.H. Freeman.

\bibitem{}%[Tivey et~al., 2006]{tivey06}
Tivey, M., Sager, W., Lee, S.-M., \& Tominaga, M. (2006).
Origin of the Pacific Jurassic quiet zone.
{\it Geology}, 34, 789--792.

\bibitem{}%[Torsvik et~al., 2008]{torsvik08}
Torsvik, T., M\"uller, R., van~der Voo, R., Steinberger, B., \& Gaina, C. (2008).
Global plate montion frames: toward a unified model.
{\it Rev. Geophys.}, 46, RG3004, doi:10.1029/2007RG000227.

\bibitem{}%[Torsvik \& van~der Voo, 2002]{torsvik02}
Torsvik, T.~H., \& van~der Voo, R. (2002).
Refining Gondwana and Pangea paleogeography: estimates of Phanerozoic nondipole (octupole) fields.
{\it Geophys. J. Int.}, 151, 771--794.

\bibitem{}%[Valet et~al., 1998]{valet98}
Valet, J.~P., Tric, E., Herrero-Bervera, E., Meynadier, L., \& Lockwood, J.~P. (1998).
Absolute paleointensity from Hawaiian lavas younger than 35 ka.
{\it Phys. Earth Planet. Int.}, 161, 19--32.

\bibitem{}%[van~der Voo, 1981]{voo81}
van~der Voo, R. (1981).
Paleomagnetism of North America---a brief review.
{\it Paleoreconstruction of the Continents, Geodynamic Series, Amer. Geophys.}, 2, 159--176.

\bibitem{}%[van~der Voo, 1990]{voo90}
van~der Voo, R. (1990).
Phanerozoic paleomagnetic poles from Europe and North America and comparisons with continental reconstructions.
{\it Rev. Geophys.}, 28, 167--206.

\bibitem{}%[van~der Voo, 1992]{voo92}
van~der Voo, R. (1992).
Jurassic paleopole controversy: contributions from the Atlantic-bordering continents.
{\it Geology}, 20, 975--978.

\bibitem{}%[van~der Voo, 1993]{voo93}
van~der Voo, R. (1993).
{\it Paleomagnetism of the Atlantic, Tethys and Iapetus Oceans}.
Cambridge: Cambridge University Press.

\bibitem{}%[van~der Voo \& French, 1974]{voo74}
van~der Voo, R., \& French, R. (1974).
Apparent polar wandering for the Atlantic-bordering continents: Late Carboniferan to Eocene.
{\it Earth-Sci. Rev.}, 10, 99--119.

\bibitem{}%[van~der Voo \& Torsvik, 2001]{voo01}
van~der Voo, R., \& Torsvik, T.~H. (2001).
Evidence for late Paleozoic and Mesozoic non-dipole fields provides an explanation for the Pangea reconstruction problems.
{\it Earth Planet. Sci. Lett.}, 187, 71--81, doi:10.1016/S0012--821X(01)00285--0.

\bibitem{}%[Van~Dongen et~al., 1967]{vandongen67}
Van~Dongen, P., van~der Voo, R., \& Raven,\break T. (1967).
Paleomagnetic research in the\break Central Lebanon Mountains and the\break Tartous Area (Syria).
{\it Tectonophysics}, 4,\break 35--53.

\bibitem{}%[Van~Fossen \& Kent, 1993]{vanfossen93}
Van~Fossen, M., \& Kent, D.~V. (1993).
A paleomagnetic study of 143 Ma kimberlite dikes in central New York State.
{\it Geophys. J. Int.}, 113, 175--185.

\bibitem{}%[Van~Fossen \& Kent, 1992]{vanfossen92}
Van~Fossen, M.~C., \& Kent, D. (1992).
Reply to Comment on ``High-latitude paleomagnetic poles from Middle Jurassic plutons and Moat volcanics in New England and the controversy regarding Jurassic {APW} for North America'' by Butler et al., 1992.
{\it J. Geophys. Res.}, 97, 1803--1805.

\bibitem{}%[Van~Fossen \& Kent, 1990]{vanfossen90}
Van~Fossen, M.~C., \& Kent, D.~V. (1990).
High-latitude paleomagnetic poles from Middle Jurassic plutons and Moat volcanics in New England and the controversy regarding Jurassic {APW} for North America.
{\it J. Geophys. Res.}, 95, 17503--17516.

\bibitem{}%[van Hinte, 1976]{vanhinte76}
van Hinte, J. (1976).
A Cretaceous time scale.
{\it Am. Assoc. Petrol. Geolog. Bull.}, 60,\break 498--516.

\bibitem{}%[Vandamme, 1994]{vandamme94}
Vandamme, D. (1994).
A new method to determine paleosecular variation.
{\it Phys. Earth Planet. Int.}, 85, 131--142.

\bibitem{}%[Vandamme \& Courtillot, 1992]{vandamme92}
Vandamme, D., \& Courtillot, V. (1992).
Paleomagnetic constraints on the structure of the Deccan traps.
{\it Phys. Earth Planet. Inter.}, 74, 241--261.

\bibitem{}%[Vandamme et~al., 1991]{vandamme91}
Vandamme, D., Courtillot, V., Besse, J., \& Montigny, R. (1991).
Paleomagnetism and age determination of the Deccan traps (India): results of the Napur-bombay traverse and review of earlier work.
{\it Rev. Geophys.}, 29, 159--190.

\bibitem{}%[Vandenberg \& Wonders, 1976]{vandenberg76}
Vandenberg, J., \& Wonders, A. A.~H. (1976).
Paleomagnetic evidence of large fault displacement around the Po-basin.
{\it Tectonophysics}, 33, 301--320.

\bibitem{}%[Vaughn et~al., 2005]{vaughn05}
Vaughn, J., Kodama, K.~P., \& Smith, D.\break (2005).
Correction of inclination shallowing and its tectonic implications: the\break Cretaceous Perforada Formation, Baja California.
{\it Earth Planet. Sci. Lett.}, 232,\break 72--82.

\bibitem{}%[Verosub, 1977]{verosub77}
Verosub, K.~L. (1977).
Depositional and postdepositional processes in the magnetization of sediments.
{\it Rev. Geophys. Space Phys.}, 15, 129--143.

\bibitem{}%[Verosub \& Roberts, 1995]{verosub95}
Verosub, K.~L., \& Roberts, A.~P. (1995).
Environmental magnetism: past, present, and future.
{\it J. Geophys. Res.}, 100, 2175--2192.

\bibitem{}%[Vine \& Matthews, 1963]{vine63}
Vine, F.~J., \& Matthews, D.~H. (1963).
Magnetic anomalies over oceanic ridges.
{\it Nature}, 199, 947--949.

\bibitem{}%[Wagner et~al., 2000]{wagner00}
Wagner, G., Beer, J., Laj, C., Kissel, C., Masarik, J., Muscheler, R., \& Synal, H.-A. (2000).
Chlorine-36 evidence for the Mono Lake event in the Summit GRIP ice core.
{\it Earth Planet. Sci. Lett.}, 181, 1--6.

\bibitem{}%[Wagner et~al., 000b]{wagner00b}
Wagner, G., Masarik, J., Beer, J., Baumgartner, S., Imboden, D., Kubik, P., Synal, H.-A., \& Suter, M. (2000b).
Reconstruction of the geomagnetic field between 20 and 60 kyr BP from cosmogenic radionuclides in the GRIP ice core.
{\it Nucl. Instrum. Meth. B}, 172, 587--604.

\bibitem{}%[Walton et~al., 1993]{walton93}
Walton, D., Share, J., Rolph, T.~C., \& Shaw, J. (1993).
Microwave magnetisation.
{\it Geophys. Res. Lett.}, 20, 109--111.

\bibitem{}%[Wang, 1948]{wang48}
Wang, C. (1948).
Discovery and application of magnetic phenomena in China. 1. The lodestone spoon of the Han.
{\it Chinese J. Arch.}, 3, 119.

\bibitem{}%[Watson, 1983]{watson83}
Watson, G. (1983).
Large sample theory of the Langevin distributions.
{\it J. Stat. Plann. Inf.}, 8, 245--256.

\bibitem{}%[Watson, 1956a]{watson56b}
Watson, G.~S. (1956a).
Analysis of dispersion on a sphere.
{\it Mon. Not. Roy. Astr. Soc., Geophys. Suppl.}, 7, 153--159.

\bibitem{}%[Watson, 1956b]{watson56}
Watson, G.~S. (1956b).
A test for randomness of directions.
{\it Mon. Not. Roy. Astron. Soc. Geophys. Supp.}, 7, 160--161.

\bibitem{}%[Widom, 2002]{widom02}
Widom, B. (2002).
{\it Statistical Mechanics: A Concise Introduction for Chemists}.
Cambridge: Cambridge University Press.

\bibitem{}%[Williams \& Dunlop, 1995]{williams95}
Williams, W., \& Dunlop, D. (1995).
Simulation of magnetic hysteresis in pseudo-single-domain grains of magnetite.
{\it J. Geophys. Res.}, 100, 3859--3871.

\bibitem{}%[Wohlfarth, 1958]{wohlfarth58}
Wohlfarth, E.~P. (1958).
Relations between different modes of acquisition of the remanent magnetisation of ferromagnetic particles.
{\it J. App. Phys.}, 29, 595--596.

\bibitem{}%[Woodcock, 1977]{woodcock77}
Woodcock, N.~H. (1977).
Specification of fabric shapes using an eigenvalue method.
{\it Geol. Soc. Amer. Bull.}, 88, 1231--1236.

\bibitem{}%
Worm, H. U., Clark, D., \& Dekkers, M. J. (1993). Magnetic
susceptibility of pyrrhotite: grain size, field and frequency
dependence. {\it Geophys. J. Int.}, 114, 127--137.

\bibitem{}%[Yamamoto et~al., 2003]{yamamoto03}
Yamamoto, Y., Tsunakawa, H., \& Shibuya, H. (2003).
Palaeointensity study of the Hawaiian 1960 lava: implications for possible causes of erroneously high intensities.
{\it Geophys. J. Int.}, 153(1), 263--276.

\bibitem{}%[Yamazaki \& Ioka, 1997]{yamazaki97}
Yamazaki, T., \& Ioka, N. (1997).
Environmental rock-magnetism of pelagic clay: implications for Asian eolian input to the North Pacific since the Pliocene.
{\it Paleoceanography}, 12, 111--124.

\bibitem{}%[York, 1966]{york66}
York, D. (1966).
Least-squares fitting of a straight line.
{\it Can. J. Phys.}, 44, 1079--1086.

\bibitem{}%[Yu \& Tauxe, 2005]{yu05b}
Yu, Y., \& Tauxe, L. (2005).
On the use of magnetic transient hysteresis in paleomagnetism for granulometry.
{\it Geochem. Geophys. Geosyst.}, 6, Q01H14, doi:10.1029/2004GC000839.

\bibitem{}%[Yu et~al., 2004]{yu04}
Yu, Y., Tauxe, L., \& Genevey, A. (2004).
Toward an optimal geomagnetic field intensity determination technique.
{\it Geochem. Geophys. Geosyst.}, 5(2), Q02H07, doi:10.1029/2003GC000630.

\bibitem{}%[Yukutake, 1967]{yukutake67}
Yukutake, T. (1967).
The westward drift of the Earth's magnetic field in historic times.
{\it J. Geomag. Geoelectr.}, 19, 103--116.

\bibitem{}%[Zijderveld, 1967]{zijderveld67}
Zijderveld, J. D.~A. (1967).
A.C. demagnetization of rocks: analysis of results.
In D. Collinson, K. Creer, \& S. Runcorn (Eds.), {\it Methods in Paleomagnetism} (pp.\ 254--286). Amsterdam: Elsevier.

\bibitem{}%[Zimmerman et~al., 2006]{zimmerman06}
Zimmerman, S., Hemming, S., Kent, D., \& Searle, S. (2006).
Revised chronology for late Pleistocene Mono Lake sediments based on paleointensity correlation to the global reference curve.
{\it Earth Planet. Sci. Lett.}, 252, 94--106.

\end{thebibliography}

\end{document}