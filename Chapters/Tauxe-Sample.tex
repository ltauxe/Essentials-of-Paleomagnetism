\documentclass[draft,plain]{tauxe}

\def\draftnote{\relax}
\begin{document}

\frontmatter

\slugline{1}{1}{1}

\halftitle{Essentials of Paleomagnetism}

\seriestext{The publisher gratefully acknowledges\\
the generous support provided by\\
the Humanities Endowment Fund of the\\
University of California Press Foundation\\
and by Edmund and Jeannie Kaufman\\
as members of the Literati Circle of the\\
University of California Press Foundation.}

\fulltitle{Essentials of Paleomagnetism}{lisa tauxe}{With Contributions from\\
Subir K. Banerjee, Robert F. Butler and Rob van der Voo}

\secmax{2}
\subsectmax{2.4}

\tableofcontents

\chapter*{Preface}


\section*{Purpose of the book}

The geomagnetic field acts both as an umbrella, shielding us from cosmic radiation and as a window,
offering one of the few glimpses of the inner workings of the Earth. Ancient records of the
geomagnetic field can inform us about geodynamics of the early Earth and changes in boundary
conditions through time. Thanks to its essentially dipolar nature, the geomagnetic field has acted as
a guide, pointing to the axis of rotation thereby providing latitudinal information for both explorers
and geologists.

Human measurements of the geomagnetic field date to about a millenium and are
quite sparse prior to about 400 years ago. Knowledge of what the field has done in the past relies
on accidental records carried by geological and archaeological materials. Teasing out meaningful
information from such materials requires an understanding of the fields of rock magnetism and paleomagnetism,
the subjects of this book. Rock and paleomagnetic data are useful in many applications
in Earth Science in addition to the study of the ancient geomagnetic field. This book attempts
to draw together essential rock magnetic theory and useful paleomagnetic techniques in a consistent
and up-to-date manner. It was written for several categories of readers:
\begin{bulletlist}
\item Earth scientists who use paleomagnetic data in their research
\item students taking a class with paleomagnetic content
\item other professionals with an interest in evaluating or using paleomagnetic data
\item anyone with at least college level chemistry, physics and a cursory knowledge of Earth
science with an interest in magnetism in the Earth
\end{bulletlist}

There are a number of excellent references on paleomagnetism and on the related specialties
(rock magnetism and geomagnetism). The ever popular but now out of print text by Butler (1992)
has largely been incorporated into the present text. For in-depth coverage of rock magnetism, we
recommend Dunlop and \"{O}zdemir (1997). Similarly for geomagnetism, please see Backus et al.
[1996]. A rigorous analysis of the statistics of spherical data is given by Fisher et al. (1987). The de\-tails
of paleomagnetic poles are covered in van der Voo (1993) and magnetostratigraphy is covered
in depth by Opdyke and Channell (1996). The Treatise in Geophysics, vol. 5 (edited by Kono, 2007)
and The Encyclopedia of Geomagnetism and Paleomagnetism (edited by Gubbins and Herrero-
Bervera, 2007) have up to date reviews of many topics covered in this book. The present book is\vspace*{-12pt}\break
%intended to augment or distill information from the broad field of paleomagnetism, complementing

\mainmatter

\setcounter{chapter}{1}

\title{The Geomagnetic Field}

\maketitle

One of the major efforts in paleomagnetism has been to study ancient geomagneticfields. Because
human measurements extend back about a millenium, measurement of ``accidental'' records provided
by archaeological or geological materials remains the only way to investigate ancient field
behavior. Therefore it is useful for students of paleomagnetism to understand something about the
present geomagnetic field. In thischapter we review the general properties of the Earth's magnetic
field.

The part of the geomagnetic field of interest to paleomagnetists is generated by convection currents
in the liquid outer core of the Earth which is composed of iron, nickel and some unknown lighter
component(s). The source of energy for this convection is not known for certain, but is thought to be
partly from cooling of the core and partly from the bouyancy of component(s). The source of energy
for this convection is not known for certain, but is thought to the iron/nickel liquid outer core caused
by freezing out of the pure iron inner core. Motions of this conducting fluid are controlled by the
bouyancy of the liquid, the spin of the Earth about its axis and by the interaction of the conducting fluid
with the magnetic field (in a horribly non-linear fashion). Solving the equations for the fluid
motions and resulting magnetic fields is a challenging computational task. Recent numerical models,
however, show that such magnetohydrodynamical systems can produce self-sustaining dynamos
which create enormous external magnetic fields.

\section{Components of magnetic vectors}

The magnetic field of a dipole aligned along the spin axis and centered in the Earth (also-called
{\it geocentric axial dipole}, or GAD) is shown in Figure 2.1a. [See Chapter 1 for a derivation of how
to find the radial and tangential components of such a field.] By convention, the sign of the Earth's
dipole is negative, pointing toward the south pole as shown dipole is negative, pointing towards~s
shown dipole is negative, pointing toward the dipole is negative, pointing toward the south pole
as shown dipole is negative, pointing toward the south pole as shown in Figure 2.1a and magnetic
field lines in Figure 2.1a and magnetic field lines point toward the north pole. They point downward
in the northern hemisphere and upward in the southern hemisphere.

Although dominantly dipolar, the geomagnetic field is not perfectly modeled bya geocentric axial dipole, but is somewhat more complicated (see Figure 2.1b). At the point on the surface labeled `P', the geomagnetic field points
nearly north and down at an angle of approximately 60$^\circ$.\vspace*{-12pt}\break

The dipole formula allows us to convert a given measurement of I to an equivalent {\it magnetic
co-latitude $\theta$m} If the field were a simple GAD field, $\theta$m would be a reasonable estimate of $\theta$, but non-GAD terms can invalidate this assumption. To get a feel for the effect of these non- GAD terms,
we consider first what would happen if we took random measurements of the Earth's present field
(see Figure 2.7). We evaluated the directions of the magnetic field using the IGRF for 2005 at 200
positions on the globe (shown in Figure 2.7a).

These directions are plotted in Figure 2.7b using the paleomagnetic convention of open symbols
pointing up and closed symbols pointing down. In Figure 2.7c, we plot the inclinations as a
function of latitude. As expected from a predominantly dipolar field, inclinations cluster around the
values for a geocentric axial dipolar field but there is considerable scatter and interestingly the scatter
is larger in the southern hemisphere than in the northern one. This is related to the low intensities
beneath South America and the Atlantic region seen in Figure 2.5a.

\begin{figure}
\setcounter{chapter}{14}\setcounter{figure}{11}
\centerline{\fbox{\vbox to 156pt{\hbox to 354pt{}}}}
\caption{Barcode: The Geomagnetic Polarity Time Scale (GPTS) for the last 160 Ma (Berggren et al., 1995; Gradstein et al., 1995). Line traces the reversal frequency (number of reversals in a four million year interval) estimated by Constable (2003).}
\end{figure}

\setcounter{chapter}{2}\setcounter{section}{4}

\subsection{\textit{D, `I'} Transformation}

Often we wish to compare directions from distant parts of the globe. There is an inherent difficulty
in doing so because of the large variability in inclination with latitude. In such cases it is appropriate
to consider the data relative to the expected direction (from GAD) at each sampling site. For this
purpose, it is useful to use a transformation whereby each direction is rotated such that the direction
expected from a geocentric axial dipole field (GAD) at the sampling site is the center of the equal
area projection.

This is accomplished as follows: where I{\it d} = the inclination expected from a GAD field (tan
\hbox{I{\it d} = 2tan$\gamma$}), $\gamma$ is the site latitude, and $\alpha$ is the inclination of the paleofield vector projected onto the N-S plane ($\alpha$ = $\tan-1({\rm x}3/{\rm x}1)$).
The {\it x i} are then converted to {\it D, I} by Equation 2.4. In Figure 2.8a we
show the geomagnetic field vectors evaluated at random longitudes along a latitude band of 45f{l}N.
The vectors are shown in their Cartesian coordinates of North, East and Down. In Figure 2.8b we
show what happens when we rotate the coordinate system to peer down the direction expected from
an axial dipolar field at 45N (which has an inclination of 63). The vectors circle about the expected
direction. Finally, we see what happens to the directions shown in Figure 2.7b after the {\it D, I}
transformation in Figure 2.8. These are unit vectors projected along the expected

We are often interested in whether the geomagnetic pole has changed, or whether a
particular piece of crust has rotated with respect to the geomagnetic pole. Yet, what we observe
at a particular location is the local direction of the field vector. Thus, we need a way to transform
an observed direction into the equivalent geomagnetic pole. In order to remove the dependence of
direction merely on position on the globe, we imagine a geocentric dipole which would give rise to
the observed magnetic field direction at a given latitude ($\gamma$) and longitude ($\phi$). The virtual geomagnetic
pole (VGP) is the point on the globe that corresponds to the geomagnetic pole of this imaginary
dipole (Figure 2.9a). Paleomagnetists use the following conventions: $\phi$ is measured positive
eastward from the Greenwich meridian and ranges from 0 - 360; $\theta$ is measured from the North
pole and goes from 0 - 180. Of course $\theta$ relates to latitude, $\gamma$ by $\theta$ = $90-\gamma$. $\theta$m is the magnetic
co-latitude and is given by Equation 2.12. Be sure not to confuse latitudes and co-latitudes. Also, be
careful with declination. Declinations between 180 and
360 are equivalent to D - 360 which are counter-clockwise with respect to North.

The first step in the problem of calculating a VGP is to determine the magnetic co-latitude $\theta$m
by Equation 2.12 which is defined in the dipole formula (Equation 2.12). The declination D is the
angle from the geographic North Pole to the great circle joining the observation site S and the pole
P, and $\delta\phi$ is the difference in longitudes between P and S, $\phi$p - $\phi$s. Now we use some tricks from
spherical trigonometry as reviewed in Appendix A.3.1

\setcounter{subsection}{2}

\subsection{Virtual dipole moment}

As pointed out earlier, magnetic intensity varies over the globe in a similar manner to inclination.
It is often convenient to express paleointensity values in terms of the equivalent geocentric dipole
moment that would have produced the observed intensity at a specific (paleo)latitude. Such an
equivalent moment is called the {\it  virtual dipole moment} (VDM) by analogy to the VGP (see Figure~2.9a). First, the magnetic (paleo)co-latitude $\theta$m is calculated as before from the observed inclination
and the dipole formula of Equation 2.10.

\begin{problem}
For this problem set, you will need the PmagPy package. Refer to Appendix F.3 for
help in downloading and installing it.

\subproblem*{Problem 1}

\begin{enumerate}
\item[a)] Write a python program that converts declination, inclination and intensity to North,\break
East, and Down (see Appendix F.1 for a brief tutorial on python programming).

\item[b)] Choose 10 random spots on the surface of the earth. Use the PmagPy program\break
igrf.py (see Appendix F.3.3 for an example) to evaluate the declination, inclination and\break intensity at each of these locations in January 2006. As with all PmagPy programs,\break
open a terminal window (called command prompt in Windows) and type the program
\end{enumerate}
\end{problem}

\backmatter

\appendix

\title{Definitions, derivations and tricks}

\maketitle

Paleomagnetism is famous for its use of a large number of incomprehensible acronyms. Here we
have them gathered together along with definitions and the Section numbers where they are explained
in more detail. You will find here a table of physical constants and paleomagnetic parameters
used in the text as well as a table listing common statis eters used in the text as well as a table
listing common statistics used in paleomagnetism. After the tics used in paleomagnetism. After the
tables, there are a few sections with useful mathematical tricks.

\section{Definitions}

\begin{table}
\vspace*{-20pt}
\tbl{Acronyms in paleomagnetism\break
TITLRTITLOIDFJGNISHDCNSUIDLINE2TITLELINE2TITLELINE2}
{\begin{tabular*}{\textwidth}{@{}ll}
\toprule
Acronym  & Definition: Section \# \\
\colrule
AMS	&Anisotropy of magnetic susceptibility: Section 13.1\\
APWP &Apparent polar wander path: Section 16.2\\
AF &Alternating field demagnetization: Section 9.4\\
ARM &Anhysteretic remanent magnetization: Section 7.10\\
ChRM &Characteristic remanent magnetization: Section 9.5\\
CNS	&Cretaceous Normal Superchron: Section 15.1\\
CRM &Chemical remanent magnetization: Section 7.5\\
DGRF &Definitive geomagnetic reference field: Section 2.2\\
DRM &Detrital remanent magnetization: Section 7.6\\
E/I &Elongation/inclination correction method: Section 16.4\\
FC &Field cooled: Section 8.8.4\\
GAD &Geocentric axial dipole: Section 2.3\\
GHA &Greenwich hour angle: Appendix A.3.8\\
GPTS &Geomagnetic polarity time scale: Chapter 15\\
GRM &Gyroremanent magnetization: Section 7.10\\
\botrule
\end{tabular*}}
{Note: 1 H = kg m2A$-$2s$-$2, 1 emu = 1 G cm3, B = $\mu$oH (in vacuum), 1 T = kg A$-$1 s$-$2}
\end{table}

\begin{thebibliography}{}
\bibitem{} Ozdemir, O. and Dunlop, D. J., J. Geophys. Res, 102,
20211-20224, 1997
\bibitem{} Ozdemir, O. and Dunlop, D. J. and Moskowitz, B. M.,
The effect of oxidation on the Verwey transition
in magnetite, Geophys. Res. Lett., 20, 1671-1674,
1993
\bibitem{} Ozdemir, O. and Xu, S. and Dunlop, D. J., Closure
domains in magnetite, J. Geophys. Res., 100, 2193-
2209, 1995
\bibitem{} Abramowitz, M. and Stegun, I. A., Handbook of
Mathematical Functions, National Bureau of Standards,
Washington, DC, 55, Applied Mathematics
Series, 1970
\bibitem{} C.J., Archaeomagnetic determination of the past geomagnetic
intensity using ancient ceramics: allowance
for anisotropy, Archaeometry, 23, 53-64, 1981
\bibitem{} Aitken, M. J. and Allsop, A. L. and Bussell, G. D. and
Winter, M. B., Determination of the intensity of the
Earth's magnetic field during archeological times:
reliability of the Thellier technique, Rev. Geophys.,
26, 3-12, 1988
\bibitem{} Alvarez, W. and Arthur, M. A. and Fischer, A. G. and
Lowrie, W. and Napoleone, G. and Premoli-Silva,
I.~and Roggenthen, W. M., Type section for the Late
Cretaceous-Paleocene reversal time scale, Geol.
Soc. Amer. Bull., 88 383-389, 1977
\bibitem{} Anonymous, Magnetostratigraphic polarity units- a
supplementary chapter of the ISC International
stratigraphic guide, Geology, 7, 578-583, 1979
\bibitem{} Anson, G. L. and Kodama, K. P., Compaction-induced
inclination shallowing of the post-depositional
remanent magnetization in a synthetic sediment,
Geophys. J. R. astr. Soc., 88, 673-692, 1987
\bibitem{} Aurnou, J. and Andreadis, S. and Zhu, L. and Olson,
P., Experiments on convection in Earth's core
tangent cylinder, Earth Planet. Sci. Lett., 212,
1-2,119-134,2003
\bibitem{} backus96, Backus, G. and Parker, R. L. and Constable,
C., Foundations of geomagnetism, Cambridge
University Press, Cambridge,1996
\bibitem{} Balsley, J. R. and Buddington, A. F., Magnetic
susceptibility anisotropy and fabric of some Adirondack
granites and orthogneisses,Amer. Jour.
Sci.,258A,6-20, 1960
\bibitem{} Banerjee, S. K., New grain
size limits for paleomagnetic stability in hematite,
Nature Phys. Sci., 232, 15-16, 1971
Banerjee, S. K., Magnetic properties of Fe-Ti oxides,
Book Oxide Minerals: Petrologic and Magnetic
Significance, Lindsley, D. H., Reviews in Mineralogy,
Mineralogical Society of America, Washington, 1991
\bibitem{} Banerjee, S. K. and King, J. and Marvin, J., A rapid
method for magnetic granulometry with applications
to environmental studies, Geophys. Res. Lett.,
8, 333-336, 1981
\bibitem{} Ben-Yosef, E. and Tauxe, L. and Ron, H. and Agnon,
A. and Avner, U. and Najjar, M. and Levy, T.E.,
A~new approach for geomagnetic archeointensity
research: insights on ancient matellurgy in the
Southern Levant, J. Archaelogical Science, 35,
2863-2879, 2008
\bibitem{} Berggren, W.A. and Kent, D.V. and Swisher II, C.C.
and Aubry, M.-P., A Revised Cenozoic Geochronology
and Chronostratigraphy, Book Geochronology
Time Scales and global Stratigraphic Correlation,
Berggren, W.A. and Kent, D.V. and Aubry, M.-P.
\end{thebibliography}

\begin{theindex}
\item Abramowitz, M., 378
\item Aitken, M.J., 195, 199, 265
\item Alvarez, W., 305
\item Amp\'{e}re's law, 2, 3
\item Anson, G.L., 122
\item Argyle, K.S., 193, 205
\item Aurnou, J., 289
\item B\"{o}hnel, H., 202
\item Banerjee, S.K., 44, 55, 62, 63, 146, 147, 148, 268
\item Behrensmeyer, A.K., 312
\item Ben-Yosef, E., 198, 205
\item Bertram, H.N., 58
\item Besse, J., 323, 327, 328, 336, 337, 340, 340
\item Biggin, A., 202
\item Bingham
\item Bingham, C., 237
\item Bitter, F., 59
\item Bloxham, J., 120
\item Bohr magneton, 36, 37, 39, 88
\item Bol'shakov, A.S., 196
\item Boltzmann's distribution law, 103, 106, 110
\item Bonhommet, N., 283
\item Borradaile, G.J., 238, 263
\item Brunhes, B., 277
\item Bullard fit, 321, 336
\item Bullard, E.C., 321, 336
\item Busse, F.H., 289
\item Butler, R.F., 62, 63, 324, 326, 338, 340
\item Cande, S.C., 302, 303, 308, 308
\item Cassata, W.S., 283
\item Cassidy, J., 283
\item Channell, J.E.T., 282, 283, 284, 304, 309, 310, 339
\item Chisolm, L., 340
\item Clement, B.M., 285, 285
\item Coe, R.S., 195
\item Coffey, W.T., 120
\item Collinson, D.W., 118, 119
\item Constable, C.G., 171, 238, 253, 256, 257, 278, 279,
\item 279, 287, 294
\item Coulomb's Law, 7
\item Courtillot, V., 323, 327, 328, 336, 337, 340, 340
\item Cox, A., 221, 278, 300, 300, 300, 301, 301
\item Creer, K.M., 236, 292
\item Cretaceous quiet zone, 303
\item Cullity, B.D., 7
\item Curie temperature, 42, 43, 44, 89, 109, 133
\subitem estimation, 133
\subitem Moskowitz method, 134
\subitem differential method, 134
\subitem intersecting tangents method, 134
\item Curie's Law, 40
\item Curie-Weiss law, 43
\item Dalrymple, G.B., 303
\item David, P., 277
\item Day, R., 76, 78, 146
\item DeMets, C., 320
\item Deamer, G.A., 120
\item Dekkers, M., 202
\item Doell, R.R., 221, 303
\item Dunlop, D.J., 44, 55, 57, 58, 76, 76, 77, 79, 92, 127,
145, 193, 196, 196, 198, 198, 202, 205
\item Egli, R., 141
\item Euler rotations, 321
\item Evans, M.E., 62
\item Fabian, K.L., 63, 82
\item Faraday's Law, 4
\item Feinberg, J.M., 80, 86
\item Field models
\subitem GUFM1, 276
\subitem ICALSxK.n, 279
\subitem IDGRF, 22
\subitem IGRF, 22
\subitem Model A, 292
\subitem Model G, 294
\end{theindex}

\chapter*{Other Styles}

Now we briefly describe essential *NIX commands. Some of these are also available with the unix-dos command set.

\subproblem*{Example 1 aarm magic.py [Chapter 13 \& MagIC: see Apendix E]}

Anisotropy of anhysteretic or other remanence can be converted to a tensor and used to correct
natural remanence data for the effects of anisotropy remanence acquisition. For example, directions
may be deflected from the geomagnetic field direction or intensities may be biased by strong anisotropies
in the magnetic fabric of the specimen. By imparting an anhysteretic or thermal remanence
in many specific orientations, the anisotropy of remanence acquisition can be characterized and
used for correction. We do this for anisotropy of anhysteretic remanence (aarm) by imparting an
ARM in 9, 12 or 15 positions. Each ARM must be preceded by an AF demagnetization step. The 15
positions are given in Appendix D.1. For the 9 position scheme, AARMs are imparted in positions
1,2,3, 6,7,8, 11,12,13, for example. Someone has kindly made the measurements, converted them
to magic measurements format (see Appendix E) and placed them in the file: aarm measurements.
txt. [One way to do this for yourself is to use the program mag magic.py.]

\begin{ulist}
\item[] \%mag\_magic.py -f bg.arm -loc Bushveld -LP AF:ANI -F aarm\_measurements.txt
-ncn 3 -ac 180 -dc 50 0 90
\item[] Warning - inconsistency in mag file with lab field d - overriding file with 0
\% aarm\_magic.py -f aarm\_measurements.txt
\end{ulist}

The geomagnetic field acts both as an umbrella, shielding us from cosmic radiation and as a window,
offering one of the few glimpses of the inner workings of the Earth. Ancient records of the
geomagnetic field can inform us about geodynamics of the early Earth and changes in boundary
conditions through time. Thanks to its essentially dipolar nature, the geomagnetic field has acted as
a guide, pointing to the axis of rotation thereby providing latitudinal information for both explorers
and geologists. Human measurements of the geomagnetic field date to about a millenium and are
quite sparse prior to about 400 years ago. Knowledge of what the field has done in the past relies
on accidental records carried by geological and archaeological materials. Teasing out meaningful
information from such materials requires an understanding of the fields of rock magnetism and paleomagnetism,
the subjects of this book. Rock and paleomagnetic data are useful in many applications
in Earth Science in addition to the study of the ancient geomagnetic field. This book attempts
to draw together essential rock magnetic theory and useful paleomagnetic techniques in a consistent
and up-to-date manner. It was written for several categories of readers:
\begin{bulletlist}
\item Earth scientists who use paleomagnetic data in their research
\item students taking a class with paleomagnetic content
\item other professionals with an interest in evaluating or using paleomagnetic data
\item anyone with at least college level chemistry, physics and a cursory knowledge of Earth
science with an interest in magnetism in the Earth
\end{bulletlist}

There are a number of excellent references on paleomagnetism and on the related specialties
(rock magnetism and geomagnetism). The ever popular but now out of print text by Butler (1992)
has largely been incorporated into the present text.
\begin{enumerate}
\item[1.] Earth scientists who use paleomagnetic data in their research
\item[2.] students taking a class with paleomagnetic content
\item[3.] other professionals with an interest in evaluating or using paleomagnetic data
\item[4.] anyone with at least college level chemistry, physics and a cursory knowledge of Earth
science with an interest in magnetism in the Earth
\end{enumerate}
For in-depth coverage of rock magnetism, we
recommend Dunlop and \"{O}zdemir (1997). Similarly for geomagnetism, please see Backus et al.
[1996].

\section*{Equation Samples}

The geomagnetic field acts both as an umbrella, shielding us from cosmic radiation and as a window,
offering one of the few glimpses of the$\Delta=\alpha+\beta+\sqrt{x-1}$ inner workings of the Earth. Ancient records of the
geomagnetic field can inform us about geodynamics of the early Earth and changes in boundary
conditions through time. Thanks to its essentially dipolar nature, the geomagnetic field has acted as
a guide, pointing to the axis of rotation thereby providing latitudinal information for both explorers
and geologists. Human measurements of the geomagnetic field
\begin{equation}%eqn1
a_1+a_2=-\frac{b}{a} \quad {\rm and} \quad a_1a_2=\frac{c}{a}.
\end{equation}
date to about a millenium and are
quite sparse prior to about 400 years ago. Knowledge of what the field has done in the past relies
on accidental records carried by geological and archaeological materials. Teasing out meaningful
information from such materials requires an understanding of the fields of rock magnetism and paleomagnetism,
the subjects of this book
\begin{eqnarray}%eqn2
a(x-a_1)(x-a_2)&=a(x^2-a_1x-a_2x+a_1a_2)\\
&=a[x^2-(a_1+a_2)x+a_1a_2].
\end{eqnarray}
Rock and paleomagnetic data are useful in many applications
in Earth Science in addition to the study of the ancient geomagnetic field. This book attempts
to draw together essential rock magnetic theory and useful paleomagnetic techniques in a consistent
and up-to-date manner.

\end{document} 